% Basic settings for this card set
\renewcommand{\cardcolor}{adventures}
\renewcommand{\cardextension}{Erweiterung VIII}
\renewcommand{\cardextensiontitle}{Abenteuer}

\clearpage
\newpage
\section{\cardextension \ - \cardextensiontitle}

\begin{tikzpicture}
	\card
	\cardstrip
	\cardbanner{banner/goldlightbrown.png}
	\cardicon{banner/coin.png}
	\cardprice{2}
	\cardtitle{\scriptsize{Königliche Münzen}}
	\cardcontent{Diese Karte ist eine kombinierte Geld- und Reservekarte. Sie hat den Wert 1 Geld. Spiele sie in der Kaufphase aus und lege sie anschließend auf das Wirtshaustableau. Sobald du eine Aktionskarte ausspielst und die Anweisungen darauf ausgeführt hast, darfst du die Königlichen Münzen von deinem Wirtshaustableau aufrufen. Lege die Karte in deinen Spielbereich und du erhältst +2 Aktionen, nicht aber +1 Geld, da diese Anweisung nur beim Ausspielen der Karte ausgeführt wird. Bis zum Ende der Aufräumphase befinden sich die Königlichen Münzen im Spiel und werden dann abgelegt.}
\end{tikzpicture}
\hspace{-1cm}
\begin{tikzpicture}
	\card
	\cardstrip
	\cardbanner{banner/lightbrown.png}
	\cardicon{banner/coin.png}
	\cardprice{2}
	\cardtitle{Rattenfänger}
	\cardcontent{Du erhältst +1 Karte und +1 Aktion. Lege diese Karte auf dein Wirtshaustableau. Du darfst diese Karte zu Beginn deines Zuges vom Tableau aufrufen. Wenn du das tust, lege die Karte in deinen Spielbereich. Dann musst du eine beliebige Karte aus deiner Hand entsorgen. Lege den Rattenfänger in der Aufräumphase ab.}
\end{tikzpicture}
\hspace{-1cm}
\begin{tikzpicture}
	\card
	\cardstrip
	\cardbanner{banner/white.png}
	\cardicon{banner/coin.png}
	\cardprice{2}
	\cardtitle{Zerstörung}
	\cardcontent{Du erhältst +1 Aktion. Dann musst du entweder diese Zerstörung oder eine Handkarte entsorgen. Je nach Höhe der Kosten der entsorgten Karte siehst du dir die entsprechende Anzahl Karten oben von deinem Nachziehstapel an. Nimm eine davon auf die Hand und lege den Rest ab. Liegt der -1 Karte Marker auf dem Nachziehstapel, wird dieser kurz zur Seite gelegt und nach dem Ziehen der Karten wieder oben auf den Stapel zurückgelegt.}
\end{tikzpicture}
\hspace{-1cm}
\begin{tikzpicture}
	\card
	\cardstrip
	\cardbanner{banner/orange.png}
	\cardicon{banner/coin.png}
	\cardprice{3}
	\cardtitle{Amulett}
	\cardcontent{Diese Karte ist eine Dauerkarte. Wähle sowohl in diesem als auch zu Beginn deines nächsten Zuges jeweils eine der folgenden Optionen: +1 Geld oder entsorge eine Handkarte oder nimm ein Silber vom Vorrat. Du darfst beim 2. Mal etwas anderes wählen. Lege die Karte in der Aufräumphase des nächsten Zuges ab.}
\end{tikzpicture}
\hspace{-1cm}
\begin{tikzpicture}
	\card
	\cardstrip
	\cardbanner{banner/orange.png}
	\cardicon{banner/coin.png}
	\cardprice{3}
	\cardtitle{Ausrüstung}
	\cardcontent{Diese Karte ist eine Dauerkarte. Du erhältst +2 Karten und legst dann bis zu 2 Karten (inkl. eventuell gerade gezogener Karten) verdeckt zur Seite. Entscheidest du dich dafür, keine Karten zur Seite zu legen, legst du die Ausrüstung am Ende des Zuges ab.
	\\
	\medskip
	\\
	Legst du 1 oder 2 Karten zur Seite, nimmst du zu Beginn deines nächsten Zuges die zur Seite gelegten Karten auf die Hand. Lege die Ausrüstung am Ende dieses Zuges ab.}
\end{tikzpicture}
\hspace{-1cm}
\begin{tikzpicture}
	\card
	\cardstrip
	\cardbanner{banner/orangeblue.png}
	\cardicon{banner/coin.png}
	\cardprice{3}
	\cardtitle{\scriptsize{Karawanenwächter}}
	\cardcontent{Diese Karte ist eine kombinierte Dauer- und Reaktionskarte. Wenn du sie ausspielst, erhältst du +1 Karte und +1 Aktion.
	\\
	\medskip
	\\
	Zu Beginn deines nächsten Zuges erhältst du +1 Geld. Lege die Karte am Ende dieses Zuges ab.
	\\
	\medskip
	\\
	Wenn ein Mitspieler eine Angriffskarte ausspielt, darfst du diese Karte aus deiner Hand ausspielen. Du erhältst +1 Karte und +1 Aktion. Die Anweisung +1 Aktion kann in diesem Fall – da gerade ein anderer Spieler am Zug ist – nicht ausgeführt werden. Zu Beginn deines nächsten Zuges erhältst du +1 Geld. Die Karte wird erst in der Aufräumphase deines nächsten Zuges abgelegt.}
\end{tikzpicture}
\hspace{-1cm}
\begin{tikzpicture}
	\card
	\cardstrip
	\cardbanner{banner/lightbrown.png}
	\cardicon{banner/coin.png}
	\cardprice{3}
	\cardtitle{Kundschafter}
	\cardcontent{Diese Karte ist eine Reservekarte. Wenn du sie ausspielst erhältst du +1 Karte und +1 Aktion. Dann legst du die Karte auf dein Wirtshaustableau.
	\\
	\medskip
	\\
	Zu Beginn deines Zuges darfst du diese Karte aufrufen. Wenn du das tust, legst du alle Karten, die du auf der Hand hast, ab und ziehst 5 Karten nach. Lege die Karte am Ende der Aufräumphase ab.}
\end{tikzpicture}
\hspace{-1cm}
\begin{tikzpicture}
	\card
	\cardstrip
	\cardbanner{banner/orange.png}
	\cardicon{banner/coin.png}
	\cardprice{3}
	\cardtitle{Verlies}
	\cardcontent{Diese Karte ist eine Dauerkarte. Du erhältst +2 Karten, +1 Aktion und legst 2 Karten ab.
	\\
	\medskip
	\\
	Zu Beginn deines nächsten Zuges erhältst du +2 Karten und legst 2 Karten ab. Lege das Verlies am Ende dieses Zuges ab.}
\end{tikzpicture}
\hspace{-1cm}
\begin{tikzpicture}
	\card
	\cardstrip
	\cardbanner{banner/lightbrown.png}
	\cardicon{banner/coin.png}
	\cardprice{4}
	\cardtitle{Duplikat}
	\cardcontent{Diese Karte ist eine Reservekarte. Wenn du sie ausspielst, legst du sie auf dein Wirtshaustableau.
	\\
	\medskip
	\\
	Sobald du (auch außerhalb deines eigenen Zuges) eine Karte nimmst, die bis zu 6 Geld kostet, darfst du diese Karte aufrufen. Wenn du das tust, nimmst du dir eine Karte mit gleichem Namen vom Vorrat. Lege diese Karte ab. Wenn keine solche Karte mehr im Vorrat ist, erhältst du nichts. Wenn du eine Karte von einem Stapel nimmst, der nicht zum Vorrat gehört, darfst du dir keine weitere Karte mit gleichem Namen nehmen. Das Duplikat wird in der Aufräumphase des Zuges, in dem es aufgerufen wird (auch außerhalb deines Zuges) abgelegt.}
\end{tikzpicture}
\hspace{-1cm}
\begin{tikzpicture}
	\card
	\cardstrip
	\cardbanner{banner/white.png}
	\cardicon{banner/coin.png}
	\cardprice{4}
	\cardtitle{Elster}
	\cardcontent{Du erhältst +1 Karte und +1 Aktion. Decke die oberste Karte deines Nachziehstapels auf. Ist dies ein Geldkarte (auch eine kombinierte oder Königreichkarte), nimm sie auf die Hand. Ist es eine Aktions- oder Punktekarte, nimmst du eine Elster vom Vorrat. Wenn keine Elster mehr im Vorrat ist, erhältst du nichts. Lege die aufgedeckte Karte zurück auf den Nachziehstapel. Ist es eine kombinierte Punkte- und Geldkarte, nimmst du die Karte auf die Hand und nimmst eine Elster vom Vorrat.}
\end{tikzpicture}
\hspace{-1cm}
\begin{tikzpicture}
	\card
	\cardstrip
	\cardbanner{banner/white.png}
	\cardicon{banner/coin.png}
	\cardprice{4}
	\cardtitle{Geizhals}
	\cardcontent{Du darfst entweder ein Kupfer aus deiner Hand auf dein Wirtshaustableau legen oder du erhältst pro Kupfer auf deinem Tableau +1 Geld. Kupfer auf deinem Wirtshaustableau be nden sich nicht im Spiel, zählen aber bei Spielende zu deinen Karten.}
\end{tikzpicture}
\hspace{-1cm}
\begin{tikzpicture}
	\card
	\cardstrip
	\cardbanner{banner/white.png}
	\cardicon{banner/coin.png}
	\cardprice{4}
	\cardtitle{Hafenstadt}
	\cardcontent{Wenn du diese Karte ausspielst, erhältst du +1 Karte und +2 Aktionen. Wenn du diese Karte kaufst (nicht wenn du sie auf andere Art und Weise erhältst), nimm dir eine Hafenstadt vom Vorrat.}
\end{tikzpicture}
\hspace{-1cm}
\begin{tikzpicture}
	\card
	\cardstrip
	\cardbanner{banner/white.png}
	\cardicon{banner/coin.png}
	\cardprice{4}
	\cardtitle{Kurier}
	\cardcontent{Wenn du diese Karte ausspielst, erhältst du +1 Kauf sowie +2 Geld. Außerdem darfst du deinen Nachziehstapel sofort komplett auf deinen Ablagestapel legen. Karten, die auf diese Weise abgelegt werden, gelten nicht als \enquote{abgelegt}.
	\\
	\medskip
	\\
	Wenn du diese Karte kaufst und dies der erste Kauf (inklusive eventuell erworbener Ereignisse) in deinem Zug ist, nimmst du eine Karte vom Vorrat, die bis zu 4 Geld kostet. Jeder Mitspieler nimmt sich – beginnend bei deinem linken Mitspieler – eine Karte mit gleichem Namen vom Vorrat. Karten, die nicht zum Vorrat gehören, dürfen nicht genommen werden.}
\end{tikzpicture}
\hspace{-1cm}
\begin{tikzpicture}
	\card
	\cardstrip
	\cardbanner{banner/lightbrown.png}
	\cardicon{banner/coin.png}
	\cardprice{4}
	\cardtitle{\footnotesize{Transformation}}
	\cardcontent{Diese Karte ist eine Reservekarte. Wenn du sie ausspielst, erhältst du +1 Aktion. Lege sie dann auf dein Wirtshaustableau.
	\\
	\medskip
	\\
	Zu Beginn deines Zuges darfst du diese Karte aufrufen. Wenn du das tust, entsorgst du eine Handkarte und nimmst eine Karte vom Vorrat, die bis zu 1 Geld mehr kostet als die entsorgte Karte. Nimm die Karte auf die Hand. Du darfst nur Karten aus dem Vorrat nehmen. Lege die Transformation in der Aufräumphase ab.}
\end{tikzpicture}
\hspace{-1cm}
\begin{tikzpicture}
	\card
	\cardstrip
	\cardbanner{banner/white.png}
	\cardicon{banner/coin.png}
	\cardprice{4}
	\cardtitle{Wildhüter}
	\cardcontent{Du erhältst +1 Kauf. Drehe deinen Reise-Marker um (zu Beginn des Spiels liegt der Marker mit der Vorderseite nach oben). Liegt jetzt die Vorderseite oben, ziehst du 5 Karten nach. Liegt die Rückseite oben, passiert nichts. Du darfst in einem Zug mehrere Wildhüter ausspielen und drehst bei jedem Wildhüter den Marker um.}
\end{tikzpicture}
\hspace{-1cm}
\begin{tikzpicture}
	\card
	\cardstrip
	\cardbanner{banner/orange.png}
	\cardicon{banner/coin.png}
	\cardprice{5}
	\cardtitle{Brückentroll}
	\cardcontent{Diese Karte ist eine Dauerkarte. Alle Mitspieler müssen ihren -1 Geld Marker vor sich ablegen. Sie erhalten beim nächsten Mal, wenn sie in irgendeiner Art und Weise mindestens 1 Geld erhalten würden, 1 Geld weniger. Danach wird der Marker wieder neben den Vorrat gelegt. Du erhältst jetzt und zu Beginn deines nächsten Zuges +1 Kauf.
	\\
	\medskip
	\\
	So lange diese Karte im Spiel ist, kosten alle Karten \emph{in deinem Zug} 1 Geld weniger, allerdings nie weniger als 0 Geld. Dies betrifft nicht nur den Kauf, sondern alle Aktionen, bei denen die Kosten einer Karte eine Rolle spielen. Der Brückentroll beeinflusst nicht die Kosten von Ereignissen. Die Wirkung des Brückentrolls ist kumulativ, d.h. wenn du zwei oder mehr Brückentrolle gleichzeitig im Spiel hast, kannst du die Kosten um 2 Geld oder mehr verringern.}
\end{tikzpicture}
\hspace{-1cm}
\begin{tikzpicture}
	\card
	\cardstrip
	\cardbanner{banner/lightbrowngreen.png}
	\cardicon{banner/coin.png}
	\cardprice{5}
	\cardtitle{Ferne Lande}
	\cardcontent{Diese Karte ist eine kombinierte Aktions-, Reserve- und Punktekarte. Wenn du sie ausspielst, lege sie auf dein Wirtshaustableau. Dort bleibt sie bis zum Spielende. Pro Ferne Lande, die du bei Spielende auf deinem Tableau liegen hast, erhältst du 4 Siegpunkte. Alle anderen Ferne Lande zählen 0 Siegpunkte.}
\end{tikzpicture}
\hspace{-1cm}
\begin{tikzpicture}
	\card
	\cardstrip
	\cardbanner{banner/orange.png}
	\cardicon{banner/coin.png}
	\cardprice{5}
	\cardtitle{Geisterwald}
	\cardcontent{Diese Karte ist eine Dauerkarte. Bis zu Beginn deines nächsten Zuges muss jeder Mitspieler, der eine Karte kauft, sofort alle restlichen Handkarten in beliebiger Reihenfolge auf den Nachziehstapel legen. Das Erwerben eines Ereignisses ist nicht mit dem Kauf einer Karte gleichzusetzen. Spieler, die mit einer Reaktionskarte reagieren wollen, müssen dies sofort beim Ausspielen des Geisterwalds tun.
	\\
	\medskip
	\\
	Ziehe zu Beginn deines nächsten Zuges 3 Karten nach. Lege die Karte in der Aufräumphase des nächsten Zuges ab.}
\end{tikzpicture}
\hspace{-1cm}
\begin{tikzpicture}
	\card
	\cardstrip
	\cardbanner{banner/white.png}
	\cardicon{banner/coin.png}
	\cardprice{5}
	\cardtitle{\tiny{Geschichtenerzähler}}
	\cardcontent{Du erhältst +1 Aktion sowie +1 Geld. Spiele bis zu 3 Geldkarten aus deiner Hand aus und zahle alle Geld, die du bisher in diesem Zug ausgespielt hast. Das beinhaltet alle Geldwerte von ausgespielten Geldkarten sowie alle, z. B. durch ausgespielte Aktionskarten, erhaltene zusätzliche Geldwerte (+1 Geld) inkl. dem +1 Geld dieses Geschichtenerzählers. Du hast danach 0 Geld. Für jedes gezahlte 1 Geld erhältst du +1 Karte. Trank-Kosten (aus Dominion – Die Alchemisten) sind davon nicht betroffen.}
\end{tikzpicture}
\hspace{-1cm}
\begin{tikzpicture}
	\card
	\cardstrip
	\cardbanner{banner/white.png}
	\cardicon{banner/coin.png}
	\cardprice{5}
	\cardtitle{\scriptsize{Kunsthandwerker}}
	\cardcontent{Du erhältst +1 Karte, +1 Aktion und +1 Geld. Lege beliebig viele Handkarten ab. Du darfst eine Karte vom Vorrat nehmen, die genau so viel kostet, wie du Karten abgelegt hast. Du kannst dich auch entscheiden, keine Karte abzulegen. Dann darfst du eine Karte nehmen, die genau 0 Geld kostet. Lege die genommene Karte verdeckt auf deinen Nachziehstapel.}
\end{tikzpicture}
\hspace{-1cm}
\begin{tikzpicture}
	\card
	\cardstrip
	\cardbanner{banner/lightbrown.png}
	\cardicon{banner/coin.png}
	\cardprice{5}
	\cardtitle{\scriptsize{Königliche Kutsche}}
	\cardcontent{Diese Karte ist eine Reservekarte. Wenn du sie ausspielst, erhältst du +1 Aktion und legst sie dann auf dein Wirtshaustableau.
	\\
	\medskip
	\\
	Sobald du eine Aktionskarte ausspielst und die Anweisungen darauf ausgeführt hast, darfst du die Königliche Kutsche sofort danach aufrufen, falls die Aktionskarte noch im Spiel ist. Wenn du das tust, führst du die Aktion sofort noch einmal aus. Du kannst die Königliche Kutsche nicht aufrufen, wenn die Aktion bereits entsorgt wurde oder auf Reservekarten, die bereits auf das Tableau gelegt wurden. Lege die Königliche Kutsche in der Aufräumphase ab. Wird die Königliche Kutsche auf eine Dauerkarte gespielt, bleibt sie solange im Spiel, bis die Dauerkarte abgelegt wird.}
\end{tikzpicture}
\hspace{-1cm}
\begin{tikzpicture}
	\card
	\cardstrip
	\cardbanner{banner/gold.png}
	\cardicon{banner/coin.png}
	\cardprice{5}
	\cardtitle{Relikt}
	\cardcontent{Diese Karte ist eine Geldkarte mit zusätzlichen Anweisungen. Sie hat den Wert 2 Geld. Jeder Mitspieler muss seinen -1 Karte Marker auf seinen Nachziehstapel legen. Sobald ein Mitspieler Karten nachziehen muss, zieht er 1 Karte weniger. Dann legt er den Marker neben den Vorrat zurück. Bei Anweisungen, durch die man seine Kartenhand auf X Karten auffüllen darf, wird zunächst der Marker zurückgelegt und dann die Kartenhand auf X Karten aufgefüllt.
	\\
	\medskip
	\\
	Das Relikt ist eine Angriffskarte. Entsprechend kann darauf mit Reaktionskarten reagiert werden. Reagiert ein Spieler mit dem Karawanenwächter, führt er zunächst die Anweisungen aus (inkl. +1 Karte) und legt dann den Marker auf seinen Nachziehstapel.}
\end{tikzpicture}
\hspace{-1cm}
\begin{tikzpicture}
	\card
	\cardstrip
	\cardbanner{banner/white.png}
	\cardicon{banner/coin.png}
	\cardprice{5}
	\cardtitle{Riese}
	\cardcontent{Drehe deinen Reise-Marker um (zu Beginn des Spiels liegt der Marker mit der Vorderseite nach oben). Liegt jetzt die Rückseite oben, erhältst du +1 Geld. Liegt die Vorderseite oben, erhältst du +5 Geld und alle Mitspieler – beginnend bei deinem linken Mitspieler – müssen die oberste Karte ihres Nachziehstapels aufdecken. Aufgedeckte Karten, die 3 bis 6 Geld kosten, müssen entsorgt werden. Karten mit Trank-Kosten müssen nicht entsorgt werden, Karten mit + Geld (z. B. aus Dominion – Die Gilden) oder * Geld müssen entsorgt werden.
	\\
	\medskip
	\\
	Ansonsten legt der Spieler die aufgedeckte Karte ab und nimmt sich einen Fluch vom Vorrat. Die Mitspieler können auf das Ausspielen eines Riesen mit einer Reaktionskarte reagieren, auch wenn die Rückseite des Reise-Markers oben liegt und der Angriff gar nicht ausgeführt wird.}
\end{tikzpicture}
\hspace{-1cm}
\begin{tikzpicture}
	\card
	\cardstrip
	\cardbanner{banner/gold.png}
	\cardicon{banner/coin.png}
	\cardprice{5}
	\cardtitle{Schatz}
	\cardcontent{Diese Karte ist eine Geldkarte mit zusätzlichen Anweisungen. Sie hat den Wert 2 Geld. Nimm ein Gold und ein Kupfer (in beliebiger Reihenfolge) vom Vorrat und lege sie ab. Ist ein Stapel leer, erhältst du nur die andere Karte.}
\end{tikzpicture}
\hspace{-1cm}
\begin{tikzpicture}
	\card
	\cardstrip
	\cardbanner{banner/orange.png}
	\cardicon{banner/coin.png}
	\cardprice{5}
	\cardtitle{Sumpfhexe}
	\cardcontent{Diese Karte ist eine Dauerkarte. Bis zu Beginn deines nächsten Zuges muss jeder Mitspieler für jede Karte, die er kauft, jeweils einen Fluch vom Vorrat nehmen. Das Erwerben eines Ereignisses ist nicht mit dem Kauf einer Karte gleichzusetzen. Spieler, die mit einer Reaktionskarte reagieren wollen, müssen dies sofort beim Ausspielen der Sumpfhexe tun.
	\\
	\medskip
	\\
	Zu Beginn deines nächsten Zuges erhältst du +3 Geld. Lege die Karte in der Aufräumphase des nächsten Zuges ab.}
\end{tikzpicture}
\hspace{-1cm}
\begin{tikzpicture}
	\card
	\cardstrip
	\cardbanner{banner/white.png}
	\cardicon{banner/coin.png}
	\cardprice{5}
	\cardtitle{\scriptsize{Verlorene Stadt}}
	\cardcontent{Wenn du diese Karte ausspielst, erhältst du +2 Karten und +2 Aktionen. Wenn du diese Karte nimmst, zieht jeder Mitspieler 1 Karte.}
\end{tikzpicture}
\hspace{-1cm}
\begin{tikzpicture}
	\card
	\cardstrip
	\cardbanner{banner/lightbrown.png}
	\cardicon{banner/coin.png}
	\cardprice{5}
	\cardtitle{Weinhändler}
	\cardcontent{Diese Karte ist eine Reservekarte. Wenn du sie ausspielst, erhältst du +1 Kauf sowie +4 Geld und legst die Karte dann auf dein Wirtshaustableau.
	\\
	\medskip
	\\
	Wenn du am Ende deiner Kaufphase mindestens 2 Geld nicht ausgegeben hast, darfst du diese Karte von deinem Wirtshaustableau ablegen. Hast du mehrere Weinhändler auf deinem Tableau liegen, kannst du alle ablegen. Du brauchst nicht für jeden Weinhändler 2 Geld übrig zu haben.}
\end{tikzpicture}
\hspace{-1cm}
\begin{tikzpicture}
	\card
	\cardstrip
	\cardbanner{banner/orange.png}
	\cardicon{banner/coin.png}
	\cardprice{6}
	\cardtitle{Gefolgsmann}
	\cardcontent{Diese Karte ist eine Dauerkarte. Wenn du sie ausspielst, lege sie neben deinen Spielbereich. Sie bleibt bis zum Spielende im Spiel und wird nicht abgelegt.
	\\
	\medskip
	\\
	Für den Rest des Spiels erhältst du zu Beginn deines Zuges +1 Karte. Wenn du einen Gefolgsmann zweimal ausspielen darfst (z. B. durch den Thronsaal aus der Basis), lege beide Karten zur Seite und du erhältst bis zum Spielende zu Beginn jedes Zuges +2 Karten.}
\end{tikzpicture}
\hspace{-1cm}
\begin{tikzpicture}
	\card
	\cardstrip
	\cardbanner{banner/white.png}
	\cardtitle{Ereignisse (1/8)\quad}
	\cardcontent{Ereignisse können nur in der Kaufphase erworben werden. Dies benötigt 1 Kauf sowie genügend (vorher ausgespielte) Geldwerte. Der benötigte Geldwert ist auf jedem Ereignis oben links zu finden. Sobald du ein Ereignis erwirbst, führst du die darauf beschriebene Anweisung aus. Du nimmst das Ereignis aber nicht an dich.
	\\
	\medskip
	\\
	\emph{Almosen (0):} Dieses Ereignis darf nur 1x pro Zug erworben werden. Falls du keine Geldkarten zu diesem Zeitpunkt im Spiel hast, nimm dir eine Karte vom Vorrat, die bis zu 4 Geld kostet. Eintausch-Karten oder Preiskarten (aus Dominion – Reiche Ernte) mit * Geld gehören nicht zum Vorrat und dürfen nicht genommen werden.
	\\
	\medskip
	\\
	\emph{Leihgabe (0):} Dieses Ereignis darf nur 1x pro Zug erworben werden. Du erhältst +1 Kauf. Wenn dein -1 Karte Marker nicht auf deinem Nachziehstapel liegt, lege ihn dorthin und erhalte +1 Geld. Das nächste Mal, wenn du Karten nachziehen musst, ziehst du 1 Karte weniger.}
\end{tikzpicture}
\hspace{-1cm}
\begin{tikzpicture}
	\card
	\cardstrip
	\cardbanner{banner/white.png}
	\cardtitle{Ereignisse (2/8)\quad}}
	\cardcontent{\emph{Quest (0):} Wähle eine der Optionen, um ein Gold zu nehmen und abzulegen: Entweder legst du 1 Angri skarte aus deiner Hand oder 2 Flüche oder 6 beliebige Karten ab. Du kannst dich entscheiden, nur 1 Fluch oder weniger als 6 Karten abzulegen, wenn du nicht genügend entsprechende Karten auf der Hand hast – dann erhältst du kein Gold.
	\\
	\medskip
	\\
	\emph{Zuflucht (1):} Dieses Ereignis darf nur 1x pro Zug erworben werden. Du erhältst +1 Kauf und legst eine beliebige Handkarte verdeckt zur Seite. Nachdem du in der Aufräumphase Karten nachgezogen hast, nimmst du die zur Seite gelegte Karte wieder auf die Hand.
	\\
	\medskip
	\\
	\emph{Spähtrupp (2):} Du erhältst +1 Kauf. Schau dir die obersten 5 Karten deines Nachziehstapels an. Lege 3 davon ab und den Rest in beliebiger Reihenfolge zurück auf den Nachziehstapel. Hast du nach dem Mischen des Ablagestapels weniger als 5 Karten zur Verfügung, legst du zuerst – wenn möglich – 3 Karten ab. Nur den Rest (0, 1 oder 2 Karten) legst du zurück auf den Nachziehstapel.
	\\
	\medskip
	\\
	Der \emph{Spähtrupp} ist nicht betroffen vom -1 Karte Marker, der eventuell auf deinem Nachziehstapel liegt, da du die Karten nicht nimmst sondern nur ansiehst. In diesem Fall legst du den Marker nach dem Ausführen des SPÄHTRUPPS auf den Nachziehstapel zurück.}
\end{tikzpicture}
\hspace{-1cm}
\begin{tikzpicture}
	\card
	\cardstrip
	\cardbanner{banner/white.png}
	\cardtitle{Ereignisse (3/8)\quad}}
	\cardcontent{\emph{Wanderzirkus (2):} Du erhältst +2 Käufe. Wenn du in dem Zug, in dem du den Wanderzirkus erwirbst, eine Karte nimmst, darfst du diese oben auf deinen Nachziehstapel legen. Wenn du dies nicht möchtest, lege die genommene Karte ab. Der Wanderzirkus funktioniert nicht bei den Eintausch-Karten mit einem * Geld, da diese nicht genommen, sondern eingetauscht werden.
	\\
	\medskip
	\\
	\emph{Expedition (3):} In der Aufräumphase des Zuges, in dem du diese Karte erwirbst, ziehst du 2 Karten zusätzlich. Normalerweise ziehst du 5 Karten nach, mit einer Expedition 7 Karten, mit zwei Expeditionen 9 Karten usw.
	\\
	\medskip
	\\
	\emph{Freudenfeuer (3):} Entsorge bis zu 2 Karten, die du gerade im Spiel hast. Du darfst keine Handkarten entsorgen. Entsorgst du eine oder zwei Geldkarten, kannst du trotzdem den durch diese Karten produzierten Geldwert nutzen.
	\\
	\medskip
	\\
	\emph{Planung (3):} Lege deinen Entsorgungs-Marker auf einen beliebigen Aktions-Vorratsstapel. Solange dieser Marker auf dem Stapel liegt, darfst du, immer wenn du eine Karte von diesem Stapel kaufst, eine beliebige Handkarte entsorgen.
	\\
	\medskip
	\\
	Wenn du eine weitere Planung erwirbst, legst du den Marker auf einen anderen Aktions-Vorratsstapel.}
\end{tikzpicture}
\hspace{-1cm}
\begin{tikzpicture}
	\card
	\cardstrip
	\cardbanner{banner/white.png}
	\cardtitle{Ereignisse (4/8)\quad}}
	\cardcontent{\emph{Überfahrt (3):} Lege deinen -2 Geld Kosten-Marker auf einen beliebigen Aktions-Vorratsstapel. Solange dieser Marker auf dem Stapel liegt, kosten Karten, die von diesem Stapel stammen, für dich für alle Belange 2 Geld weniger, niemals aber weniger als 0 Geld.
	\\
	\medskip
	\\
	Wenn du eine weitere Überfahrt erwirbst, legst du den Marker auf einen anderen Aktions-Vorratsstapel.
	\\
	\medskip
	\\
	\emph{Mission (4):} Dieses Ereignis darf nur 1x pro Zug erworben werden. Du spielst nach diesem Zug einen weiteren Zug, falls der vorangegangene Zug von einem Mitspieler gespielt wurde und nicht von dir (z. B. durch den Außenposten aus Dominion – Seaside). In diesem Zug darfst du keine Karten kaufen. Du darfst allerdings Ereignisse erwerben, Karten auf andere Weise nehmen oder Reisende eintauschen. Erwirbst du in deinem Extrazug eine weitere Mission, darfst du keinen weiteren Zug ausführen, da der vorherige Zug dein eigener war.}
\end{tikzpicture}
\hspace{-1cm}
\begin{tikzpicture}
	\card
	\cardstrip
	\cardbanner{banner/white.png}
	\cardtitle{Ereignisse (5/8)\quad}}
	\cardcontent{\emph{Pilgerfahrt (4):} Dieses Ereignis darf nur 1x pro Zug erworben werden. Drehe deinen Reise-Marker um (zu Spielbeginn liegt dieser mit der Vorderseite nach oben). Liegt dann die Rückseite oben, passiert nichts. Liegt die Vorderseite oben, wählst du bis zu 3 Karten mit unterschiedlichem Namen, die du gerade im Spiel hast, nimmst von jeder dieser Karten eine weitere Karte vom Vorrat und legst sie ab. Eintausch-Karten oder Preiskarten (aus Dominion – Reiche Ernte) mit * Geld gehören nicht zum Vorrat und dürfen nicht genommen werden.
	\\
	\medskip
	\\
	\emph{Ball (5):} Nimm deinen -1 Geld Marker und lege ihn vor dir ab. Das nächste Mal, wenn du mindestens 1 Geld erhältst, erhältst du stattdessen 1 Geld weniger. Danach legst du den Marker wieder neben den Vorrat.
	\\
	\medskip
	\\
	Nimm 2 Karten vom Vorrat, die jede bis zu 4 Geld kosten. Das können gleiche oder verschiedene Karten sein. Lege beide Karten ab. Eintausch-Karten oder Preiskarten (aus Dominion – Reiche Ernte) gehören nicht zum Vorrat und dürfen nicht genommen werden.
	\\
	\medskip
	\\
	\emph{Handel (5):} Entsorge bis zu 2 Handkarten. Du darfst auch 1 oder gar keine Karte entsorgen. Für jede so entsorgte Karte nimmst du ein Silber vom Vorrat und legst es ab.}
\end{tikzpicture}
\hspace{-1cm}
\begin{tikzpicture}
	\card
	\cardstrip
	\cardbanner{banner/white.png}
	\cardtitle{Ereignisse (6/8)\quad}}
	\cardcontent{\emph{Seeweg (5):} Nimm eine Aktionskarte vom Vorrat, die bis zu 4 Geld kostet, und lege diese ab. Lege deinen +1 Kauf-Marker auf den Vorratsstapel, von dem du die Aktionskarte genommen hast (dieser kann dann auch leer sein). Du musst eine Karte vom Vorrat nehmen, um den Marker auf den entsprechenden Stapel legen zu dürfen. Solange dieser Marker auf dem Stapel liegt, erhältst du +1 Kauf, sobald du eine Karte, die von diesem Stapel stammt, ausspielst. Kostet die Aktionskarte, die du durch den Seeweg nimmst, nur in diesem Zug bis zu 4 Geld, z. B. wenn du vorher einen Brückentroll gespielt hast und dadurch alle Karte in diesem Zug 1 Geld weniger kosten, bleibt der Marker später liegen, auch wenn die entsprechende Karte nach deinem Zug wieder 5 Geld kostet. Auf Stapel, die nicht zum Vorrat gehören, darf der Marker nicht gelegt werden.
	\\
	\medskip
	\\
	Durch den Erwerb eines weiteren Seeweges oder das Aufrufen eines Lehrers darfst du den Marker auf einen anderen Aktions-Vorratsstapel legen.}
\end{tikzpicture}
\hspace{-1cm}
\begin{tikzpicture}
	\card
	\cardstrip
	\cardbanner{banner/white.png}
	\cardtitle{Ereignisse (7/8)\quad}}
	\cardcontent{\emph{Überfall (5):} Für jedes Silber, das du gerade im Spiel hast, wenn du diese Karte erwirbst, nimm ein weiteres Silber vom Vorrat und lege es ab. Außerdem legen alle Mitspieler ihren -1 Karte Marker auf ihren Nachziehstapel. Das nächste Mal, wenn sie Karten nachziehen müssen, ziehen sie stattdessen 1 Karte weniger. Auch wenn diese Anweisung wie eine Angriff fungiert, ist der Überfall keine Angriffskarte. Entsprechend können die Mitspieler nicht mit einer Reaktionskarte darauf reagieren.
	\\
	\medskip
	\\
	\emph{Training (6):} Lege deinen +1 Geld Marker auf einen beliebigen Aktions-Vorratsstapel. Solange dieser Marker auf dem Stapel liegt, erhältst du +1 Geld, sobald du eine Karte, die von diesem Stapel stammt, ausspielst.
	\\
	\medskip
	\\
	Durch den Erwerb eines weiteren Trainings oder das Aufrufen eines Lehrers darfst du den Marker auf einen anderen Aktions-Vorratsstapel legen.
	\\
	\medskip
	\\
	\emph{Verlorene Kunst (6):} Lege deinen +1 Aktion-Marker auf einen beliebigen Aktions-Vorratsstapel. Solange dieser Marker auf dem Stapel liegt, erhältst du +1 Aktion, sobald du eine Karte, die von diesem Stapel stammt, ausspielst.
	\\
	\medskip
	\\
	Durch den Erwerb einer weiteren Verlorenen Kunst oder das Aufrufen eines Lehrers darfst du den Marker auf einen anderen Aktions-Vorratsstapel legen.}
\end{tikzpicture}
\hspace{-1cm}
\begin{tikzpicture}
	\card
	\cardstrip
	\cardbanner{banner/white.png}
	\cardtitle{Ereignisse (8/8)\quad}}
	\cardcontent{\emph{Erbschaft (7):} Dieses Ereignis darf nur 1x pro Spiel und Spieler erworben werden. Lege eine beliebige Nicht-Punktekarte vom Vorrat, die eine Aktionskarte ist und bis zu 4 Geld kostet, zur Seite. Lege deinen Anwesen-Marker darauf. Ab jetzt fungieren alle deine Anwesen wie die markierte Karte. Du kannst Dauer- oder Reservekarten, Geldkarten, Angriffskarten usw. markieren. Wenn du ein Anwesen auf der Hand hast, darfst du es in der Phase, in der die markierte Karte ausgespielt werden würde, auch ausspielen. Führe alle Anweisungen der markierten Karte aus.
	\\
	\medskip
	\\
	Eigenschaften, Anweisungen und Typ (z. B. Aktion, Angriff etc.) der markierten Karte werden durch die Erbschaft kopiert, nicht aber Name oder Wert bzw. von welchem Stapel die markierte Karte stammt. So kann ein Spieler, der mit Hilfe der Erbschaft einen Wildhüter markiert und gleichzeitig auf diesem Stapel seinen +1 Aktion-Marker liegen hat, durch das Ausspielen eines Anwesens zwar die Anweisungen des Wildhüters nutzen, er erhält aber nicht zusätzlich +1 Aktion.
	\\
	\medskip
	\\
	Bei Spielende zählen deine Anwesen weiterhin jeweils 1 Siegpunkt.
	\\
	\medskip
	\\
	\emph{Wegsuche (8):} Lege deinen +1 Karte-Marker auf einen beliebigen Aktions-Vorratsstapel. Solange dieser Marker auf dem Stapel liegt, erhältst du +1 Karte, sobald du eine Karte, die von diesem Stapel stammt, ausspielst.}
\end{tikzpicture}
\hspace{-1cm}
\begin{tikzpicture}
	\card
	\cardstrip
	\cardbanner{banner/white.png}
	\cardicon{banner/coin.png}
	\cardprice{2}
	\cardtitle{Kleinbauer}
	\cardcontent{Der Kleinbauer ist eine Königreichkarte und kann wie jede andere Königreichkarte gekauft oder genommen werden. Der Kleinbauer ist außerdem ein \emph{Reisender}, der im Spielverlauf in einen Soldaten eingetauscht werden kann.
	\\
	\medskip
	\\
	Wenn du den Kleinbauern ausspielst, erhältst du +1 Kauf sowie +1 Geld. In der Aufräumphase darfst du entscheiden, ob du den Kleinbauern ablegst oder zurück auf den Vorratsstapel legst. Wenn du das tust, erhältst du einen Soldaten und legst ihn ab. Die angegebenen Kosten werden \emph{nicht} bezahlt.}
\end{tikzpicture}
\hspace{-1cm}
\begin{tikzpicture}
	\card
	\cardstrip
	\cardbanner{banner/white.png}
	\cardicon{banner/coin.png}
	\cardprice{3\textsuperscript{*}}
	\cardtitle{Soldat}
	\cardcontent{Einen Soldaten erhältst du nur, wenn du einen Kleinbauern eintauschst. Der Soldat ist ein \emph{Reisender}, der im Spielverlauf in einen Flüchtling eingetauscht werden kann.
	\\
	\medskip
	\\
	Wenn du den Soldaten ausspielst, erhältst du +2 Geld sowie für jede andere Angriffskarte (außer diesem Soldaten), die du zu diesem Zeitpunkt im Spiel hast, +1 Geld. Außerdem muss jeder Mitspieler, der 4 oder mehr Karten auf der Hand hat, 1 Karte ablegen.
	\\
	\medskip
	\\
	In der Aufräumphase darfst du entscheiden, ob du den Soldaten ablegst oder zurück auf den entsprechenden Stapel legst. Wenn du das tust, erhältst du einen Flüchtling und legst ihn ab. Die angegebenen Kosten werden \emph{nicht} bezahlt.}
\end{tikzpicture}
\hspace{-1cm}
\begin{tikzpicture}
	\card
	\cardstrip
	\cardbanner{banner/white.png}
	\cardicon{banner/coin.png}
	\cardprice{4\textsuperscript{*}}
	\cardtitle{Flüchtling}
	\cardcontent{Einen Flüchtling erhältst du nur, wenn du einen Soldaten eintauschst. Der Flüchtling ist ein \emph{Reisender}, der im Spielverlauf in einen Schüler eingetauscht werden kann.
	\\
	\medskip
	\\
	Wenn du den Flüchtling ausspielst, erhältst du + 2 Karten sowie + 1 Aktion und du musst eine Handkarte ablegen.
	\\
	\medskip
	\\
	In der Aufräumphase darfst du entscheiden, ob du den Soldaten ablegst oder zurück auf den entsprechenden Stapel legst. Wenn du das tust, erhältst du einen Schüler und legst ihn ab. Die angegebenen Kosten werden \emph{nicht} bezahlt.}
\end{tikzpicture}
\hspace{-1cm}
\begin{tikzpicture}
	\card
	\cardstrip
	\cardbanner{banner/white.png}
	\cardicon{banner/coin.png}
	\cardprice{5\textsuperscript{*}}
	\cardtitle{Schüler}
	\cardcontent{Einen Schüler erhältst du nur, wenn du einen Flüchtling eintauschst. Der Schüler ist ein \emph{Reisender}, der im Spielverlauf in einen Lehrer eingetauscht werden kann.
	\\
	\medskip
	\\
	Wenn du den Schüler ausspielst, darfst du eine beliebige Aktionskarte aus deiner Hand zweimal direkt hintereinander ausspielen. Ist die ausgespielte Aktionskarte eine Karte vom Vorrat, nimm dir eine Karte mit gleichem Namen vom Vorrat und lege sie ab. Gehört der Stapel der ausgespielten Karte \emph{nicht} zum Vorrat (z. B. Eintausch-Karten oder Preiskarten aus Dominion – Reiche Ernte), darfst du dir keine weitere Karte nehmen.
	\\
	\medskip
	\\
	In der Aufräumphase darfst du entscheiden, ob du den Schüler ablegst oder zurück auf den entsprechenden Stapel legst. Wenn du das tust, erhältst du einen Lehrer und legst ihn ab. Die angegebenen Kosten werden \emph{nicht} bezahlt.}
\end{tikzpicture}
\hspace{-1cm}
\begin{tikzpicture}
	\card
	\cardstrip
	\cardbanner{banner/lightbrown.png}
	\cardicon{banner/coin.png}
	\cardprice{6\textsuperscript{*}}
	\cardtitle{Lehrer}
	\cardcontent{Einen Lehrer erhältst du nur, wenn du einen Schüler eintauschst. Er kann in keine andere Karte eingetauscht werden.
	\\
	\medskip
	\\
	Wenn du den Lehrer ausspielst, lege ihn auf dein Wirtshaustableau.
	\\
	\medskip
	\\
	Zu Beginn deines Zuges darfst du den Lehrer von deinem Tableau aufrufen und in den Spielbereich legen. Wenn du das tust, legst du entweder den +1 Karte-, +1 Aktion-, +1 Kauf- oder +1 Geld-Marker auf einen beliebigen Aktions-Vorratsstapel, auf dem du zu diesem Zeitpunkt keine weiteren eigenen Marker liegen hast. Immer wenn du eine Karte, die von diesem Stapel stammt, spielst, erhältst du zuerst (bevor die Anweisungen der Karte ausgeführt werden) den entsprechenden Bonus des Markers. Lege den Lehrer in der Aufräumphase ab.}
\end{tikzpicture}
\hspace{-1cm}
\begin{tikzpicture}
	\card
	\cardstrip
	\cardbanner{banner/white.png}
	\cardicon{banner/coin.png}
	\cardprice{2}
	\cardtitle{Page}
	\cardcontent{Der Page ist eine Königreichkarte und kann wie jede andere Königreichkarte gekauft oder genommen werden. Der Page ist außerdem ein \emph{Reisender}, der im Spielverlauf in einen Schatzsucher eingetauscht werden kann.
	\\
	\medskip
	\\
	Wenn du den Pagen ausspielst, erhältst du +1 Karte sowie +1 Aktion.
	\\
	\medskip
	\\
	In der Aufräumphase darfst du entscheiden, ob du den Pagen ablegst oder zurück auf den Vorratsstapel legst. Wenn du das tust, erhältst du einen Schatzsucher und legst ihn ab. Die angegebenen Kosten werden \emph{nicht} bezahlt.}
\end{tikzpicture}
\hspace{-1cm}
\begin{tikzpicture}
	\card
	\cardstrip
	\cardbanner{banner/white.png}
	\cardicon{banner/coin.png}
	\cardprice{3\textsuperscript{*}}
	\cardtitle{Schatzsucher}
	\cardcontent{Einen Schatzsucher erhältst du nur, wenn du einen Pagen eintauschst. Der Schatzsucher ist ein \emph{Reisender}, der im Spielverlauf in einen Krieger eingetauscht werden kann.
	\\
	\medskip
	\\
	Wenn du den Schatzsucher ausspielst, erhältst du +1 Aktion sowie +1 Geld. Für jede Karte, die dein rechter Mitspieler in seinem letzten Zug gekauft oder genommen hat, nimmst du ein Silber vom Vorrat und legst es ab.
	\\
	\medskip
	\\
	In der Aufräumphase darfst du entscheiden, ob du den Schatzsucher ablegst oder zurück auf den entsprechenden Stapel legst. Wenn du das tust, erhältst du einen Krieger und legst ihn ab. Die angegebenen Kosten werden \emph{nicht} bezahlt.}
\end{tikzpicture}
\hspace{-1cm}
\begin{tikzpicture}
	\card
	\cardstrip
	\cardbanner{banner/white.png}
	\cardicon{banner/coin.png}
	\cardprice{4\textsuperscript{*}}
	\cardtitle{Krieger}
	\cardcontent{Einen Krieger erhältst du nur, wenn du einen Schatzsucher eintauschst. Der Krieger ist ein \emph{Reisender}, der im Spielverlauf in einen Helden eingetauscht werden kann.
	\\
	\medskip
	\\
	Wenn du den Krieger ausspielst, erhältst du +2 Karten. Für jeden Reisenden, den du (inkl. dieses Kriegers) zu diesem Zeitpunkt im Spiel hast, müssen alle Mitspieler – beginnend bei deinem linken Nachbarn – die oberste Karte ihres Nachziehstapels aufdecken und ablegen. Abgelegte Karten, die genau 3 Geld oder 4 Geld kosten, müssen entsorgt werden. Karten mit Trank-Kosten müssen nicht entsorgt werden. Karten mit * Geld oder + Geld müssen entsorgt werden, wenn sie 3 Geld oder 4 Geld kosten. Alle aufgedeckten Karten, die nicht genau 3 Geld oder 4 Geld kosten, müssen abgelegt werden.
	\\
	\medskip
	\\
	In der Aufräumphase darfst du entscheiden, ob du den Krieger ablegst oder zurück auf den entsprechenden Stapel legst. Wenn du das tust, erhältst du einen Helden und legst ihn ab. Die angegebenen Kosten werden \emph{nicht} bezahlt.}
\end{tikzpicture}
\hspace{-1cm}
\begin{tikzpicture}
	\card
	\cardstrip
	\cardbanner{banner/white.png}
	\cardicon{banner/coin.png}
	\cardprice{5\textsuperscript{*}}
	\cardtitle{Held}
	\cardcontent{Einen Helden erhältst du nur, wenn du einen Krieger eintauschst. Der Held ist ein \emph{Reisender}, der im Spielverlauf in einen Champion eingetauscht werden kann.
	\\
	\medskip
	\\
	Wenn du den Helden ausspielst, erhältst du +2 Geld und darfst dir eine beliebige Geldkarte nehmen, die in diesem Spiel verwendet wird. Dazu gehören auch Geldkarten, die zu den Königreichkarten gehören. Lege die Geldkarte ab.
	\\
	\medskip
	\\
	In der Aufräumphase darfst du entscheiden, ob du den Helden ablegst oder zurück auf den entsprechenden Stapel legst. Wenn du das tust, erhältst du einen Champion und legst ihn ab. Die angegebenen Kosten werden \emph{nicht} bezahlt.}
\end{tikzpicture}
\hspace{-1cm}
\begin{tikzpicture}
	\card
	\cardstrip
	\cardbanner{banner/orange.png}
	\cardicon{banner/coin.png}
	\cardprice{6\textsuperscript{*}}
	\cardtitle{Champion}
	\cardcontent{Einen Champion erhältst du nur, wenn du einen Helden eintauschst. Er kann in keine andere Karte eingetauscht werden.
	\\
	\medskip
	\\
	Wenn du den Champion ausspielst, erhältst du +1 Aktion. Der Champion ist eine Dauerkarte. Er bleibt bis zum Spielende im Spiel.
	\\
	\medskip
	\\
	Immer wenn ein Mitspieler ab jetzt eine Angriffskarte ausspielt, bist du davon nicht betroffen (auch wenn du das möchtest). Für den Rest des Spiels erhältst du jedes Mal, wenn du eine Aktionskarte ausspielst, +1 Aktion. Die Anweisung über der Trennlinie wird nur beim Ausspielen des Champion ausgeführt.}
\end{tikzpicture}
\hspace{-1cm}
\begin{tikzpicture}
	\card
	\cardstrip
	\cardbanner{banner/white.png}
	\cardicon{}
	\cardprice{}
	\cardtitle{\scriptsize{Empfohlene 10er Sätze\qquad}}
	\cardcontent{\tiny{\emph{Sanfte Einführung}
	\\
	\smallskip
	\\
	Spähtrupp (Ereignis), Amulett, Duplikat, Ferne Lande, Gefolgsmann, Hafenstadt, Rattenfänger, Riese, Schatz, Verlies, Wildhüter
	\\
	\smallskip
	\\
	\emph{Profi-Einführung}
	\\
	\smallskip
	\\
	Mission (Ereignis), Planung (Ereignis), Elster, Geisterwald, Karawanenwächter, Kleinbauer (+ Eintauschkarten), Königliche Münzen, Sumpfhexe, Verlorene Stadt, Weinhändler, Zerstörung, Transformation
	\\
	\smallskip
	\\
	\emph{Auf höchster Ebene (mit Basisspiel)}
	\\
	\smallskip
	\\
	Training (Ereignis), Ausrüstung, Geizhals, Kundschafter, Verlies, Verlorene Stadt, Markt, Miliz, Spion, Thronsaal, Werkstatt
	\\
	\smallskip
	\\
	\emph{Verzerrte Größen (mit Basisspiel)}
	\\
	\smallskip
	\\
	Freudenfeuer (Ereignis), Überfall (Ereignis), Amulett, Duplikat, Kurier, Riese, Schatz, Bürokrat, Dieb, Gärten, Geldverleiher, Hexe
	\\
	\smallskip
	\\
	\emph{Lizenz Fabrik (mit Die Intrige)}
	\\
	\smallskip
	\\
	Pilgerfahrt (Ereignis), Brückentroll, Duplikat, Königliche Kutsche, Page (+ Eintauschkarten), Zerstörung, Adelige, Geheimkammer, Harem, Trickser, Verschwörer
	\\
	\smallskip
	\\
	\emph{Meister der Finanzen (mit Die Intrige)}
	\\
	\smallskip
	\\
	Ball (Ereignis), Leihgabe (Ereignis), Ausrüstung, Ferne Lande, Kunsthandwerker, Transformation, Weinhändler, Anbau, Armenviertel, Brücke, Handlanger, Verwalter
	\\}}
\end{tikzpicture}
\hspace{-1cm}
\begin{tikzpicture}
	\card
	\cardstrip
	\cardbanner{banner/white.png}
	\cardicon{}
	\cardprice{}
	\cardtitle{\scriptsize{Empfohlene 10er Sätze\qquad}}
	\cardcontent{\tiny{\emph{Prinz von Oranien (mit Seaside)}
	\\
	\smallskip
	\\
	Mission (Ereignis), Amulett, Geisterwald, Page (+ Eintauschkarten), Sumpfhexe, Verlies, Fischerdorf, Handelsschiff, Karawane, Schatzkarte, Taktiker
	\\
	\smallskip
	\\
	\emph{Geschenke und triviale Präsente (mit Seaside)}
	\\
	\smallskip
	\\
	Expedition (Ereignis), Quest (Ereignis), Brückentroll, Gefolgsmann, Karawanenwächter, Kurier, Verlorene Stadt, Botschafter, Embargo, Hafen, Müllverwerter, Schmuggler
	\\
	\smallskip
	\\
	\emph{Letzter Wille und das Denkmal (mit Blütezeit)}
	\\
	\smallskip
	\\
	Erbschaft (Ereignis), Königliche Münzen, Kurier, Relikt, Schatz, Verlies, Bischof, Denkmal, Gesindel, Gewölbe, Leihhaus, Platin- und Kolonie-Karten
	\\
	\smallskip
	\\
	\emph{Gedanken im großen Stil (mit Blütezeit)}
	\\
	\smallskip
	\\
	Ball (Ereignis), Überfahrt (Ereignis), Ferne Lande, Geizhals, Gefolgsmann, Geschichtenerzähler, Riese, Ausbau, Hausierer, Hort, Königshof, Schmuggelware, Platin- und Kolonie-Karten
	\\
	\smallskip
	\\
	\emph{Rückkehr des Helden (mit Reiche Ernte)}
	\\
	\smallskip
	\\
	Wanderzirkus (Ereignis), Geizhals, Kunsthandwerker, Page (+ Eintauschkarten), Relikt, Wildhüter, Bauerndorf, Festplatz, Harlekin, Menagerie, Pferdehändler
	\\
	\smallskip
	\\
	\emph{Seefahrerei und Hexerei (mit Reiche Ernte)}
	\\
	\smallskip
	\\
	Seeweg (Ereignis), Überfahrt (Ereignis). Geschichtenerzähler, Kleinbauer (+ Eintauschkarten), Kundschafter (als Bannkarte für die Junge Hexe), Sumpfhexe, Transformation, Weinhändler, Füllhorn, Junge Hexe, Turnier, Wahrsagerin, Weiler
	\\}}
\end{tikzpicture}
\hspace{-1cm}
\begin{tikzpicture}
	\card
	\cardstrip
	\cardbanner{banner/white.png}
	\cardicon{}
	\cardprice{}
	\cardtitle{\scriptsize{Empfohlene 10er Sätze\qquad}}
	\cardcontent{\tiny{\emph{Händler und Räuber (mit Hinterland)}
	\\
	\smallskip
	\\
	Überfall (Ereignis). Geisterwald, Hafenstadt, Page (+ Eintauschkarten), Verlorene Stadt, Weinhändler, Aufbau, Fahrender Händler, Feilscher, Fruchtbares Land, Gewürzhändler
	\\
	\smallskip
	\\
	\emph{Reisen (mit Hinterland)}
	\\
	\smallskip
	\\
	Erbschaft (Ereignis), Expedition (Ereignis). Brückentroll, Ferne Lande, Kundschafter, Riese, Wildhüter, Fernstraße, Gasthaus, Kartograph, Seidenstraße, Wegkreuzung
	\\
	\smallskip
	\\
	\emph{Friedhofspolka (mit Dark Ages)}
	\\
	\smallskip
	\\
	Almosen (Ereignis). Amulett, Gefolgsmann, Karawanenwächter, Kleinbauer (+ Eintauschkarten), Relikt, Barde, Grabräuber, Marodeur, Prozession, Schurke, Unterschlupf-Karten
	\\
	\smallskip
	\\
	\emph{Starker Verfall (mit Dark Ages)}
	\\
	\smallskip
	\\
	Verlorene Kunst (Ereignis), Wegsuche (Ereignis). Geisterwald, Rattenfänger, Transformation, Verlies, Zerstörung, Festung, Kultist, Leichenkarren, Ratten, Ritter, Unterschlupf-Karten
	\\
	\smallskip
	\\
	\emph{Verschwender (mit Die Gilden)}
	\\
	\smallskip
	\\
	Verlorene Kunst (Ereignis). Ausrüstung, Elster, Geizhals, Geschichtenerzähler, Kunsthandwerker, Arzt, Kaufmannsgilde, Meisterstück, Steinmetz, Wahrsager
	\\
	\smallskip
	\\
	\emph{Königin der Sonnenbräune (mit Die Gilden)}
	\\
	Wegsuche (Ereignis), Zuflucht (Ereignis). Duplikat, Königliche Kutsche, Königliche Münzen, Kundschafter, Rattenfänger, Berater, Herold, Leuchtenmacher, Metzger, Wandergeselle
	\\}}
\end{tikzpicture}
\hspace{1cm}