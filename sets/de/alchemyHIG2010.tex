% Basic settings for this card set
\renewcommand{\cardcolor}{alchemy}
\renewcommand{\cardextension}{Erweiterung II}
\renewcommand{\cardextensiontitle}{Die Alchemisten}
\renewcommand{\seticon}{alchemy.png}

\clearpage
\newpage
\section{\cardextension \ - \cardextensiontitle \ (Hans Im Glück 2010)}

\begin{tikzpicture}
	\card
	\cardstrip
	\cardbanner{banner/green.png}
	\cardicon{icons/potion.png}
	\cardtitle{Weinberg}
	\cardcontent{Diese Königreichkarte ist eine Punktekarte, keine Aktionskarte. Sie hat bis zum Ende des Spiels keine Funktion. Bei der Wertung zählt sie 1 Punkt pro volle 3 Aktionskarten im gesamten Kartensatz (Nachziehstapel, Ablagestapel und Handkarten) des Spielers. Du zählst alle deine Aktionskarten bei Spielende, teilst die Anzahl durch 3 und rundest ab. Kombinierte Aktionskarten sind auch Aktionskarten. Für 11 Aktionskarten erhältst du beispielsweise für jeden deiner Weinberge 3 Punkte. Im Spiel zu 3. und 4. werden 12 Karten verwendet, im Spiel zu 2. werden 8 Karten verwendet.}
\end{tikzpicture}
\hspace{-0.6cm}
\begin{tikzpicture}
	\card
	\cardstrip
	\cardbanner{banner/white.png}
	\cardicon{icons/potion.png}
	\cardtitle{Verwandlung}
	\cardcontent{Hast du keine Karte mehr auf der Hand, die du entsorgen könntest, erhältst du nichts. Wenn du einen Fluch entsorgst erhältst du nichts. Der Fluch ist keine Punkte-, keine Aktions- und keine Geldkarte. Entsorgst du eine Karte mit kombiniertem Kartentyp, erhältst du den Bonus für beide Kartentypen. Für die Adeligen (Dominion – Die Intrige) nimmst du dir z. B. ein Herzogtum und ein Gold. Die Karten nimmst du dir vom Vorrat. Ist keine entsprechende Karte mehr im Vorrat, erhältst du nichts.
	}
\end{tikzpicture}
\hspace{-0.6cm}
\begin{tikzpicture}
	\card
	\cardstrip
	\cardbanner{banner/white.png}
	\cardicon{icons/coin.png}
	\cardprice{2}
	\cardtitle{\scriptsize{Kräuterkundiger}}
	\cardcontent{Wenn du diese Karte ausspielst erhältst du +1 Geld für die Kaufphase und du darfst eine weitere Karte kaufen. Wenn du den Kräuterkundigen ablegst (normalerweise in der Aufräumphase), darfst du eine Geldkarte, die vor dir ausliegt, zurück auf deinen Nachziehstapel legen, statt sie abzulegen. Wenn dein Nachziehstapel leer ist, legst du nur die Geldkarte als neuen Nachziehstapel bereit. Du entscheidest, in welcher Reihenfolge du die vor dir ausliegenden Karten ablegst. Hast du z. B. einen Kräuterkundigen, einen Alchemisten und einen Trank ausliegen, darfst du zuerst den Alchemisten zurück auf deinen Nachziehstapel legen, dann den Kräuterkundigen ablegen und dafür den Trank zurück auf deinen Nachziehstapel legen. Wenn du mehrere Kräuterkundige im Spiel hast, darfst du für jeden davon eine Geldkarte zurück auf deinen Nachziehstapel legen.
	}
\end{tikzpicture}
\hspace{-0.6cm}
\begin{tikzpicture}
	\card
	\cardstrip
	\cardbanner{banner/white.png}
	\cardicon{icons/coin.png}
	\cardprice{2}
	\cardiconaddition{icons/potion.png}
	\cardtitle{\quad \footnotesize{Apotheker}}
	\cardcontent{Wenn du diese Karte ausspielst musst du sofort eine Karte nachziehen. Dann deckst du die obersten 4 Karten von deinem Nachziehstapel auf. Sollte dein Nachziehstapel beim Aufdecken zu Ende gehen, mischt du deinen Ablagestapel. Hast du nach dem Mischen immernoch weniger als 4 Karten zum Aufdecken, deckst du nur so viele Karten auf wie möglich. Danach musst du alle aufgedeckten Kupfer- und Trankkarten auf die Hand nehmen. Die übrigen aufgedeckten Karten legst du in beliebiger Reihenfolge zurück auf deinen Nachziehstapel. Wenn dein Nachziehstapel nach dem aufdecken der Karten leer ist, werden die zurück gelegten Karten dein neuer Nachziehstapel.
	}
\end{tikzpicture}
\hspace{-0.6cm}
\begin{tikzpicture}
	\card
	\cardstrip
	\cardbanner{banner/white.png}
	\cardicon{icons/coin.png}
	\cardprice{2}
	\cardiconaddition{icons/potion.png}
	\cardtitle{\quad \footnotesize{Universität}}
	\cardcontent{Du darfst die eine Aktionskarte, die bis zu 5 Geld kostet, vom Vorrat nehmen. Du musst jedoch keine Karte nehmen. Kombinierte Aktionskarten sind auch Aktionskarten. Du darfst keine Karte mit Trank in den Kosten nehmen.
	}
\end{tikzpicture}
\hspace{-0.6cm}
\begin{tikzpicture}
	\card
	\cardstrip
	\cardbanner{banner/white.png}
	\cardicon{icons/coin.png}
	\cardprice{2}
	\cardiconaddition{icons/potion.png}
	\cardtitle{\quad Vision}
	\cardcontent{Zuerst decken alle Spieler die oberste Karte ihres Nachziehstapels auf. Du entscheidest dann Spieler für Spieler extra (auch bei dir selbst), ob er die aufgedeckte Karte auf seinen Ablagestapel oder zurück auf seinen Nachziehstapel legt. Danach deckst du solange Karten von deinem Nachziehstapel auf, bis du eine Karte aufgedeckt hast, die keine Aktionskarte ist. Nun nimmst du alle gerade aufgedeckten Karten auf die Hand (auch die Nicht-Aktionskarte). Ist bereits die erste Karte, die du dabei aufgedeckt hast, keine Aktionskarte, so nimmst du nur diese eine Karte auf die Hand. Wenn du nach dem Mischen deines Ablagestapels alle Karten aufgedeckt hast und nur Aktionskarten o en liegen, so nimmst du alle aufgedeckten Karten auf die Hand. Kombinierte Aktionskarten sind auch Aktionskarten. Die durch die erste Anweisung aufgedeckten Karten aller Spieler werden hierbei nicht mehr beachtet. Du kannst auch keine dieser Karten auf die Hand nehmen.
	}
\end{tikzpicture}
\hspace{-0.6cm}
\begin{tikzpicture}
	\card
	\cardstrip
	\cardbanner{banner/white.png}
	\cardicon{icons/coin.png}
	\cardprice{3}
	\cardiconaddition{icons/potion.png}
	\cardtitle{\quad Alchemist}
	\cardcontent{Wenn du diese Karte ausspielst musst du sofort 2 Karten nachziehen und darfst dann eine weitere Aktionskarte ausspielen. Wenn mindestens ein Trank im Spiel ist, darfst du in der Aufräumphase alle vor dir ausliegenden Alchemisten zurück auf deinen Nachziehstapel legen, statt sie abzulegen. Hast du mehrere Alchemisten im Spiel, darfst du für jeden einzeln entscheiden, ob du ihn normal ablegst oder zurück auf deinen Nachziehstapel legst. Ist dein Nachziehstapel leer, legst du den Alchemisten alleine als Nachziehstapel bereit. Du legst den oder die Alchemisten ab oder zurück auf deinen Nachziehstapel, bevor du die Karten am Ende deines Zuges nachziehst. Du darfst in der Kaufphase einen Trank ausspielen, auch wenn du keine Karte damit kaufst.
	}
\end{tikzpicture}
\hspace{-0.6cm}
\begin{tikzpicture}
	\card
	\cardstrip
	\cardbanner{banner/gold.png}
	\cardicon{icons/coin.png}
	\cardprice{3}
	\cardiconaddition{icons/potion.png}
	\cardtitle{\quad \tiny{Stein der Weisen}}
	\cardcontent{Diese Karte ist eine Geldkarte und eine Königreichkarte. Sie ist nur im Spiel, wenn sie als eine der 10 ausliegenden Königreichkarten für dieses Spiel ausgewühlt wurde. Der Stein der Weisen darf, wie andere Geldkarten, in der Kaufphase ausgespielt werden. Wenn du den Stein der Weisen ausspielst zählst du zunächst die momentane Anzahl der Karten in deinem Nachziehstapel und in deinem Ablagestapel. Dann zählst du die Anzahl beider Stapel zusammen, teilst die Summe durch 5 und rundest ab. Das Ergebnis gibt den Wert der Karte für diese Kaufphase an. Der Wert gilt für diese gesamte Kaufphase, auch wenn sich die Anzahl der Karten noch verändert. Spielt ein Spieler mehrere dieser Karten, so hat jede den ermittelten Wert. Wird die Karte in einem späteren Zug erneut ausgespielt, wird auch der Wert neu ermittelt. Du darfst beim Durchzählen deines Nachziehstapels weder die Karten ansehen, noch deren Reihenfolge verändern. Beim Durchzählen deines Ablagestapels darfst du die Karten ansehen und deren Reihenfolge verändern. Du zählst nur die Karten der beiden Stapel, ausgespielte, beiseite gelegte und Handkarten zählst du nicht mit. Du darfst keine Geldkarten mehr ausspielen, nachdem du eine Karte gekauft hast. Du darfst also weder den Stein der Weisen noch andere Geldkarten ausspielen, wenn du in dieser Kaufphase bereits eine Karte gekauft hast.
	}
\end{tikzpicture}
\hspace{-0.6cm}
\begin{tikzpicture}
	\card
	\cardstrip
	\cardbanner{banner/white.png}
	\cardicon{icons/coin.png}
	\cardprice{3}
	\cardiconaddition{icons/potion.png}
	\cardtitle{\quad \footnotesize{Vertrauter}}
	\cardcontent{Sind nicht mehr genügend Fluchkarten im Vorrat, wenn du den Vertrauten aus- spielst, werden die restlichen Fluchkarten, beginnend beim Spieler links von dir, in Spielerreihenfolge verteilt. Du ziehst immer eine Karte nach und darfst eine weitere Aktionskarte ausspielen, auch wenn keine Fluchkarten mehr im allgemeinen Vorrat sind. Die Fluchkarten legen die Spieler sofort auf ihren Ablagestapel.
	}
\end{tikzpicture}
\hspace{-0.6cm}
\begin{tikzpicture}
	\card
	\cardstrip
	\cardbanner{banner/white.png}
	\cardicon{icons/coin.png}
	\cardprice{4}
	\cardiconaddition{icons/potion.png}
	\cardtitle{\quad Golem}
	\cardcontent{Du deckst solange Karten von deinem Nachziehstapel auf, bis 2 Aktionskarten offen liegen, die keine Golemkarten sind. Dann legst du alle aufgedeckten Golemkarten und alle Karten, die keine Aktionskarten sind, ab. Wenn du auch nach dem Mischen deines Ablagestapels keine oder nur 1 Aktionskarte (außer dem Golem) aufdecken kannst, so führst du die Anweisung mit weniger als 2 Karten aus. Hast du auf diese Weise 1 oder 2 Aktionskarten aufgedeckt, musst du diese in beliebiger Reihenfolge ausspielen. Du darfst auf diese Weise aufgedeckte Aktionskarten nicht auf die Hand nehmen. Alle Anweisungen, die sich auf deine Handkarten beziehen, haben keinen Effekt auf die beiden aufgedeckten Karten. Ist z. B. eine der aufgedeckten Karten ein Thronsaal, so kannst du die andere der beiden Karten nicht für diesen Thronsaal auswühlen.
	}
\end{tikzpicture}
\hspace{-0.6cm}
\begin{tikzpicture}
	\card
	\cardstrip
	\cardbanner{banner/white.png}
	\cardicon{icons/coin.png}
	\cardprice{5}
	\cardtitle{Lehrling}
	\cardcontent{Wenn du eine Karte auf deiner Hand hast, musst du eine Karte entsorgen. Wenn du eine Karte, die 0 Geld kostet (z. B. Kupfer oder Fluch) entsorgst oder wenn du keine Karte mehr auf der Hand hast, ziehst du keine Karte nach. Ansonsten ziehst du für eine Karte, die x Geld kostet, X Karten nach. Zusätzlich dazu ziehst du 2 Karten nach, wenn die entsorgte Karte kostet. Wenn du z. B. die Karte Golem (4 Geld, Trank) entsorgst, ziehst du 6 Karten nach.
	}
\end{tikzpicture}
\hspace{-0.6cm}
\begin{tikzpicture}
	\card
	\cardstrip
	\cardbanner{banner/white.png}
	\cardicon{icons/coin.png}
	\cardprice{6}
	\cardiconaddition{icons/potion.png}
	\cardtitle{\quad \scriptsize{Besessenheit}}
	\cardcontent{\tiny{\begin{Spacing}{1}
	\vspace{1em}
	(Teil 1)

	Der Spieler links von dir ist der aktive Spieler bei dem Extra-Zug durch die Besessenheit. Kartenanweisungen, die sich auf den aktiven Spieler beziehen, betreffen also den Spieler links von dir und seinen Kartensatz. In den Kartenanweisungen wird der aktive Spieler meist mit \enquote{du} angesprochen. Du darfst alle Karten sehen, die der Spieler links von dir in seinem Extra-Zug sieht. Das betrifft auch die Karten, die er in der Aufräumphase für seinen nächsten Zug nachzieht. Du darfst auch alle Karten ansehen, die er ansehen darf, z. B. zur Seite gelegte Karten auf dem Eingeborenendorf (Dominion – Seaside). Und du darfst alle Kartenstapel durchzählen, die er durchzählen darf. Du füllst alle Entscheidungen für den Spieler links von dir, welche Karten er ausspielt und in welcher Reihenfolge, alle Entscheidungen, die durch Kartenanweisungen erlaubt werden und welche Karten er kauft. Alle Karten, die der Spieler links von dir in seinem Extra-Zug nimmt oder kauft, legst du auf deinen Ablagestapel. Dies gilt auch, wenn er die Karte aufgrund einer bestimmten Anweisung auf die Hand oder nehmen oder anderswo ablegen müsste. Du erhältst nur Karten, die er nimmt oder kauft, keine Marker, z. B. Piratenschiff (Dominion – Seaside). Wenn der Spieler links von dir in seinem Extra-Zug Karten entsorgt, werden diese zunächst zur Seite gelegt und am Ende des Zuges (nach der Aufräumphase) auf seinen Ablagestapel gelegt. Für die Kartenanweisung, die das Entsorgen der Karte fordert, gilt diese als entsorgt, z. B. könntest du das Bergwerk (Dominion – Die Intrige) entsorgen und + erhalten. Das Bergwerk wird dabei nicht auf den Müllstapel gelegt, der Spieler verliert die Karte also nicht. Karten anderer Spieler, die während dieses Zuges entsorgt werden (z. B. durch Angriffskarten, wie Trickser oder Saboteur, Dominion – Die Intrige), werden dauerhaft entsorgt. Karten, die weitergegeben werden (z. B. durch die Maskerade, Dominion – Die Intrige), erhält der Spieler am Ende des Zuges nicht zurück.
	\end{Spacing}}}
\end{tikzpicture}
\hspace{-0.6cm}
\begin{tikzpicture}
	\card
	\cardstrip
	\cardbanner{banner/white.png}
	\cardicon{icons/coin.png}
	\cardprice{6}
	\cardiconaddition{icons/potion.png}
	\cardtitle{\quad \scriptsize{Besessenheit}}
	\cardcontent{\tiny{\begin{Spacing}{1}
	\vspace{1em}
	(Teil 2)
	
	Karten, die in den Vorrat zurückgelegt werden (z. B. Botschafter, Dominion – Seaside), erhält der Spieler auch nicht zurück. Spielt der Spieler links von dir (auf deinen Wunsch hin) eine Angriffskarte, so bist du, wie alle übrigen Spieler, normal betroffen. Du kannst, wie üblich, mit Reaktionskarten aus deiner eigenen Hand auf den Angriff reagieren. Du kannst keine Reaktionskarten des Spielers links von dir verwenden um auf den Angriff zu reagieren. Besessenheit bewirkt einen Extra-Zug, wie z. B. der Außenposten (Dominion – Seaside). Dieser Extra-Zug  findet erst nach deinem Zug statt. Du hast also alle Karten abgelegt und die Handkarten für deinen nächsten Zug nachgezogen. Der Außenposten verhindert nur einen weiteren Extra-Zug durch einen weiteren Außenposten. Du kannst weitere Extrazüge durch andere Karten, wie z. B. die Besessenheit erhalten. Wenn du in deinem Zug den Außenposten (Dominion – Seaside) und die Besessenheit spielst, führst du zuerst den Extra-Zug für den Außenposten aus, danach den Extra-Zug für die Besessenheit. Wenn der Spieler links von dir (auf deinen Wunsch hin) einen Außenposten spielt, erhält er dadurch einen Extra-Zug. In diesem Extra-Zug füllt er selbst seine Entscheidungen und erhält auch Karten, die er nimmt oder kauft selbst. Wenn der Spieler links von dir (auf deinen Wunsch hin) eine weitere Besessenheit ausspielt, so wird ein weiterer Extra-Zug gespielt, in dem der Spieler links von dir (nicht du) die Entscheidungen für den Spieler links von ihm füllt. Extra-Züge (z. B. durch Besessenheit oder Außenposten) werden für die Siegbedingung nicht beachtet. Im Gegensatz zum Außenposten ist die Besessenheit keine Dauer-Karte und wird in der Aufräumphase abgelegt. Die Wirkung der Karte Besessenheit ist kumulativ, spielst du z. B. die Karte in deinem Zug zweimal, so werden auch 2 Extra-Züge durchgeführt. Wichtig: Der Extrazug durch die Besessenheit ist nicht dein Extrazug, sondern der Extrazug des Spielers links von dir. Nach dem Extrazug (oder den Extrazügen) führt der Spieler seinen normalen Zug aus.
	\end{Spacing}}}
\end{tikzpicture}
\hspace{-0.6cm}
\begin{tikzpicture}
	\card
	\cardstrip
	\cardbanner{banner/gold.png}
	\cardicon{icons/coin.png}
	\cardprice{4}
	\cardtitle{Trank}
\end{tikzpicture}
\hspace{-0.6cm}
\begin{tikzpicture}
	\card
	\cardstrip
	\cardbanner{banner/white.png}
	\cardtitle{\scriptsize{Empfohlene 10er Sätze\qquad}}
	\cardcontent{\emph{Verbotene Künste} (Alchemisten + \textit{Basisspiel}\\
	Besessenheit, Lehrling, Universität, Vertrauter, \textit{Dieb}, \textit{Gärten}, \textit{Keller}, \textit{Laboratorium}, \textit{Ratsversammlung}, \textit{Thronsaal}

	\smallskip

	\emph{Quacksalber:} (Alchemisten + \textit{Basisspiel}):\\
	Alchemist, Apotheker, Golem, Kräuterkundiger, Verwandlung, \textit{Jahrmarkt}, \textit{Kanzler}, \textit{Keller}, \textit{Miliz}, \textit{Schmiede}

	\smallskip

	\emph{Chemiestunde:} (Alchemisten + \textit{Basisspiel}):\\
	Alchemist, Golem, Stein der Weisen, Universität, \textit{Burggraben}, \textit{Bürokrat}, \textit{Hexe}, \textit{Holzfäller}, \textit{Markt}, \textit{Umbau}

	\smallskip

	\emph{Diener:} (Alchemisten + \textit{Die Intrige}):\\
	Besessenheit, Golem, Verwandlung, Vision, Weinberg, \textit{Große Halle}, \textit{Handlanger}, \textit{Lakai}, \textit{Verschwörer}, \textit{Verwalter}

	\smallskip

	\emph{Geheime Forschungen:} (Alchemisten + \textit{Die Intrige}):\\
	Kräuterkundiger, Stein der Weisen, Universität, Vertrauter, \textit{Adlige}, \textit{Armenviertel}, \textit{Brücke}, \textit{Kerkermeister}, \textit{Lakai}, \textit{Maskerade}

	\smallskip

	\emph{Tröpfe, Tränke, Trottel:} (Alchemisten + \textit{Die Intrige}):\\
	Apotheker, Golem, Lehrling, Vision, \textit{Adlige}, \textit{Baron}, \textit{Eisenhütte}, \textit{Handelsposten}, \textit{Kupferschmied}, \textit{Wunschbrunnen}}
\end{tikzpicture}
\hspace{0.6cm}

