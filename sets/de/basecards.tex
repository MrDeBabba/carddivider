% Basic settings for this card set
\renewcommand{\cardcolor}{}
\renewcommand{\cardextension}{}
\renewcommand{\cardextensiontitle}{}

\clearpage
\newpage
\section{Anleitung und Grundausstattung}

\begin{tikzpicture}
	\card
	\cardbanner{banner/white.png}
	\cardtitle{Anleitung (1)\quad}
	\cardcontent{\tiny{\emph{Spielablauf:} Dominion wird zugweise gespielt. Der Spieler an der Reihe hat am Beginn seines Zuges normalerweise 5 Karten auf der Hand. Er führt nun seinen Zug aus, der aus den 3 folgenden Phasen besteht, die immer in dieser Reihenfolge gespielt werden müssen.

	\medskip

	\emph{1. Phase:} Aktion - Der Spieler \emph{darf} Aktionskarten ausspielen.

	\emph{2. Phase:} Kauf - Der Spieler \emph{darf} Karten kaufen.

	\emph{3. Phase:} Aufräumen - Der Spieler \emph{muss} alle  ausgespielten \emph{und} alle Handkarten offen auf seinen Ablagestapel legen und \emph{sofort} 5 Karten für den nächsten Zug nachziehen.

	\medskip

	Die 1. und die 2. Phase darf, die 3. Phase muss gespielt werden. Wenn der Spieler seinen Zug beendet hat, ist der nächste Spieler an der Reihe. Das Spiel verläuft in dieser Weise bis Spielende.
	
	\medskip

	\begin{tabbing}
		Stapel der anderen Spieler: \= Links \kill
		Eigener Ablagestapel: \> Darf weder durchgezählt noch durchgesehen werden. \\
		Eigener Nachziehstapel: \> Darf durchgezählt, nicht aber durchgesehen werden. \\
		Stapel der anderen Spieler: \> Dürfen weder durchgezählt noch durchgesehen werden. \\
		Stapel im Vorrat: \> Dürfen jederzeit durchgezählt und durchgesehen werden. \\
		Müllstapel: \> Darf jederzeit durchgezählt und durchgesehen werden. \\
	\end{tabbing}}}
\end{tikzpicture}
\hspace{-0.6cm}
\begin{tikzpicture}
	\card
	\cardbanner{banner/white.png}
	\cardtitle{Anleitung (2)\quad}
	\cardcontent{\tiny{\begin{tabbing}
	Spielende:xxx \=  18x Provinzen,xxx \= 12 andere Punktekarten,xxx \= 50 Flüche,xxx \= Geld: \kill
	2 Spieler: \> 8x Provinzen, \> 8 andere Punktekarten, \> 10 Flüche, \> Geld: 46 K, 40 S, 30 G\\
	3 Spieler: \> 12x Provinzen, \> 12 andere Punktekarten, \> 20 Flüche, \> Geld: 39 K, 40 S, 30 G\\
	4 Spieler: \> 12x Provinzen, \> 12 andere Punktekarten, \> 30 Flüche, \> Geld: 32 K, 40 S, 30 G\\
	5 Spieler: \> 15x Provinzen, \> 12 andere Punktekarten, \> 40 Flüche, \> Geld: 85 K, 80 S, 60 G\\
	6 Spieler: \> 18x Provinzen, \> 12 andere Punktekarten, \> 50 Flüche, \> Geld: 78 K, 80 S, 60 G\\
	Spielende: \> Provinzstapel, Kolonienstapel (Dominion – Blütezeit) \emph{oder} \\
					\>3 Stapel (1 - 4 Spieler) bzw. 4 Stapel (5 - 6 Spieler) aus dem Vorrat leer \\
	\end{tabbing}
	Punktekarten aus Erweiterungen werden in Anzahl der \enquote{anderen Punktekarten} ausgelegt. Zur Spielende-Bedingung zählen alle Karten im Vorrat, also auch Fluch-, Geld- und Punktekarten, nicht jedoch z. B. der Müllstapel.

	\medskip

	\emph{Platin und Kolonie (Dominion – Blütezeit):} Die im Spiel befindlichen Königreich-Platzhalterkarten werden gemischt. Ist die erste gezogene Königreichskarte eine Karte aus der Blütezeit-Erweiterung, so wird mit Platin und Kolonie gespielt, ansonsten ohne.

	\medskip

	\emph{Unterschlupf-Karten (Dominion – Dark Ages):} Die im Spiel befindlichen Königreich-Platzhalterkarten werden gemischt. Ist die erste gezogene Königreichskarte eine Karte aus der Dark Ages-Erweiterung, so startet jeder Spieler mit 7 Kupfer, 1 Hütte, 1 Totenstadt und 1 Verfallenes Anwesen, andernfalls erhält jeder Spieler 7 Kupfer und 3 Anwesen.}}
\end{tikzpicture}
\hspace{-0.6cm}
\begin{tikzpicture}
	\card
	\cardbanner{banner/white.png}
	\cardtitle{Platzhalter\quad}
\end{tikzpicture}
\hspace{-0.6cm}
\begin{tikzpicture}
	\card
	\cardbanner{banner/gold.png}
	\cardicon{banner/coin.png}
	\cardprice{0}
	\cardtitle{Kupfer}
\end{tikzpicture}
\hspace{-0.6cm}
\begin{tikzpicture}
	\card
	\cardbanner{banner/gold.png}
	\cardicon{banner/coin.png}
	\cardprice{3}
	\cardtitle{Silber}
\end{tikzpicture}
\hspace{-0.6cm}
\begin{tikzpicture}
	\card
	\cardbanner{banner/gold.png}
	\cardicon{banner/coin.png}
	\cardprice{6}
	\cardtitle{Gold}
\end{tikzpicture}
\hspace{-0.6cm}
\begin{tikzpicture}
	\card
	\cardbanner{banner/green.png}
	\cardicon{banner/coin.png}
	\cardprice{2}
	\cardtitle{Anwesen}
\end{tikzpicture}
\hspace{-0.6cm}
\begin{tikzpicture}
	\card
	\cardbanner{banner/green.png}
	\cardicon{banner/coin.png}
	\cardprice{5}
	\cardtitle{Herzogtum}
\end{tikzpicture}
\hspace{-0.6cm}
\begin{tikzpicture}
	\card
	\cardbanner{banner/green.png}
	\cardicon{banner/coin.png}
	\cardprice{8}
	\cardtitle{Provinz}
\end{tikzpicture}
\hspace{-0.6cm}
\begin{tikzpicture}
	\card
	\cardbanner{banner/purple.png}
	\cardicon{banner/coin.png}
	\cardprice{0}
	\cardtitle{Fluch}
\end{tikzpicture}	
\hspace{0.6cm}
