% Basic settings for this card set
\renewcommand{\cardcolor}{basicgame}
\renewcommand{\cardextension}{Edition I}
\renewcommand{\cardextensiontitle}{Das Basisspiel}
\renewcommand{\seticon}{basic1.png}

\clearpage
\newpage
\section{\cardextension \ - \cardextensiontitle \ (Hans Im Glück 2008)}

\begin{tikzpicture}
	\card
	\cardstrip
	\cardbanner{banner/blue.png}
	\cardicon{icons/coin.png}
	\cardprice{2}
	\cardtitle{Burggraben}
	\cardcontent{Angriffskarten sind mit der Aufschrift \enquote{Angriff} (normalerweise \enquote{Aktion - Angriff}) gekennzeichnet. Spielt ein anderer Spieler eine Angriffskarte aus, kannst du die Karte Burggraben vorzeigen, falls du sie auf der Hand hast. Du nimmst den Burggraben zurück auf deine Hand, bevor der Angriff abgewickelt wird. Du bist vom Angriff nicht betroffen: Du musst dir keine Fluchkarte von der Hexe nehmen, musst beim Spion keine Karten aufdecken usw. Es ist so, als wärst du nicht im Spiel. Dein Burggraben hat keine Auswirkungen auf die übrigen Mitspieler, diese sind wie üblich vom Angriff betroffen. Für den \enquote{Angriffsspieler} selbst gilt: Auch wenn einer oder mehrere Spieler einen Burggraben vorzeigen, kann er die übrigen Anweisungen der Angriffskarte durchführen. Spielt er z. B. eine Hexe aus, so zieht er trotzdem 2 Karten nach.

	\medskip

	Spielst du den Burggraben in deinem Zug aus, musst du 2 Karten nachziehen.}
\end{tikzpicture}
\hspace{-0.6cm}
\begin{tikzpicture}
	\card
	\cardstrip
	\cardbanner{banner/white.png}
	\cardicon{icons/coin.png}
	\cardprice{2}
	\cardtitle{Keller}
	\cardcontent{Du kannst den ausgespielten Keller selbst nicht ablegen, da du die Karte nicht mehr auf der Hand hältst, wenn du die Anweisung ausführst. Du sagst zuerst an, wieviele Handkarten du ablegst, legst diese Karten auf deinen Ablagestapel und ziehst dann die gleiche Anzahl Karten vom Nachziehstapel. Geht der Nachziehstapel zu Ende, werden die gerade abgelegten Karten wieder eingemischt.}
\end{tikzpicture}
\hspace{-0.6cm}
\begin{tikzpicture}
	\card
	\cardstrip
	\cardbanner{banner/white.png}
	\cardicon{icons/coin.png}
	\cardprice{3}
	\cardtitle{Dorf}
	\cardcontent{Spielst du mehrere Dörfer aus, zählst du laut mit, wie viele Aktionskarten du insgesamt noch ausspielen darfst, z.B.: Ich spiele das Dorf und darf noch 2 Aktionskarten ausspielen. Ich spiele einen Markt und darf noch insgesamt 2 Aktionskarten ausspielen. Ich spiele ein Dorf und darf noch insgesamt 3 Aktionskarten ausspielen ...}
\end{tikzpicture}
\hspace{-0.6cm}
\begin{tikzpicture}
	\card
	\cardstrip
	\cardbanner{banner/white.png}
	\cardicon{icons/coin.png}
	\cardprice{3}
	\cardtitle{Werkstatt}
	\cardcontent{Du nimmst dir eine Karte aus dem Vorrat und legst diese sofort auf deinen Ablagestapel. Du kannst keine Geldkarten oder virtuelles Geld von anderen Aktionskarten verwenden, um den Betrag zu erhöhen.}
\end{tikzpicture}
\hspace{-0.6cm}
\begin{tikzpicture}
	\card
	\cardstrip
	\cardbanner{banner/white.png}
	\cardicon{icons/coin.png}
	\cardprice{3}
	\cardtitle{Holzfäller}
	\cardcontent{In der Kaufphase darfst du +2 Geld zu deinen ausgelegten Geldkarten hinzuzählen und du darfst eine weitere Karte kaufen.}
\end{tikzpicture}
\hspace{-0.6cm}
\begin{tikzpicture}
	\card
	\cardstrip
	\cardbanner{banner/white.png}
	\cardicon{icons/coin.png}
	\cardprice{4}
	\cardtitle{Schmiede}
	\cardcontent{Du musst 3 Karten von deinem Nachziehstapel nachziehen.}
\end{tikzpicture}
\hspace{-0.6cm}
\begin{tikzpicture}
	\card
	\cardstrip
	\cardbanner{banner/white.png}
	\cardicon{icons/coin.png}
	\cardprice{4}
	\cardtitle{Umbau}
	\cardcontent{Du kannst den ausgespielten Umbau selbst nicht entsorgen, da du die Karte nicht mehr auf der Hand hältst, wenn du die Anweisung ausführst. Hast du einen weiteren Umbau auf der Hand, kannst du diesen entsorgen. Hast du keine Karte auf der Hand, die du entsorgen kannst, nimmst du dir auch keine neue Karte. Die Karte, die du nimmst, kann bis zu 2 Geld mehr kosten als die entsorgte Karte. Du kannst keine Geldkarten oder virtuelles Geld von anderen Aktionskarten verwenden, um den Betrag zu erhöhen. Du kannst auch eine Karte entsorgen und dir eine identische Karte aus dem Vorrat nehmen.}
\end{tikzpicture}
\hspace{-0.6cm}
\begin{tikzpicture}
	\card
	\cardstrip
	\cardbanner{banner/white.png}
	\cardicon{icons/coin.png}
	\cardprice{4}
	\cardtitle{Miliz}
	\cardcontent{Deine Mitspieler müssen Handkarten ablegen, bis sie nur noch 3 Karten auf der Hand haben. Spieler, die bereits 3 oder weniger Karten auf der Hand haben, wenn die Miliz ausgespielt wird, müssen keine Karten ablegen.}
\end{tikzpicture}
\hspace{-0.6cm}
\begin{tikzpicture}
	\card
	\cardstrip
	\cardbanner{banner/white.png}
	\cardicon{icons/coin.png}
	\cardprice{5}
	\cardtitle{Markt}
	\cardcontent{Du musst zuerst eine Karte nachziehen. Dann kannst du eine weitere Aktionskarte ausspielen. In der Kaufphase hast du einen weiteren Kauf und +1 Geld zur Verfügung.}
\end{tikzpicture}
\hspace{-0.6cm}
\begin{tikzpicture}
	\card
	\cardstrip
	\cardbanner{banner/white.png}
	\cardicon{icons/coin.png}
	\cardprice{5}
	\cardtitle{Mine}
	\cardcontent{Normalerweise entsorgst du eine Kupferkarte und nimmst dir dafür eine Silberkarte aus dem Vorrat auf die Hand oder du entsorgst eine Silberkarte und nimmst dir dafür eine Goldkarte auf die Hand. Du kannst jedoch auch eine identische oder eine kleinere Geldkarte nehmen. Die aufgenommene Karte kannst du noch in diesem Zug einsetzen. Hast du keine Geldkarte, die du entsorgen kannst, erhältst du nichts.

	\medskip

	Du darfst mit der Mine (neben Kupfer, Silber und Gold) auch Königreich-Geldkarten (z. B. Füllhorn) entsorgen und nehmen. \emph{Achtung:} Du darfst das Diadem (Dominion – Reiche Ernte) nicht nehmen, da dies eine Preiskarte ist und nicht zum Vorrat gehört.}
\end{tikzpicture}
\hspace{-0.6cm}
\begin{tikzpicture}
	\card
	\cardstrip
	\cardbanner{banner/white.png}
	\cardicon{icons/coin.png}
	\cardprice{6}
	\cardtitle{Abenteurer}
	\cardcontent{Die aufgedeckten Karten legst du zunächst offen vor dir aus. Sollte dein Nachziehstapel beim Aufdecken zu Ende gehen, mischt du deinen Ablagestapel. Die bereits aufgedeckten Karten mischst du jedoch nicht mit ein, da diese Karten erst am Ende der Aktion auf den Ablagestapel gelegt werden. Hast du keine Karten mehr zum Aufdecken und noch immer nur eine Geldkarte, bekommst du nur diese eine.}
\end{tikzpicture}
\hspace{-0.6cm}
\begin{tikzpicture}
	\card
	\cardstrip
	\cardbanner{banner/white.png}
	\cardicon{icons/coin.png}
	\cardprice{5}
	\cardtitle{Bibliothek}
	\cardcontent{Du kannst Aktionskarten zur Seite legen, musst dies jedoch nicht. Hast du bereits 7 oder mehr Karten auf der Hand, wenn du die Bibliothek ausspielst, ziehst du keine Karten nach. Solltest du den Ablagestapel mischen müssen, mischst du die zur Seite gelegten Karten jedoch nicht mit ein. Diese Karten legst du erst auf den Ablagestapel, sobald du 7 Karten auf der Hand hast. Sollten die Karten nicht ausreichen, ziehst du nur so viele Karten nach wie möglich.}
\end{tikzpicture}
\hspace{-0.6cm}
\begin{tikzpicture}
	\card
	\cardstrip
	\cardbanner{banner/white.png}
	\cardicon{icons/coin.png}
	\cardprice{4}
	\cardtitle{Bürokrat}
	\cardcontent{Ist dein Nachziehstapel aufgebraucht, wenn du diese Karte ausspielst, legst du die Silberkarte verdeckt ab. Dein eigener Nachziehstapel besteht dann nur aus dieser Karte. Das Gleiche gilt für die übrigen Spieler, die eine Karte auf ihren eigenen Nachziehstapel legen müssen.}
\end{tikzpicture}
\hspace{-0.6cm}
\begin{tikzpicture}
	\card
	\cardstrip
	\cardbanner{banner/white.png}
	\cardicon{icons/coin.png}
	\cardprice{4}
	\cardtitle{Dieb}
	\cardcontent{Die beiden aufgedeckten Karten legen deine Mitspieler zunächst o en vor sich aus. Hat ein Spieler nur noch eine Karte im eigenen Nachziehstapel, deckt er diese auf, mischt seinen Ablagestapel (ohne die gerade aufgedeckte Karte) und deckt dann die zweite Karte auf. Hat ein Spieler keine Karten mehr im eigenen Nachziehstapel, mischt er sofort und deckt dann 2 Karten auf. Hat ein Spieler nach dem Mischen immer noch nicht genug Karten, deckt er nur so viele auf wie möglich. Jeder Spieler entsorgt höchstens eine Geldkarte, nach deiner Wahl. Dann nimmst du beliebig viele der gerade (nicht in früheren Zügen) entsorgten Karten und legst sie auf deinen Ablagestapel. Die übrigen aufgedeckten Karten legen die Spieler auf ihren Ablagestapel.}
\end{tikzpicture}
\hspace{-0.6cm}
\begin{tikzpicture}
	\card
	\cardstrip
	\cardbanner{banner/white.png}
	\cardicon{icons/coin.png}
	\cardprice{4}
	\cardtitle{Festmahl}
	\cardcontent{Du nimmst dir 1 Karte, die bis zu 5 Geld kostet, aus dem Vorrat und legst diese sofort auf deinen Ablagestapel. Du kannst keine Geldkarten oder virtuelles Geld von anderen Aktionskarten verwenden, um den Betrag zu erhöhen. Spielst du das Festmahl nach dem Thronsaal, erhältst du 2 Karten, obwohl du das Festmahl nur einmal entsorgen kannst.}
\end{tikzpicture}
\hspace{-0.6cm}
\begin{tikzpicture}
	\card
	\cardstrip
	\cardbanner{banner/green.png}
	\cardicon{icons/coin.png}
	\cardprice{4}
	\cardtitle{Gärten}
	\cardcontent{Diese Königreichkarte ist eine Punktekarte, keine Aktionskarte. Sie hat bis zum Ende des Spiels keine Funktion. Bei der Wertung zählt sie 1 Punkt pro volle 10 Karten im gesamten Kartensatz des Spielers. Du zählst alle deine Karten bei Spielende (auch die Gärten), teilst die Anzahl durch 10 und rundest ab. Für 39 Karten erhältst du beispielsweise 3 Punkte.}
\end{tikzpicture}
\hspace{-0.6cm}
\begin{tikzpicture}
	\card
	\cardstrip
	\cardbanner{banner/white.png}
	\cardicon{icons/coin.png}
	\cardprice{4}
	\cardtitle{\footnotesize{Geldverleiher}}
	\cardcontent{Hast du kein Kupfer auf der Hand, das du entsorgen könntest, so erhältst du auch kein virtuelles Geld für die Kaufphase.}
\end{tikzpicture}
\hspace{-0.6cm}
\begin{tikzpicture}
	\card
	\cardstrip
	\cardbanner{banner/white.png}
	\cardicon{icons/coin.png}
	\cardprice{5}
	\cardtitle{Hexe}
	\cardcontent{Sind nicht mehr genügend Fluchkarten im Vorrat, wenn du die Hexe ausspielst, werden die restlichen Fluchkarten, beginnend beim Spieler links von dir, in Spielreihenfolge verteilt. Du ziehst immer 2 Karten nach, auch wenn keine Fluchkarten mehr im allgemeinen Vorrat sind. Die Fluchkarten legen die Spieler sofort auf ihren Ablagestapel.}
\end{tikzpicture}
\hspace{-0.6cm}
\begin{tikzpicture}
	\card
	\cardstrip
	\cardbanner{banner/white.png}
	\cardicon{icons/coin.png}
	\cardprice{5}
	\cardtitle{Jahrmarkt}
	\cardcontent{Spielst du mehrere Jahrmärkte aus, zählst du laut mit, wie viele Aktionskarten du noch ausspielen darfst.
	}
\end{tikzpicture}
\hspace{-0.6cm}
\begin{tikzpicture}
	\card
	\cardstrip
	\cardbanner{banner/white.png}
	\cardicon{icons/coin.png}
	\cardprice{3}
	\cardtitle{Kanzler}
	\cardcontent{Wenn du deinen Nachziehstapel auf den Ablagestapel legst, musst du das tun, bevor du weitere Aktionskarten ausspielst oder die Aktionsphase beendest. Du darfst deinen Nachziehstapel nicht durchsehen, bevor du ihn ablegst.

	\medskip

	\emph{Anmerkung:} Durch das direkte Ablegen wird die Karte Tunnel (Dominion – Hinterland) nicht ausgelöst.\\}
\end{tikzpicture}
\hspace{-0.6cm}
\begin{tikzpicture}
	\card
	\cardstrip
	\cardbanner{banner/white.png}
	\cardicon{icons/coin.png}
	\cardprice{2}
	\cardtitle{Kapelle}
	\cardcontent{Du kannst die ausgespielte Kapelle selbst nicht entsorgen, da du die Karte nicht mehr auf der Hand hältst, wenn du die Anweisung ausführst. Hast du eine weitere Kapelle auf der Hand, kannst du diese entsorgen.}
\end{tikzpicture}
\hspace{-0.6cm}
\begin{tikzpicture}
	\card
	\cardstrip
	\cardbanner{banner/white.png}
	\cardicon{icons/coin.png}
	\cardprice{5}
	\cardtitle{\footnotesize{Laboratorium}}
	\cardcontent{Du musst zuerst 2 Karten nachziehen und kannst dann eine weitere Aktionskarte ausspielen.}
\end{tikzpicture}
\hspace{-0.6cm}
\begin{tikzpicture}
	\card
	\cardstrip
	\cardbanner{banner/white.png}
	\cardicon{icons/coin.png}
	\cardprice{5}
	\cardtitle{\scriptsize{Ratsversammlung}}
	\cardcontent{Die Mitspieler müssen eine Karte nachziehen, auch wenn sie nicht wollen.}
\end{tikzpicture}
\hspace{-0.6cm}
\begin{tikzpicture}
	\card
	\cardstrip
	\cardbanner{banner/white.png}
	\cardicon{icons/coin.png}
	\cardprice{4}
	\cardtitle{Spion}
	\cardcontent{Du musst zuerst 1 Karte nachziehen. Erst danach decken alle Spieler (auch du selbst) die oberste Karte von ihrem Nachziehstapel auf. Du entscheidest dann bei jedem Spieler (auch bei dir selbst), ob er die aufgedeckte Karte auf seinen Ablagestapel oder zurück auf seinen Nachziehstapel legt.

	\medskip

	Hat ein Spieler keine Karte mehr im Nachziehstapel, so mischt er seinen Ablagestapel und deckt dann die oberste Karte auf. Spieler, die dann immer noch keine Karten haben, decken keine Karte auf. Wenn den Spielern die Reihenfolge wichtig ist, deckst du zuerst eine Karte auf, danach folgen die Mitspieler in Spielreihenfolge.}
\end{tikzpicture}
\hspace{-0.6cm}
\begin{tikzpicture}
	\card
	\cardstrip
	\cardbanner{banner/white.png}
	\cardicon{icons/coin.png}
	\cardprice{4}
	\cardtitle{Thronsaal}
	\cardcontent{Wenn du den Thronsaal ausspielst, wählst du 1 weitere Aktionskarte aus deiner Hand, legst diese o en aus und führst die angegebene(n) Anweisung(en) aus. Dann nimmst du diese Karte zurück auf die Hand, legst sie ein weiteres Mal aus und führst die angegebene(n) Anweisung(en) nochmals aus. Dieses Auslegen der Aktionskarte kostet keine Aktionen. Du musst die Anweisung(en) der Karte soweit möglich vollständig ausführen, bevor du sie zurück auf die Hand nimmst und ein weiteres Mal auslegst. Legst du einen weiteren Thronsaal nach einem Thronsaal aus, legst du eine Aktionskarte zweimal aus und danach eine weitere ebenfalls zweimal. Du kannst nicht ein und dieselbe Karte 4 mal auslegen. Wenn die nach dem Thronsaal ausgelegte Karte \enquote{+1 Aktion} erlaubt (wie z. B. der Markt), darfst du anschließend noch 2 weitere Aktionen ausführen. Würdest du 2 Märkte regulär ausspielen, dürftest du nur noch eine zusätzliche Aktion ausführen, da das Ausspielen des zweiten Markts eine Aktion aufbraucht. Hier ist es besonders wichtig, die verbleibenden Aktionen laut mitzuzählen. Du kannst keine andere Aktionskarte ausspielen, bevor der Thronsaal vollständig abgehandelt ist.}
\end{tikzpicture}
\hspace{-0.6cm}
\begin{tikzpicture}
	\card
	\cardstrip
	\cardbanner{banner/white.png}
	\cardtitle{\scriptsize{Empfohlene 10er Sätze\qquad}}
	\cardcontent{\emph{Erstes Spiel:}\\
	Burggraben, Dorf, Holzfäller, Keller, Markt, Miliz, Mine, Schmiede, Umbau, Werkstatt

	\smallskip

	\emph{Großes Geld:}\\
	Abenteurer, Bürokrat, Festmahl, Geldverleiher, Kanzler, Kapelle, Laboratorium, Mark, Mine, Thronsaal

	\smallskip

	\emph{Interaktion:}\\
	Bibliothek, Burggraben, Bürokrat, Dieb, Dorf, Jahrmarkt, Kanzler, Miliz, Ratsversammlung, Spion

	\smallskip

	\emph{Im Wandel:}\\
	Dieb, Dorf, Festmahl, Gärten, Hexe, Holzfäller, Kapelle, Keller, Laboratorium, Werkstatt

	\smallskip

	\emph{Dorfplatz:}\\
	Bibliothek, Bürokrat, Dorf, Holzfäller, Jahrmarkt, Keller, Markt, Schmiede, Thronsaal, Umbau}
\end{tikzpicture}
\hspace{0.6cm}
