% Basic settings for this card set
\renewcommand{\cardcolor}{basicgame}
\renewcommand{\cardextension}{Special Edition}
\renewcommand{\cardextensiontitle}{Das Basisspiel}
\renewcommand{\seticon}{basic1.png}

\clearpage
\newpage
\section{\cardextension \ - \cardextensiontitle \ (Rio Grande Games 2013)}

\begin{tikzpicture}
	\card
	\cardstrip
	\cardbanner{banner/blue.png}
	\cardicon{icons/coin.png}
	\cardprice{2}
	\cardtitle{Burggraben}
	\cardcontent{Spielt ein anderer Spieler eine Angriffskarte (mit der Aufschrift AKTION -- ANGRIFF), kannst du die Karte \emph{BURGGRABEN} vorzeigen, falls du sie in diesem Moment auf der Hand hast. In diesem Fall bist du von den Auswirkungen des Angriffs nicht betroffen, d.h. du musst bei der \emph{HEXE} keine Fluchkarte nehmen usw. Haben mehrere Spieler einen \emph{BURGGRABEN} auf der Hand, dürfen diese auch eingesetzt und vorgezeigt werden. Danach nehmen die Spieler ihre Karte zurück auf die Hand. Der Spieler, der den Angriff gespielt hat, darf unabhängig davon, ob ein oder mehrere \emph{BURGGRÄBEN} gespielt werden, die weiteren Anweisungen seiner Aktionskarte ausführen. Der \emph{BURGGRABEN} darf auch in der eigenen Aktionsphase gespielt werden – dann ziehst du 2 Karten nach.}
\end{tikzpicture}
\hspace{-0.6cm}
\begin{tikzpicture}
	\card
	\cardstrip
	\cardbanner{banner/white.png}
	\cardicon{icons/coin.png}
	\cardprice{2}
	\cardtitle{Keller}
	\cardcontent{Der ausgespielte \emph{KELLER} selbst darf nicht abgelegt werden, da er sich nicht mehr in deiner Hand befindet. Sage an, wie viele Karten du ablegst und lege diese auf deinen Ablagestapel. Danach ziehst du die gleiche Anzahl Karten vom Nachziehstapel. Sollte während dieses Vorgangs der Nachziehstapel aufgebraucht werden, wird dein Ablagestapel zusammen mit den soeben abgelegten Karten gemischt und als neuer Nachziehstapel bereitgelegt.}
\end{tikzpicture}
\hspace{-0.6cm}
\begin{tikzpicture}
	\card
	\cardstrip
	\cardbanner{banner/white.png}
	\cardicon{icons/coin.png}
	\cardprice{3}
	\cardtitle{Dorf}
	\cardcontent{Spielst du mehrere \emph{DÖRFER} hintereinander, zählst du am besten laut mit, wie viele Aktionen du noch ausspielen darfst, damit du den Überblick behältst.}
\end{tikzpicture}
\hspace{-0.6cm}
\begin{tikzpicture}
	\card
	\cardstrip
	\cardbanner{banner/white.png}
	\cardicon{icons/coin.png}
	\cardprice{3}
	\cardtitle{Werkstatt}
	\cardcontent{Nimm dir eine Karte aus dem Vorrat und lege diese sofort auf deinen Ablagestapel. Du kannst weder Geldkarten noch zusätzlich über Aktionskarten erhaltenes Geld oder Münzen (bei Erweiterungen mit Münzen) einsetzen, um den angegebenen Betrag auf der Karte zu erhöhen.}
\end{tikzpicture}
\hspace{-0.6cm}
\begin{tikzpicture}
	\card
	\cardstrip
	\cardbanner{banner/white.png}
	\cardicon{icons/coin.png}
	\cardprice{3}
	\cardtitle{Holzfäller}
	\cardcontent{In der Kaufphase darfst du 1 zusätzliche Karte kaufen, also insgesamt 2 Käufe tätigen. Für deine Käufe stehen dir in diesem Zug insgesamt 2 Geld zusätzlich zur Verfügung.}
\end{tikzpicture}
\hspace{-0.6cm}
\begin{tikzpicture}
	\card
	\cardstrip
	\cardbanner{banner/white.png}
	\cardicon{icons/coin.png}
	\cardprice{4}
	\cardtitle{Schmiede}
	\cardcontent{Du musst 3 Karten von deinem Nachziehstapel ziehen und auf die Hand nehmen.}
\end{tikzpicture}
\hspace{-0.6cm}
\begin{tikzpicture}
	\card
	\cardstrip
	\cardbanner{banner/white.png}
	\cardicon{icons/coin.png}
	\cardprice{4}
	\cardtitle{Umbau}
	\cardcontent{Der ausgespielte \emph{UMBAU} selbst darf nicht entsorgt werden, da er sich nicht mehr in deiner Hand befindet. Weitere \emph{UMBAU}-Karten in deiner Hand dürfen entsorgt werden. Wenn du keine Karte zum Entsorgen auf der Hand hast, darfst du dir auch keine neue Karte nehmen. Die neue Karte, die du dir nimmst, darf maximal bis zu \coin[2] mehr als die entsorgte Karte kosten. Der Betrag darf weder durch weitere Geldkarten, Münzen oder zusätzliches Geld von anderen Aktionskarten erhöht werden. Die neue Karte kann die gleiche Karte sein wie die, die du entsorgt hast. Lege die neue Karte auf deinen Ablagestapel.}
\end{tikzpicture}
\hspace{-0.6cm}
\begin{tikzpicture}
	\card
	\cardstrip
	\cardbanner{banner/white.png}
	\cardicon{icons/coin.png}
	\cardprice{4}
	\cardtitle{Miliz}
	\cardcontent{Deine Mitspieler müssen Karten aus ihrer Hand ablegen, bis sie nur noch 3 Karten auf der Hand haben. Spieler, die zum Zeitpunkt des Angriffs bereits 3 oder weniger Karten auf der Hand haben, müssen keine weiteren Karten ablegen.}
\end{tikzpicture}
\hspace{-0.6cm}
\begin{tikzpicture}
	\card
	\cardstrip
	\cardbanner{banner/white.png}
	\cardicon{icons/coin.png}
	\cardprice{5}
	\cardtitle{Markt}
	\cardcontent{Du musst eine Karte vom Nachziehstapel auf die Hand nehmen. Du \emph{darfst} in der Aktionsphase eine weitere Aktionskarte ausspielen. Du \emph{darfst} in der Kaufphase einen zusätzlichen Kauf tätigen und hast dafür ein zusätzliches Geld zur Verfügung.}
\end{tikzpicture}
\hspace{-0.6cm}
\begin{tikzpicture}
	\card
	\cardstrip
	\cardbanner{banner/white.png}
	\cardicon{icons/coin.png}
	\cardprice{5}
	\cardtitle{Mine}
	\cardcontent{\emph{Errata:} Der Kartentext ist falsch, es sollte \enquote{Du \emph{darfst} eine beliebige Geldkarte aus der Hand entsorgen. Nimm eine Geldkarte vom Vorrat auf die Hand, die bis zu \coin[3] mehr kostet.} heißen.

	\medskip

	Normalerweise entsorgst du ein Kupfer und nimmst dir dafür ein Silber, oder du entsorgst ein Silber und nimmst dir ein Gold. Du kannst dir aber auch eine gleichwertige oder billigere Karte nehmen. Die neue Karte nimmst du sofort auf die Hand und darfst sie noch während deines Zuges einsetzen. Wer keine Geldkarte zum Entsorgen hat, erhält keine neue Karte.}
\end{tikzpicture}
\hspace{-0.6cm}
\begin{tikzpicture}
	\card
	\cardstrip
	\cardbanner{banner/white.png}
	\cardicon{icons/coin.png}
	\cardprice{6}
	\cardtitle{Abenteurer}
	\cardcontent{Sollte dein Nachziehstapel während des Aufdeckens aufgebraucht werden, mische deinen Ablagestapel. Die bereits aufgedeckten Karten werden nicht mit gemischt, sondern bleiben zunächst offen liegen. Solltest du auch mit Hilfe des neuen Nachziehstapels nicht genügend Geldkarten aufdecken, bekommst du nur diejenigen Geldkarten, die du aufgedeckt hast.}
\end{tikzpicture}
\hspace{-0.6cm}
\begin{tikzpicture}
	\card
	\cardstrip
	\cardbanner{banner/white.png}
	\cardicon{icons/coin.png}
	\cardprice{5}
	\cardtitle{Bibliothek}
	\cardcontent{Aktionskarten darfst du zur Seite legen, sobald du sie ziehst, musst dies aber nicht tun. Hast du bereits 7 oder mehr Karten auf der Hand, wenn du die \emph{BIBLIOTHEK} ausspielst, ziehst du keine Karten nach. Wenn dein Nachziehstapel während des Ziehens aufgebraucht ist, mischst du den Ablagestapel, mischst aber die zur Seite gelegten Aktionskarten nicht mit ein. Diese werden erst auf den Ablagestapel gelegt, sobald du 7 Karten auf der Hand hast. Sollten die Karten nicht reichen, ziehst du nur so viele Karten wie möglich.}
\end{tikzpicture}
\hspace{-0.6cm}
\begin{tikzpicture}
	\card
	\cardstrip
	\cardbanner{banner/white.png}
	\cardicon{icons/coin.png}
	\cardprice{4}
	\cardtitle{Bürokrat}
	\cardcontent{Ist dein Nachziehstapel aufgebraucht, wenn du diese Karte spielst, legst du die Silberkarte verdeckt ab. Sie bildet dann deinen Nachziehstapel. Das Gleiche gilt für alle Mitspieler, die eine Punktekarte verdeckt auf den eigenen Nachziehstapel legen müssen.}
\end{tikzpicture}
\hspace{-0.6cm}
\begin{tikzpicture}
	\card
	\cardstrip
	\cardbanner{banner/white.png}
	\cardicon{icons/coin.png}
	\cardprice{4}
	\cardtitle{Dieb}
	\cardcontent{Jeder Mitspieler legt die beiden aufgedeckten Karten zunächst offen vor sich ab. Wer nur noch 1 Karte im Nachziehstapel hat, legt diese vor sich ab und mischt erst dann seinen Ablagestapel. Hat ein Spieler nach dem Mischen noch immer nicht genug Karten, deckt er nur so viele auf wie möglich. Hat ein Spieler 2 Geldkarten offen liegen, wählst du eine davon aus, die der Spieler entsorgen muss. Die andere Karte legt er auf seinen Ablagestapel. Hat ein Spieler 1 Geldkarte offen liegen, muss er diese entsorgen. Hat ein Spieler keine Geldkarte aufgedeckt, muss er keine Karte entsorgen. Von den auf diese Weise entsorgten Karten darfst du eine beliebige Anzahl nehmen.}
\end{tikzpicture}
\hspace{-0.6cm}
\begin{tikzpicture}
	\card
	\cardstrip
	\cardbanner{banner/white.png}
	\cardicon{icons/coin.png}
	\cardprice{4}
	\cardtitle{Festmahl}
	\cardcontent{Du nimmst dir eine beliebige Karte aus dem Vorrat, die höchstens \coin[5] kostet und legst sie sofort auf deinen Ablagestapel. Du darfst den Betrag weder durch weitere Geldkarten, Münzen oder zusätzliches Geld von anderen Aktionskarten erhöhen. Spielst du das \emph{FESTMAHL} direkt nach dem \emph{THRONSAAL}, erhältst du 2 Karten, obwohl du das \emph{FESTMAHL} nur einmal entsorgen kannst.}
\end{tikzpicture}
\hspace{-0.6cm}
\begin{tikzpicture}
	\card
	\cardstrip
	\cardbanner{banner/green.png}
	\cardicon{icons/coin.png}
	\cardprice{4}
	\cardtitle{Gärten}
	\cardcontent{Diese Karte ist die einzige Punktekarte unter den Königreichkarten. Sie hat bis zum Ende des Spiels keine Funktion. Bei der Wertung des Spiels erhält der Spieler, der diese Karte in seinem Kartensatz (Nachziehstapel, Handkarten und Ablagestapel) hat, für jeweils 10 Karten einen Siegpunkt. Es wird immer abgerundet, d.h. 39 Karten ergeben 3 Siegpunkte, ebenso wie 31 Karten 3 Siegpunkte ergeben. Wer mehrere \emph{GÄRTEN} besitzt, erhält jeden \emph{GARTEN} die entsprechende Anzahl an Siegpunkten.}
\end{tikzpicture}
\hspace{-0.6cm}
\begin{tikzpicture}
	\card
	\cardstrip
	\cardbanner{banner/white.png}
	\cardicon{icons/coin.png}
	\cardprice{4}
	\cardtitle{\footnotesize{Geldverleiher}}
	\cardcontent{\emph{Errata:} Der Kartentext ist falsch, es sollte \enquote{Du \emph{darfst} ein Kupfer aus der Hand entsorgen. Wenn du das tust: + \coin[3].} heißen.

	\medskip

	Wenn du kein Kupfer zum Entsorgen auf der Hand hast, erhältst du kein zusätzliches Geld für die Kaufphase.}
\end{tikzpicture}
\hspace{-0.6cm}
\begin{tikzpicture}
	\card
	\cardstrip
	\cardbanner{banner/white.png}
	\cardicon{icons/coin.png}
	\cardprice{5}
	\cardtitle{Hexe}
	\cardcontent{Wenn du die \emph{HEXE} spielst und nicht mehr genügend Fluchkarten vorrätig sind, werden diese im Uhrzeigersinn (beginnend mit deinem linken Nachbarn) verteilt. Die Mitspieler legen die Fluchkarten sofort auf ihren Ablagestapel. Du ziehst immer 2 Karten von deinem Nachziehstapel, auch wenn keine Fluchkarten mehr im Vorrat sind.}
\end{tikzpicture}
\hspace{-0.6cm}
\begin{tikzpicture}
	\card
	\cardstrip
	\cardbanner{banner/white.png}
	\cardicon{icons/coin.png}
	\cardprice{5}
	\cardtitle{Jahrmarkt}
	\cardcontent{Spielst du mehrere \emph{JAHRMÄRKTE} hintereinander, zählst du am besten laut mit, wie viele Aktionen du noch ausspielen darfst, damit du den Überblick behältst.}
\end{tikzpicture}
\hspace{-0.6cm}
\begin{tikzpicture}
	\card
	\cardstrip
	\cardbanner{banner/white.png}
	\cardicon{icons/coin.png}
	\cardprice{3}
	\cardtitle{Kanzler}
	\cardcontent{Legst du deinen Nachziehstapel auf deinen Ablagestapel, musst du dies tun, bevor du eine weitere Aktion ausspielst oder zur Kaufphase übergehst. Du darfst deinen Nachziehstapel nicht einsehen, bevor du ihn ablegst.}
\end{tikzpicture}
\hspace{-0.6cm}
\begin{tikzpicture}
	\card
	\cardstrip
	\cardbanner{banner/white.png}
	\cardicon{icons/coin.png}
	\cardprice{2}
	\cardtitle{Kapelle}
	\cardcontent{Die ausgespielte \emph{KAPELLE} selbst darf nicht entsorgt werden, da sie sich nicht mehr auf der Hand befindet. Weitere \emph{KAPELLEN} auf der Hand dürfen entsorgt werden.}
\end{tikzpicture}
\hspace{-0.6cm}
\begin{tikzpicture}
	\card
	\cardstrip
	\cardbanner{banner/white.png}
	\cardicon{icons/coin.png}
	\cardprice{5}
	\cardtitle{\footnotesize{Laboratorium}}
	\cardcontent{Du \emph{musst} zuerst zwei Karten vom Nachziehstapel auf die Hand nehmen. Dann \emph{darfst} du eine weitere Aktionskarte ausspielen.}
\end{tikzpicture}
\hspace{-0.6cm}
\begin{tikzpicture}
	\card
	\cardstrip
	\cardbanner{banner/white.png}
	\cardicon{icons/coin.png}
	\cardprice{5}
	\cardtitle{\scriptsize{Ratsversammlung}}
	\cardcontent{Jeder Mitspieler \emph{muss} eine Karte von seinem Nachziehstapel auf die Hand nehmen.}
\end{tikzpicture}
\hspace{-0.6cm}
\begin{tikzpicture}
	\card
	\cardstrip
	\cardbanner{banner/white.png}
	\cardicon{icons/coin.png}
	\cardprice{4}
	\cardtitle{Spion}
	\cardcontent{\emph{Zuerst} musst du eine Karte vom Nachziehstapel auf die Hand nehmen. Dann \emph{muss} jeder Spieler (auch du) die oberste Karte seines Nachziehstapels aufdecken. Du entscheidest für jeden Spieler, ob die aufgedeckte Karte auf den Nachziehstapel zurück- oder auf den Ablagestapel abgelegt werden soll. Spieler, deren Nachziehstapel aufgebraucht ist, mischen ihren Ablagestapel und legen ihn als neuen Nachziehstapel bereit. Nur wer weder einen Nachzieh- noch einen Ablagestapel hat, braucht keine Karte aufzudecken. Ist den Mitspielern die Reihenfolge des Aufdeckens wichtig, deckst du zuerst auf – die anderen Spieler folgen im Uhrzeigersinn.}
\end{tikzpicture}
\hspace{-0.6cm}
\begin{tikzpicture}
	\card
	\cardstrip
	\cardbanner{banner/white.png}
	\cardicon{icons/coin.png}
	\cardprice{4}
	\cardtitle{Thronsaal}
	\cardcontent{\emph{Errata:} Der Kartentext ist falsch, es sollte \enquote{Du \emph{darfst} eine beliebige Aktionskarte aus der Hand zweimal ausspielen.} heißen.

	\medskip

	Wähle eine Aktionskarte aus deiner Hand und spiele sie zweimal aus, d. h. du legst die Aktionskarte aus, führst die Anweisungen der Karte komplett aus, nimmst die Karte zurück auf die Hand, legst sie noch einmal aus und führst die Anweisungen erneut aus. Für das doppelte Ausspielen dieser Aktionskarte muss der Spieler keine zusätzlichen Aktionen (+1 Aktion) zur Verfügung haben – sie ist sozusagen \enquote{kostenlos}. Legst du zwei \emph{THRONSAAL}-Karten aus, darfst du zuerst eine Aktion doppelt ausführen und dann eine andere Aktion ebenfalls doppelt ausführen. Du darfst aber nicht ein und dieselbe Aktion viermal ausführen.

	\medskip

	Erlaubt die doppelt ausgespielte Karte +1 Aktion (z.B. der \emph{MARKT}), hast du nach der vollständigen Ausführung des \emph{THRONSAALS} zwei weitere Aktionen zur Verfügung. Hättest du zwei \emph{MARKT}-Karten regulär hintereinander ausgespielt, bliebe dir nur noch eine zusätzliche Aktion zur Verfügung, da das Ausspielen der zweiten Marktkarte schon die zusätzliche Aktion der ersten Karte aufgebraucht hätte. Beim \emph{THRONSAAL} ist es besonders wichtig, laut die verbleibende Anzahl an Aktionen mitzuzählen. Du darfst keine weitere Aktion ausspielen, bevor der \emph{THRONSAAL} komplett abgearbeitet ist.}
\end{tikzpicture}
\hspace{-0.6cm}
\begin{tikzpicture}
	\card
	\cardstrip
	\cardbanner{banner/white.png}
	\cardicon{icons/coin.png}
	\cardprice{3}
	\cardtitle{\footnotesize{Schwarzmarkt}}
	\cardcontent{\tiny{Habt ihr den \emph{SCHWARZMARKT} als eine der 10 Königreichkarten-Sätze ausgewählt, müsst ihr vor Spielbeginn zusätzlich einen verdeckten \emph{SCHWARZMARKT}-Stapel bilden.

	\medskip

	\emph{Der SCHWARZMARKT-Stapel ...}
	... darf nur aus Königreichkarten bestehen, die nicht im Spiel sind, sich also nicht im Vorrat be nden. Er muss mindestens 15 Karten umfassen.

	\medskip

	Ihr einigt euch darauf, welche Königreichkarten ihr im \emph{SCHWARZMARKT}-Stapel haben wollt. Das können auch mehr als 15 Karten sein, ja sogar alle (nicht verwendeten) König- reichkarten dürfen im \emph{SCHWARZMARKT}-Stapel vorkommen.

	\medskip

	Dann kommt von jeder ausgewählten Königreichkarte eine Karte in den \emph{SCHWARZMARKT}-Stapel. Alle Spieler dürfen sich noch mal die Karten im Stapel anschauen. Danach wird der \emph{SCHWARZMARKT}-Stapel gemischt und als verdeckter Stapel neben dem Vorrat bereitgelegt. Dieser Stapel gehört nicht zum Vorrat.

	\medskip

	\emph{Der SCHWARZMARKT-Kauf}
	Er  findet in der Aktionsphase statt, d.h. schon vor der eigentlichen Kaufphase. Zunächst deckt der Spieler die obersten drei Karten des \emph{SCHWARZMARKT}-Stapels auf. Dann darf er beliebig viele Geldkarten ausspielen und eine der drei aufgedeckten Karten kaufen, wenn das Geld (ausgespielte Geldkarten und Geldwerte auf Aktionskarten) dazu reicht. Er darf auch darauf verzichten, eine der aufgedeckten Karten zu kaufen. Nicht gekaufte Karten werden in beliebiger Reihenfolge zurück unter den \emph{SCHWARZMARKT}-Stapel gelegt. Die ausgespielten Geldkarten bleiben zunächst offen liegen. Nicht verwendetes Geld, Geldwerte und Münzen darf der Spieler noch in der Kaufphase seines Zuges nutzen.

	\smallskip}}
\end{tikzpicture}
\hspace{-0.6cm}
\begin{tikzpicture}
	\card
	\cardstrip
	\cardbanner{banner/white.png}
	\cardtitle{\scriptsize{Empfohlene 10er Sätze\qquad}}
	\cardcontent{\emph{Erstes Spiel:}\\
	Burggraben, Dorf, Holzfäller, Keller, Markt, Miliz, Mine, Schmiede, Umbau, Werkstatt

	\smallskip

	\emph{Großes Geld:}\\
	Abenteurer, Bürokrat, Festmahl, Geldverleiher, Kanzler, Kapelle, Laboratorium, Mark, Mine, Thronsaal

	\smallskip

	\emph{Interaktion:}\\
	Bibliothek, Burggraben, Bürokrat, Dieb, Dorf, Jahrmarkt, Kanzler, Miliz, Ratsversammlung, Spion

	\smallskip

	\emph{Im Wandel:}\\
	Dieb, Dorf, Festmahl, Gärten, Hexe, Holzfäller, Kapelle, Keller, Laboratorium, Werkstatt

	\smallskip

	\emph{Dorfplatz:}\\
	Bibliothek, Bürokrat, Dorf, Holzfäller, Jahrmarkt, Keller, Markt, Schmiede, Thronsaal, Umbau}
\end{tikzpicture}
\hspace{0.6cm}
