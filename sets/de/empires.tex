% Basic settings for this card set
\renewcommand{\cardcolor}{empires}
\renewcommand{\cardextension}{Erweiterung IX}
\renewcommand{\cardextensiontitle}{Empires}

\clearpage
\newpage
\section{\cardextension \ - \cardextensiontitle}

\begin{tikzpicture}
	\card
	\cardstrip
	\cardbanner{banner/white.png}
	\cardicon{banner/coin.png}
	\cardprice{2}
	\cardtitle{Feldlager}
	\cardcontent{Du darfst ein Gold oder ein Diebesgut aus der Hand aufdecken. Wenn du das nicht kannst oder möchtest, legst du diese Karte zur Seite und legst sie zu Beginn deiner Aufräumphase zurück in den Vorrat. Sollte dort zu diesem Zeitpunkt bereits ein Diebesgut offen liegen, muss nun erst wieder das zurückgelegte Feldlager genommen werden, bevor das Diebesgut genommen werden darf.}
\end{tikzpicture}
\hspace{-0.6cm}
\begin{tikzpicture}
	\card
	\cardstrip
	\cardbanner{banner/white.png}
	\cardicon{banner/coin.png}
	\cardprice{2}
	\cardtitle{Patrizier}
	\cardcontent{Du musst die oberste Karte deines Nachziehstapels aufdecken. Wenn der Stapel aufgebraucht ist, mischst du deinen Ablagestapel und legst ihn als Nachziehstapel bereit. Wenn auch dort keine Karten liegen, erhältst du nichts.

	\medskip

	Nur eine Karte, die mehr als 5 Geld kostet, darfst du auf die Hand nehmen. Ob die Karte außerdem noch Kosten in Form von Schulden aufweist, ist dabei unerheblich (z. B. Reichtum darf auf die Hand genommen werden, Stadtviertel nicht).}
\end{tikzpicture}
\hspace{-0.6cm}
\begin{tikzpicture}
	\card
	\cardstrip
	\cardbanner{banner/white.png}
	\cardicon{banner/coin.png}
	\cardprice{2}
	\cardtitle{Siedler}
	\cardcontent{Auch wenn du weißt, dass sich kein Kupfer in deinem Ablagestapel befindet, darfst du ihn ansehen. Du musst kein Kupfer auf die Hand nehmen, wenn du das nicht möchtest.}
\end{tikzpicture}
\hspace{-0.6cm}
\begin{tikzpicture}
	\card
	\cardstrip
	\cardbanner{banner/white.png}
	\cardicon{banner/coin.png}
	\cardprice{3}
	\cardtitle{\footnotesize{Bauernmarkt}}
	\cardcontent{Diese Karte beinhaltet den neuen Typ Sammlung, d. h. hier kommen die Siegpunkt-Marker zum Einsatz.

	\medskip

	Wenn diese Karte das erste Mal ausgespielt wird, legt der Spieler einen Siegpunkt-Marker auf den Bauernmarkt-Vorratsstapel und erhält dann +1 Geld für den gerade gelegten Marker. Wird die Karte zum zweiten, dritten und vierten Mal ausgespielt, legt der Spieler jeweils einen weiten Marker auf den Stapel und erhält +2 Geld, +3 Geld bzw. +4 Geld, egal welcher Spieler die vorherigen Marker auf den Stapel gelegt hat. Wird die Karte danach erneut ausgespielt, nimmt der Spieler die 4 Siegpunkt-Marker (dafür aber kein Geld) und muss den ausgespielten Bauernmarkt entsorgen.

	\medskip

	Danach beginnt der Vorgang wieder von vorn und wird fortgesetzt, auch wenn der Vorratsstapel leer ist.}
\end{tikzpicture}
\hspace{-0.6cm}
\begin{tikzpicture}
	\card
	\cardstrip
	\cardbanner{banner/white.png}
	\cardicon{banner/coin.png}
	\cardprice{3}
	\cardtitle{Gladiator}
	\cardcontent{Wenn du mindestens 1 Handkarte hast, musst du diese aufdecken. Wenn dein linker Mitspieler keine Karte mit gleichem Namen aufdecken kann oder will (z. B. auch, wenn du keine Handkarte aufdecken konntest, weil du keine hast), erhältst du zusätzlich +1 Geld. Sind noch Karten auf dem Gladiator-Vorratsstapel vorhanden, musst du eine entsorgen. Deckt der Mitspieler eine Karte mit gleichem Namen auf, erhältst du nur +2 Geld und darfst keinen Gladiator entsorgen.}
\end{tikzpicture}
\hspace{-0.6cm}
\begin{tikzpicture}
	\card
	\cardstrip
	\cardbanner{banner/white.png}
	\cardicon{banner/coin.png}
	\cardprice{3}
	\cardtitle{Katapult}
	\cardcontent{Wenn du mindestens 1 Handkarte hast, musst du auch eine entsorgen. Kostet die entsorgte Karte 3 Geld oder mehr, nimmt sich jeder Mitspieler (beginnend bei deinem linken Nachbarn) einen Fluch. Karten mit Schulden kosten nur dann 3 Geld oder mehr, wenn sie zusätzlich zu etwaigen Schulden-Kosten mindestens 3 Geld kosten. Ist die entsorgte Karte eine Geldkarte muss jeder Mitspieler - unabhängig von den Kosten der Karte - seine Handkarten auf 3 Geld reduzieren.}
\end{tikzpicture}
\hspace{-0.6cm}
\begin{tikzpicture}
	\card
	\cardstrip
	\cardbanner{banner/white.png}
	\cardicon{banner/coin.png}
	\cardprice{3}
	\cardtitle{\footnotesize{Wagenrennen}}
	\cardcontent{Nimm deine aufgedeckte Karte nach dem Vergleich der Kosten mit der aufgedeckten Karte deines linken Mitspielers auf die Hand. Der Mitspieler legt seine aufgedeckte Karte zurück auf den Nachziehstapel.

	\medskip

	Kosten beide Karten gleich viel oder kostet die Karte des Mitspielers mehr, erhältst du nichts. Kostet deine Karte mehr erhältst du +1 Geld und +1 Siegpunkt-Marker. Hast entweder du oder dein linker Mitspieler (auch nach dem eventuellen Mischen des Ablagestapels) keine Karte zum Aufdecken, erhältst du nichts.}
\end{tikzpicture}
\hspace{-0.6cm}
\begin{tikzpicture}
	\card
	\cardstrip
	\cardbanner{banner/orange.png}
	\cardicon{banner/coin.png}
	\cardprice{3}
	\cardtitle{Zauberin}
	\cardcontent{Spieler, die mit einer Reaktionskarte wie dem Burggraben (aus dem Basisspiel) reagieren möchten, müssen dies tun, sobald die Zauberin ausgespielt wurde, auch wenn der Angriff sie erst in ihrem nächsten Zug betrifft.

	\medskip

	Jeder Mitspieler erhält in seinem nächsten Zug für die erste gespielte Aktionskarte +1 Karte sowie +1 Aktion, darf aber den eigentlichen Effekt der Karte beim Ausspielen nicht durchführen. Anweisungen, die sich auf einen anderen Zeitpunkt im Spiel beziehen (z. B. die beim Kauf der Karte zum Tragen kommen), werden nicht beeinflusst.

	\medskip

	Um anzuzeigen, dass die erste ausgespielte Aktionskarte von der Zauberin beeinflusst wird, empfehlen wir, diese beim Ausspielen quer auszulegen. Karten, die bereits ausgespielt wurden (z. B. Dauerkarten wie das Archiv), werden zu Beginn des Zuges normal abgehandelt und nicht von der Zauberin beeinflusst. Spielt ein Spieler in seiner Aktionsphase keine Aktionskarte aus, dafür aber in seiner Kaufphase eine Krone (kombinierte Aktions- und Geldkarte), kommt der Effekt der Zauberin zum Tragen, da es sich um eine Aktionskarte handelt, auch wenn diese in der Kaufphase ausgespielt wurde. Normalerweise kann der Spieler die +1 Aktion zu diesem Zeitpunkt nicht nutzen, es sei denn, er kauft zum Beispiel eine Villa.}
\end{tikzpicture}
\hspace{-0.6cm}
\begin{tikzpicture}
	\card
	\cardstrip
	\cardbanner{banner/gold.png}
	\cardicon{banner/coin.png}
	\cardprice{4}
	\cardtitle{Felsen}
	\cardcontent{Wenn du diese Karte in deiner Kaufphase nimmst oder entsorgst, nimm ein Silber und lege es auf deinen Nachziehstapel. Wenn du diese Karte zu einem anderen Zeitpunkt (auch während des Zuges eines anderen Spielers) nimmst oder entsorgst, nimm ein Silber auf die Hand.}
\end{tikzpicture}
\hspace{-0.6cm}
\begin{tikzpicture}
	\card
	\cardstrip
	\cardbanner{banner/white.png}
	\cardicon{banner/coin.png}
	\cardprice{4}
	\cardtitle{Opfer}
	\cardcontent{Wenn die entsorgte Karte eine kombinierte Karte ist, erhältst du die Boni aller entsprechenden Typen dieser Karte. Entsorgst du eine Karte, die keinem der angegebenen Typen entspricht (z. B. einen Fluch), erhältst du nichts.}
\end{tikzpicture}
\hspace{-0.6cm}
\begin{tikzpicture}
	\card
	\cardstrip
	\cardbanner{banner/white.png}
	\cardicon{banner/coin.png}
	\cardprice{4}
	\cardtitle{Tempel}
	\cardcontent{Es dürfen nur Karten mit unterschiedlichem Namen entsorgt werden, z. B. ein Kupfer und ein Anwesen.

	\medskip

	Auch wenn der Tempel-Vorratsstapel leer ist, legst du einen Siegpunkt-Marker auf den leeren Platz. Das kann relevant werden, wenn durch Anweisungen auf anderen Karten ein Tempel in den Vorrat zurückgelegt wird (z. B. durch den Botschafter aus Seaside).

	\medskip

	Wenn du einen Tempel nimmst, nimmst du auch alle Siegpunkt-Marker, die zu diesem Zeitpunkt auf dem Vorratsstapel liegen.}
\end{tikzpicture}
\hspace{-0.6cm}
\begin{tikzpicture}
	\card
	\cardstrip
	\cardbanner{banner/white.png}
	\cardicon{banner/coin.png}
	\cardprice{4}
	\cardtitle{Villa}
	\cardcontent{Wenn du diese Karte in deiner Aktionsphase nimmst (z. B. durch die Ingenieurin), nimm sie sofort auf die Hand und erhalte +1 Aktion. Dadurch kannst du z. B. die gerade genommene Villa sofort ausspielen. Wenn du diese Karte in deiner Kaufphase nimmst (z. B. indem du sie kaufst), nimm sie auf die Hand und kehre sofort in die Aktionsphase zurück, wo du +1 Aktion hast. Hast du die Aktionsphase erneut komplett abgeschlossen, kehrst du wieder zur Kaufphase zurück. Hier kannst du weitere Geldkarten ausspielen (und z. B. die Arena kommt wieder zum Tragen). Wenn du diese Karte während des Zuges eines Mitspielers nimmst, nimmst du die Karte auf die Hand und erhältst zwar +1 Aktion, kannst diese aber nicht nutzen, da es nicht dein Zug ist. Es ist möglich, mehrmals pro Zug (z. B. durch das Nehmen mehrerer Villen) in die Aktionsphase zurückzukehren. Dies bedeutet aber nicht, dass du an den \enquote{Beginn deines Zuges} zurückkehrst. Anweisungen, die sich darauf beziehen, haben keine Auswirkung.}
\end{tikzpicture}
\hspace{-0.6cm}
\begin{tikzpicture}
	\card
	\cardstrip
	\cardbanner{banner/orange.png}
	\cardicon{banner/coin.png}
	\cardprice{5}
	\cardtitle{Archiv}
	\cardcontent{Lege die obersten drei Karten deines Nachziehstapels zur Seite und schau sie dir an. Nimm eine der Karten sofort auf die Hand und lege die anderen Karten unter dieses Archiv. Spielst du zwei Archive, lege die Karten für die nächsten Züge unter das jeweils ausgespielte Archiv. Hast du nicht genügend Karten, um drei Karten zur Seite zu legen, legst du nur so viele wie möglich zur Seite. Das Archiv wird in dem Spielzug abgelegt, in dem die letzte zur Seite gelegte Karte des jeweiligen Archivs auf die Hand genommen wurde.}
\end{tikzpicture}
\hspace{-0.6cm}
\begin{tikzpicture}
	\card
	\cardstrip
	\cardbanner{banner/gold.png}
	\cardicon{banner/coin.png}
	\cardprice{5}
	\cardtitle{Diebesgut}
	\cardcontent{Nimm dir jedes Mal, wenn du diese Karte spielst, einen Siegpunkt-Marker und lege ihn bei dir ab.}
\end{tikzpicture}
\hspace{-0.6cm}
\begin{tikzpicture}
	\card
	\cardstrip
	\cardbanner{banner/white.png}
	\cardicon{banner/coin.png}
	\cardprice{5}
	\cardtitle{Emsiges Dorf}
	\cardcontent{Du darfst deinen Ablagestapel auch dann durchsehen, wenn du weißt, dass du keine Siedler darin hast. Du darfst die Reihenfolge der Karten in deinem Ablagestapel nicht verändern.}
\end{tikzpicture}
\hspace{-0.6cm}
\begin{tikzpicture}
	\card
	\cardstrip
	\cardbanner{banner/white.png}
	\cardicon{banner/coin.png}
	\cardprice{5}
	\cardtitle{Forum}
	\cardcontent{Wenn du diese Karte kaufst, erhältst du +1 Kauf. Du kannst beispielsweise mit 13 Geld und nur einem freien Kauf, zuerst diese Karte kaufen und dann mit dem zusätzlichen Kauf noch eine Provinz.}
\end{tikzpicture}
\hspace{-0.6cm}
\begin{tikzpicture}
	\card
	\cardstrip
	\cardbanner{banner/white.png}
	\cardicon{banner/coin.png}
	\cardprice{5}
	\cardtitle{Gärtnerin}
	\cardcontent{Ist diese Karte im Spiel und du nimmst eine Punktekarte – egal in welcher Spielphase – nimmst du dir einen Siegpunkt-Marker und legst ihn bei dir ab. Wenn du mehrere Punktekarten nimmst, nimmst du dir für jede genommene Punktekarte einen Siegpunkt-Marker. Hast du mehrere Gärtnerinnen im Spiel, nimmst du dir für jede Gärtnerin pro genommener Punktekarte einen Siegpunkt-Marker.

	\medskip

	Wenn du z. B. eine Gärtnerin auf eine Krone spielst, befindet sich die Gärtnerin trotzdem nur einmal im Spiel und du darfst dir pro genommener Punktekarte nur einen Siegpunkt-Marker nehmen.}
\end{tikzpicture}
\hspace{-0.6cm}
\begin{tikzpicture}
	\card
	\cardstrip
	\cardbanner{banner/white.png}
	\cardicon{banner/coin.png}
	\cardprice{5}
	\cardtitle{\footnotesize{Handelsplatz}}
	\cardcontent{Zu den Aktionskarten, die du zu diesem Zeitpunkt im Spiel hast zählen alle Aktionskarten, die du ausgespielt hast, Dauerkarten, die sich aus vergangenen Zügen im Spiel befinden und Reservekarten (aus Abenteuer), die du in diesem Zug bereits aufgerufen hast. Wenn du diese Karte außerhalb deines Zuges nimmst, hast du keine Aktionskarten im Spiel und du darfst dir keine Siegpunkt-Marker nehmen.}
\end{tikzpicture}
\hspace{-0.6cm}
\begin{tikzpicture}
	\card
	\cardstrip
	\cardbanner{banner/whitegold.png}
	\cardicon{banner/coin.png}
	\cardprice{5}
	\cardtitle{Krone}
	\cardcontent{Diese Karte ist eine kombinierte Aktions- und Geldkarte. Wenn du sie in deiner Aktionsphase ausspielst, darfst du eine Aktionskarte von deiner Hand wählen und ausspielen. Du nimmst die gewählte Karte nicht wieder auf die Hand, sondern spielst die Aktion ein zweites Mal. Dafür benötigst du keine weiteren Aktionen. Wählst du eine Krone, musst du diese auch als Aktionskarte ausspielen (und dann darfst du bis zu zwei weitere Aktionskarten jeweils zweimal spielen).

	\medskip

	Spielst du diese Karte in deiner Aktionsphase als Geldkarte aus (z. B. durch den Geschichtenerzähler aus Abenteuer), darfst du trotzdem eine Aktionskarte zweimal ausspielen.

	\medskip

	Spielst du diese Karte in deiner Kaufphase, darfst du eine beliebige Geldkarte von deiner Hand wählen, sie ausspielen und zweimal ausführen. Wählst du eine Krone, spielst du diese aus und dann eine weitere Geldkarte von der Hand zweimal und dann noch eine Geldkarte zweimal.}
\end{tikzpicture}
\hspace{-0.6cm}
\begin{tikzpicture}
	\card
	\cardstrip
	\cardbanner{banner/white.png}
	\cardicon{banner/coin.png}
	\cardprice{5}
	\cardtitle{Legionär}
	\cardcontent{Mitspieler, die auf das Ausspielen dieser Karte mit einer Reaktionskarte reagieren möchten, müssen dies tun, bevor du dich entscheidest, ob du ein Gold aufdeckst oder nicht.

	\medskip

	Mitspieler, die bereits zwei oder weniger Karten auf der Hand haben, müssen keine Karte ablegen, müssen gleichwohl aber eine Karte ziehen.}
\end{tikzpicture}
\hspace{-0.6cm}
\begin{tikzpicture}
	\card
	\cardstrip
	\cardbanner{banner/gold.png}
	\cardicon{banner/coin.png}
	\cardprice{5}
	\cardtitle{Vermögen}
	\cardcontent{\emph{Errata:} Der Geldwert oben links und rechts auf der Karte muss 6 Geld lauten, nicht 5 Geld.

	\medskip

	Diese Karte ist eine Geldkarte mit zusätzlichen Anweisungen. Sie hat den Wert 6 Geld. Außerdem erhältst du +1 Kauf.

	\medskip

	Wenn du diese Karte ablegst (in der Regel in deiner Aufräumphase), nimm 6 Schulden-Marker vom Vorrat. Dann kannst du sofort beliebig viele Schulden-Marker (auch mehr als die 6 Schulden-Marker, die du durch das Ablegen dieser Karte erhalten hast) zurückzahlen. Wenn du diese Karte nicht ablegst (z. B. wenn du sie stattdessen entsorgst), erhältst du keine Schulden-Marker. Wenn du diese Karte zweimal ausgespielt hast (z. B. durch eine Krone), erhältst du trotzdem nur 6 Schulden-Marker, da du nur eine Karte ablegst.}
\end{tikzpicture}
\hspace{-0.6cm}
\begin{tikzpicture}
	\card
	\cardstrip
	\cardbanner{banner/white.png}
	\cardicon{banner/coin.png}
	\cardprice{5}
	\cardtitle{Wilde Jagd}
	\cardcontent{Wählst du die erste Option, lege einen Siegpunkt-Marker vom Vorrat auf den Wilde-Jagd-Vorratsstapel.

	\medskip

	Wählst du die zweite Option und der Anwesen-Vorratsstapel ist leer (d. h. du kannst dir kein Anwesen nehmen), darfst du dir die Siegpunkt-Marker vom Wilde-Jagd-Vorratsstapel nicht nehmen. Du darfst aber diese Option trotzdem wählen.

	\medskip

	Ist der Wilde-Jagd-Vorratsstapel leer, funktioniert das Ausspielen dieser Karte trotzdem in der beschriebenen Weise weiter. Nutzt die Platzhalterkarte, um den Vorratsstapel zu markieren.}
\end{tikzpicture}
\hspace{-0.6cm}
\begin{tikzpicture}
	\card
	\cardstrip
	\cardbanner{banner/gold.png}
	\cardicon{banner/coin.png}
	\cardprice{5}
	\cardtitle{Zauber}
	\cardcontent{Wenn du diese Karte ausspielst und dich für die zweite Option entscheidest, darfst du (musst aber nicht) sofort, wenn du die \emph{nächste} Karte in deinem Zug kaufst, eine Karte mit anderem Namen nehmen, die \emph{exakt so viel} kostet, wie die gekaufte Karte. Dann erst nimmst du die gekaufte Karte. Das kann wichtig bei Karten sein, die Anweisungen beim Nehmen einer Karte beinhalten.

	\medskip

	Spielst du mehrere Zauber in einem Zug, darfst du dir für die nächste gekaufte Karte mehrere Karten mit anderem Namen als die gekaufte Karte aber mit den gleichen Kosten nehmen. Die Karten, die du nimmst müssen zwar einen anderen Namen
	als die gekaufte Karte haben, dürfen aber untereinander alle den gleichen Namen haben.}
\end{tikzpicture}
\hspace{-0.6cm}
\begin{tikzpicture}
	\card
	\cardstrip
	\cardbanner{banner/white.png}
	\cardicon{banner/hex.png}
	\cardprice{\textcolor{white}{4}}
	\cardtitle{Ingenieurin}
	\cardcontent{Du darfst dir keine Karte nehmen, die mehr als 4 Geld kostet oder Schulden in den Kosten hat. Nimm die gewählte Karte.

	\medskip

	Dann darfst du diese Ingenieurin entsorgen. Wenn du das tust, nimm eine weitere Karte, die bis zu 4 Geld kostet. Dies kann die gleiche Karte wie die erste sein oder eine andere.}
\end{tikzpicture}
\hspace{-0.6cm}
\begin{tikzpicture}
	\card
	\cardstrip
	\cardbanner{banner/white.png}
	\cardicon{banner/hex.png}
	\cardprice{\textcolor{white}{8}}
	\cardtitle{\miniscule{Königlicher Schmied}}
	\cardcontent{Du musst, nachdem du 5 Karten nachgezogen hast, alle deine Handkarten vorzeigen und jedes Kupfer, das du zu diesem Zeitpunkt auf der Hand hast, ablegen.}
\end{tikzpicture}
\hspace{-0.6cm}
\begin{tikzpicture}
	\card
	\cardstrip
	\cardbanner{banner/white.png}
	\cardicon{banner/hex.png}
	\cardprice{\textcolor{white}{8}}
	\cardtitle{Lehnsherr}
	\cardcontent{Wähle eine Karte vom Vorrat, die zu diesem Zeitpunkt bis zu 5 Geld kostet, d. h. du darfst keine Karte eines leeren Stapels, eine nicht sichtbare Karte eines gemischten Stapels oder eine Karte eines Nicht-Vorratsstapels wählen.

	\medskip

	Behandle nun den ausgespielten Lehnsherr, wie die gewählte Karte (und nicht mehr als Lehnsherr) – bis sie nicht mehr im Spiel ist. Das heißt du befolgst alle Anweisungen der anderen Karte. Auch nimmt der Lehnsherr den Namen, die Kosten und den Typ der gewählten Karte an, bis er nicht mehr im Spiel ist. Als Dauerkarte bleibt dieser Lehnsherr ebenso im Spiel, wie er als Reservekarte (aus Abenteuer) zur Seite gelegt wird. Spielst du diesen Lehnsherr auf einen Thronsaal (aus dem Basisspiel), wählst du beim ersten Ausspielen die Karte, die dieser Lehnsherr ab sofort ist – beim zweiten Ausspielen ist er damit wieder genau diese Karte – du darfst keine andere Karte wählen. Erst mit dem Ausspielen des Lehnsherrn nimmt er Typ und Namen der gewählten Karte an – d. h. du darfst ihn nicht als Krone in deiner Kaufphase spielen, da er selbst keine Geldkarte ist und nicht in der Kaufphase ausgespielt werden darf.}
\end{tikzpicture}
\hspace{-0.6cm}
\begin{tikzpicture}
	\card
	\cardstrip
	\cardbanner{banner/white.png}
	\cardicon{banner/hex.png}
	\cardprice{\textcolor{white}{8}}
	\cardtitle{Stadtviertel}
	\cardcontent{\emph{Errata:} Der Kartentext sollte lauten: \enquote{+2 Aktionen}

	\medskip

	Du erhältst 2 zusätzliche Aktionen. Außerdem deckst du deine Handkarten auf und ziehst anschließend eine Karte von deinem Nachziehstapel für jede Aktionskarte, die du aufgedeckt hast.

	\medskip

	Kombinierte Karten, die den Kartentyp Aktion besitzen, wie z. B. die Krone, sind auch Aktionskarten.}
\end{tikzpicture}
\hspace{-0.6cm}
\begin{tikzpicture}
	\card
	\cardstrip
	\cardbanner{banner/white.png}
	\cardicon{banner/coin.png}
	\cardprice{8}
	\cardiconaddition{banner/hex.png}
	\cardpriceaddition{\textcolor{white}{8}}
	\cardtitle{\quad Reichtum}
	\cardcontent{Es werden nur alle Geld verdoppelt, die du vor dem Ausspielen dieser Karte ausgespielt hast und nur, wenn du in diesem Zug noch keinen Reichtum ausgespielt hast. Für jedes weitere Ausspielen eines Reichtums erhältst du nur +1 Kauf.}
\end{tikzpicture}
\hspace{-0.6cm}
\begin{tikzpicture}
	\card
	\cardstrip
	\cardbanner{banner/green.png}
	\cardtitle{Schlösser (1/2)\quad}
	\cardcontent{Der Schloss-Stapel ist ein gemischter Vorratsstapel. Alle Schlösser werden nach Kosten sortiert auf dem Vorratsstapel bereitgelegt (die teuerste zuunterst). Es darf immer nur die oberste Karte gekauft oder genommen werden.

	\medskip

	\emph{Bescheidenes Schloss:} Spielst du sie in deiner Kaufphase aus, ist sie 1 Geld wert. Bei Spielende erhältst du pro Karte, die den Typ Schloss beinhaltet (inklusive dieser Karte), einen Siegpunkt-Marker.

	\medskip

	\emph{Verfallendes Schloss:} Wenn du diese Karte während des Spiels nimmst, nimm dir einen Siegpunkt-Marker sowie ein Silber vom Vorrat. Wenn du diese Karte während des Spiels entsorgst, nimm dir einen weiteren Siegpunkt-Marker sowie ein Silber vom Vorrat. Bei Spielende ist diese Karte 1 Siegpunkt wert.

	\medskip

	\emph{Kleines Schloss:} Spielst du sie in deiner Aktionsphase aus, entsorge dieses Kleine Schloss oder eine andere Schloss-Karte aus deiner Hand. Wenn du das tust, nimm dir die Schloss-Karte vom Vorratsstapel, die zu diesem Zeitpunkt oben liegt. Dies kann eine teurere sein, als die, die du entsorgst. Du musst die Kosten nicht bezahlen. Bei Spielende ist diese Karte 2 Siegpunkte wert.}
\end{tikzpicture}
\hspace{-0.6cm}
\begin{tikzpicture}
	\card
	\cardstrip
	\cardbanner{banner/green.png}
	\cardtitle{Schlösser (2/2)\quad}
	\cardcontent{\emph{Spukschloss:} Wenn du diese Karte während deines Zuges nimmst (kaufst oder auf andere Art und Weise nimmst), nimm dir ein Gold vom Vorrat. Ist kein Gold mehr im Vorrat, erhältst du nichts. Außerdem (egal ob du ein Gold nehmen kannst oder nicht) müssen alle Mitspieler mit 5 oder mehr Handkarten 2 Handkarten auf ihren Nachziehstapel zurücklegen. Da diese Karte keine Angriffskarte ist, dürfen die Mitspieler keine Reaktionskarte spielen. Bei Spielende ist diese Karte 2 Siegpunkte wert.

	\smallskip

	\emph{Reiches Schloss:} Spielst du sie in deiner Aktionsphase aus, lege beliebig viele Punktekarten (auch ggf. kombinierte) aus deiner Hand ab. Pro abgelegter Karte erhältst du +2 Geld. Bei Spielende ist diese Karte 3 Siegpunkte wert.

	\smallskip

	\emph{Ausgedehntes Schloss:} Wenn du diese Karte kaufst oder auf andere Art und Weise nimmst, nimm ein Herzogtum oder drei Anwesen. Bei Spielende ist diese Karte 4 Siegpunkte wert.

	\smallskip

	\emph{Prunkschloss:} Wenn du diese Karte kaufst oder auf andere Art und Weise nimmst, zeige deine Handkarten vor. Nimm einen Siegpunkt-Marker vom Vorrat für jede Punktekarte (auch ggf. kombinierte), die du zu diesem Zeitpunkt auf der Hand oder im Spiel hast. Bei Spielende ist diese Karte 5 Siegpunkte wert.

	\smallskip

	\emph{Königsschloss:} Bei Spielende erhältst du pro Karte, die den Typ Schloss beinhaltet (inklusive dieser Karte) 2 Siegpunkte.}
\end{tikzpicture}
\hspace{-0.6cm}
\begin{tikzpicture}
	\card
	\cardstrip
	\cardbanner{banner/white.png}
	\cardtitle{Katapult/Felsen\qquad}
	\cardcontent{Spielvorbereitung: Legt auf diese Karte 5 Felsen und oben darauf 5 Katapulte.

	\bigskip

	Es darf immer nur die oberste Karte des Stapels genommen oder gekauft werden.}
\end{tikzpicture}
\hspace{-0.6cm}
\begin{tikzpicture}
	\card
	\cardstrip
	\cardbanner{banner/white.png}
	\cardtitle{\scriptsize{Gladiator/Reichtum}\qquad}
	\cardcontent{Spielvorbereitung: Legt auf diese Karte 5 Reichtum und oben darauf 5 Gladiatoren. 

	\bigskip

	Es darf immer nur die oberste Karte des Stapels genommen oder gekauft werden.}
\end{tikzpicture}
\hspace{-0.6cm}
\begin{tikzpicture}
	\card
	\cardstrip
	\cardbanner{banner/white.png}
	\cardtitle{\scriptsize{Siedler/Emsiges Dorf}\qquad}
	\cardcontent{Spielvorbereitung: Legt auf diese Karte 5 Emsige Dörfer und oben darauf 5 Siedler. 

	\bigskip

	Es darf immer nur die oberste Karte des Stapels genommen oder gekauft werden.}
\end{tikzpicture}
\hspace{-0.6cm}
\begin{tikzpicture}
	\card
	\cardstrip
	\cardbanner{banner/white.png}
	\cardtitle{\tiny{Patrizier/Handelsplatz}\qquad}
	\cardcontent{Spielvorbereitung: Legt auf diese Karte 5 Handelsplätze und oben darauf 5 Patrizier. 

	\bigskip

	Es darf immer nur die oberste Karte des Stapels genommen oder gekauft werden.}
\end{tikzpicture}
\hspace{-0.6cm}
\begin{tikzpicture}
	\card
	\cardstrip
	\cardbanner{banner/white.png}
	\cardtitle{\scriptsize{Feldlager/Diebesgut}\qquad}
	\cardcontent{Spielvorbereitung: Legt auf diese Karte 5 Diebesgut und oben darauf 5 Feldlager. 

	\bigskip

	Es darf immer nur die oberste Karte des Stapels genommen oder gekauft werden.}
\end{tikzpicture}
\hspace{-0.6cm}
\begin{tikzpicture}
	\card
	\cardstrip
	\cardbanner{banner/white.png}
	\cardtitle{Ereignisse (1/3)\quad}
	\cardcontent{\emph{Aufstieg:} Wenn du keine Aktionskarte entsorgst passiert nichts weiter.

	\medskip

	\emph{Erforschen:} Jeder Erwerb eines Erforschen gibt dir den Kauf zurück, den du für den Erwerb benötigt hast. Mit 7 Geld und 1 Kauf kannst du zum Beispiel 2 Erforschen erwerben und dann eine Karte kaufen oder ein Ereignis für 3 Geld erwerben.

	\medskip

	\emph{Steuer:} Auf jeden Vorratsstapel (d.h. alle Königreichkarten, Fluchkarten und Basiskarten, nicht Ereignisse und Landmarken) wird in der Spielvorbereitung 1 Schulden-Marker gelegt. Spieler, die eine Karte von einem Stapel kaufen, auf dem Schulden-Marker liegen, müssen alle Marker des Stapels nehmen. Nimmt ein Spieler eine Karte auf andere Art und Weise (d.h. er kauft sie nicht), werden eventuelle Schulden-Marker auf die nächste Karte des Vorratsstapels gelegt. Wenn du dieses Ereignis erwirbst, legst du 2 Schulden-Marker auf einen beliebigen Vorratsstapel – egal ob dort zu diesem Zeitpunkt bereits Schulden-Marker liegen oder nicht.

	\medskip

	\emph{Bankett:} Du kannst dieses Ereignis auch kaufen, wenn der Kupfer-Vorratsstapel aufgebraucht ist.}
\end{tikzpicture}
\hspace{-0.6cm}
\begin{tikzpicture}
	\card
	\cardstrip
	\cardbanner{banner/white.png}
	\cardtitle{Ereignisse (2/3)\quad}
	\cardcontent{\emph{Versalztes Land:} Wenn die entsorgte Karte eine Anweisung beinhaltet, die eintritt wenn diese Karte entsorgt wird, musst du diese Anweisung ausführen.

	\medskip

	\emph{Ritual:} Wenn du keinen Fluch nehmen kannst (z.B. weil der Vorratsstapel leer ist), passiert nichts. Es werden nur die Geld-Kosten gezählt – für Schulden-Kosten oder Trank-Kosten (aus Alchemisten) erhältst du nichts.

	\medskip

	\emph{Glücksfall:} Wenn weniger als 3 Gold im Vorrat sind, nimm dir die restlichen Gold.

	\medskip

	\emph{Eroberung:} Pro Silber, das du in diesem Zug genommen hast (inklusive der 2 Silber durch diese Karte), nimm dir einen Siegpunkt-Marker vom Vorrat. Dies ist kumulativ. Erwirbst du z.B. eine Eroberung und erhältst dafür 2 Siegpunkt-Marker (für die beiden Silber durch diese Karte) und dann noch eine Eroberung, für die du 2 Silber nehmen kannst, erhältst du für die zweite Eroberung schon 4 Siegpunkt-Marker. Sind nicht genügend Silber im Vorrat, nimmst du dir so viele wie möglich. Dann erhältst du aber auch entsprechend weniger Siegpunkt-Marker.

	\medskip

	\emph{Beherrschen:} Ist der Provinz-Vorratsstapel leer oder du kannst aus einem anderen Grund keine Provinz nehmen, hat dieses Ereignis keine Auswirkung.}
\end{tikzpicture}
\hspace{-0.6cm}
\begin{tikzpicture}
	\card
	\cardstrip
	\cardbanner{banner/white.png}
	\cardtitle{Ereignisse (3/3)\quad}
	\cardcontent{\emph{Hochzeit:} Den Siegpunkt-Marker nimmst du in jedem Fall – auch wenn der Gold-Vorratsstapel leer ist.

	\medskip

	\emph{Siegeszug:} Wenn du ein Anwesen nimmst, nimmst du für jede Karte, die du in diesem Zug bereits genommen hast (inklusive dem Anwesen jedoch nicht für Ereignisse), einen Siegpunkt-Marker. Wenn du kein Anwesen nehmen kannst (z.B. weil der Vorratsstapel leer ist), passiert nichts.

	\medskip

	\emph{Schlacht:} Du kannst dieses Ereignis auch erwerben, wenn der Herzogtum-Vorratsstapel leer ist. Die bis zu 5 ausgewählten Karten verbleiben in deinem Ablagestapel. Die restlichen Karten mischst du in deinen Nachziehstapel.

	\medskip

	\emph{Spende:} Befinden sich unter den entsorgten Karten welche, die Anweisungen beinhalten, die beim Entsorgen ausgeführt werden, musst du diese ausführen, bevor du die restlichen Karten mischst. Die Spende wird erst nach dem Zug, in dem sie erworben wird, ausgeführt (d.h. zwischen zwei Zügen). Damit hat zum Beispiel die Besessenheit (aus Alchemisten) auf diese Anweisung keine Auswirkung.}
\end{tikzpicture}
\hspace{-0.6cm}
\begin{tikzpicture}
	\card
	\cardstrip
	\cardbanner{banner/green.png}
	\cardtitle{\footnotesize{Landmarken (1/8)}\quad}
	\cardcontent{\emph{Aquädukt:} Wenn du eine Geldkarte von einem Vorratsstapel nimmst, auf dem ein oder mehrere Siegpunkt-Marker liegen (auch ggf. kombinierte Karten oder Kupfer, wenn dort durch Anweisungen auf Karten oder Ereignissen Siegpunkt-Marker platziert wurden), nimm einen Siegpunkt-Marker und lege ihn hierher auf das Aquädukt. Wenn du eine Punktekarte (auch ggf. kombinierte) nimmst, nimm dir alle Siegpunkt-Marker, die zu diesem Zeitpunkt hier auf dem Aquädukt liegen. Wenn du eine kombinierte Geld- und Punktekarte nimmst, kannst du dich entscheiden, in welcher Reihenfolge du die Anweisungen ausführst. 

	\emph{Errata:} Der Kartentext sollte lauten: \enquote{Wenn du ein\emph{e} Geld\emph{karte} nimmst, ...}

	\medskip

	\emph{Arena:} Beginnst du (z.B. durch die Villa) in deinem Zug mehrfach mit deiner Kaufphase, kannst du die Arena mehrfach nutzen.

	\medskip

	\emph{Badehaus:} Egal ob du eine Karte kaufst oder auf andere Art und Weise nimmst (bzw. nehmen musst), erhältst du in diesem Fall keine Siegpunkt-Marker vom Badehaus. Wer ein Ereignis erwirbt, nimmt damit keine Karte und kann, insofern keine andere Karte genommen wurde, 2 Siegpunkt- Marker von hier nehmen.}
\end{tikzpicture}
\hspace{-0.6cm}
\begin{tikzpicture}
	\card
	\cardstrip
	\cardbanner{banner/green.png}
	\cardtitle{\footnotesize{Landmarken (2/8)}\quad}
	\cardcontent{\emph{Basilika:} Für jede Karte die du kaufst, nimmst du 2 Siegpunkt-Marker von der Basilika, falls du zu diesem Zeitpunkt mindestens 2 Geld ausgespielt aber noch nicht verbraucht hast. Hast du beispielsweise 4 Geld und 3 Käufe, kannst du ein Kupfer kaufen (4 Geld übrig), dir 2 Siegpunkt-Marker nehmen, ein Anwesen kaufen (2 Geld übrig), dir 2 Siegpunkt-Marker nehmen und ein weiteres Anwesen kaufen (0 Geld übrig) – für den letzten Kauf erhältst du keine Siegpunkt-Marker.

	\medskip

	\emph{Bollwerk:} Hier werden alle Geldkarten (auch ggf. kombinierte) ausgewertet, die im Spiel benutzt wurden (auch ggf. Geldkarten, die im Schwarzmarkt (aus Basisspiel Special Edition bzw. Promokarte) enthalten waren). Haben zwei oder mehrere Spieler die gleiche höchste Anzahl einer Geldkarte, erhalten alle diese Spieler 5 Siegpunkte.

	\medskip

	\emph{Brunnen:} Du erhältst entweder 15 Siegpunkte oder 0 Siegpunkte. Es gibt keinen Extra-Bonus, wenn du mehr als 10 Kupfer besitzt.}
\end{tikzpicture}
\hspace{-0.6cm}
\begin{tikzpicture}
	\card
	\cardstrip
	\cardbanner{banner/green.png}
	\cardtitle{\footnotesize{Landmarken (3/8)}\quad}
	\cardcontent{\emph{Entweihter Schrein:} Immer wenn du eine beliebige Aktionskarte nimmst und auf dem entsprechenden Vorratsstapel ein oder mehrere Siegpunkt-Marker liegen (egal ob sie dort auf Grund der Anweisung auf dieser Landmarken-Karte oder einer anderen Karte, Ereignis oder Landmarken-Karte liegen), nimm einen Siegpunkt-Marker von dort und lege ihn hierher auf den Entweihten Schrein. Nur wenn du einen Fluch kaufst (nicht, wenn du ihn auf andere Art und Weise nimmst), nimmst du alle Siegpunkt-Marker, die zu diesem Zeitpunkt hier liegen. In der Spielvorbereitung legt ihr auf jeden Vorratsstapel, der den Typ Aktion, nicht aber den Typ Sammlung (also nicht auf die Karten Bauernmarkt, Tempel und Wilde Jagd) beinhaltet, 2 Siegpunkt-Marker.}
\end{tikzpicture}
\hspace{-0.6cm}
\begin{tikzpicture}
	\card
	\cardstrip
	\cardbanner{banner/green.png}
	\cardtitle{\footnotesize{Landmarken (4/8)}\quad}
	\cardcontent{\emph{Gebirgspass:} Diese Landmarken-Karte wird genau einmal pro Spiel ausgeführt – nämlich nach Beendigung des Zuges, in dem ein Spieler die erste Provinz aus dem Vorrat nimmt. Entsorgt vorher ein Spieler bereits eine Provinz (z.B. durch das Ereignis Versalztes Land), hat jener Spieler diese Provinz aber nicht vorher genommen und erfüllt deshalb diese Bedingung auch noch nicht. In einem Spiel, indem keine Provinz genommen wird, findet diese Landmarken-Karte keine Anwendung.

	\medskip

	Der Gebirgspass wird zwischen zwei Zügen ausgeführt und kann damit z.B. von der Besessenheit (aus Alchemisten) nicht beeinflusst werden. Der Mitspieler links von dem Spieler, der die erste Provinz genommen hat, beginnt mit einem Gebot oder passt. Ein Gebot besteht aus einer Anzahl Schulden zwischen 1 und 40. Der nächste Spieler muss mindestens 1 Schulden mehr bieten als der vorherige oder passen. Ein Gebot von 40 Schulden kann nicht überboten werden. Haben alle Spieler ein Gebot abgegeben oder gepasst, bzw. wurde bereits das Höchstgebot von 40 Schulden erreicht, erhält der Spieler mit dem höchsten Gebot die entsprechende Anzahl Schulden-Marker sowie 8 Siegpunkt-Marker. Passen alle Spieler, erhält keiner etwas.}
\end{tikzpicture}
\hspace{-0.6cm}
\begin{tikzpicture}
	\card
	\cardstrip
	\cardbanner{banner/green.png}
	\cardtitle{\footnotesize{Landmarken (5/8)}\quad}
	\cardcontent{\emph{Grabmal:} Dies funktioniert auch außerhalb deines Zuges (z.B. mit dem Trickser aus Intrige) oder wenn du eine Karte entsorgst, die nicht deine eigene ist (z.B. durch das Ereignis Versalztes Land).

	\medskip

	\emph{Kolonnaden:} Wenn du eine Aktionskarte kaufst (nicht, wenn du sie auf andere Art und Weise nimmst), musst du eine Karte mit dem gleichen Namen bereits im Spiel haben, um 2 Siegpunkt-Marker von hier zu erhalten. Karten eines Stapels haben nicht unbedingt alle den gleichen Namen (z.B. bei gemischten Stapeln).

	\medskip

	\emph{Labyrinth:} Dies kann nur einmal pro Zug eines Spielers eintreten, nämlich genau in dem Moment, in dem er die zweite Karte in seinem Zug nimmt. Nimmt er außerhalb seines Zuges zwei Karten, erhält er nichts.

	\medskip

	\emph{Mauer:} Hast du mehr als 15 Karten in deinem Kartensatz, bekommst für jede Karte darüber hinaus 1 Siegpunkt. Spieler, die zum Beispiel 27 Karten im Kartensatz haben, erhalten -12 Siegpunkte, Spieler mit 14 Karten im Kartensatz erhalten keinen Siegpunkt Abzug. Die Gesamtpunktzahl kann damit auch negativ sein.}
\end{tikzpicture}
\hspace{-0.6cm}
\begin{tikzpicture}
	\card
	\cardstrip
	\cardbanner{banner/green.png}
	\cardtitle{\footnotesize{Landmarken (6/8)}\quad}
	\cardcontent{\emph{Museum:} Auch Karten, die vom gleichen Stapel stammen, aber unterschiedliche Namen haben (z.B. gemischte Stapel), werden mit jeweils 2 Siegpunkten abgerechnet.

	\medskip

	\emph{Obelisk:} Es zählen alle Karten des gewählten Stapels, auch wenn sie unterschiedliche Namen haben (z.B. bei gemischten Stapeln). Zu Spielbeginn ermittelt ihr einen zufälligen Stapel, der den Typ Aktion beinhaltet (auch ggf. kombinierte Karten) und zum Vorrat gehört. Ruinen (aus Dark Ages) können bestimmt werden, ebenfalls der Stapel, der auch als Bannstapel für die Junge Hexe (aus Reiche Ernte) genutzt wird. Dazu zählen jedoch nicht die Eintauschkarten (aus Abenteuer) und Preiskarten (aus Reiche Ernte), da diese nicht zum Vorrat gehören.

	\medskip

	\emph{Obstgarten:} Du erhältst keinen zusätzlichen Bonus, wenn du zum Beispiel von einer Aktionskarte 6 Exemplare besitzt, d.h. du erhältst für eine Aktionskarte, von der du 3 Exemplare besitzt genauso 4 Siegpunkte wie für eine, von der du 7 Exemplare besitzt.}
\end{tikzpicture}
\hspace{-0.6cm}
\begin{tikzpicture}
	\card
	\cardstrip
	\cardbanner{banner/green.png}
	\cardtitle{\footnotesize{Landmarken (7/8)}\quad}
	\cardcontent{\emph{Palast:} Wenn du bei Spielende beispielsweise 7 Kupfer, 5 Silber und 2 Gold in deinem Kartensatz hast, erhältst du 6 Siegpunkte, da du zwei komplette Sätze aus je 1 Kupfer, Silber und Gold besitzt. Hättest du noch ein drittes Gold, würdest du 9 Siegpunkte erhalten.

	\medskip

	\emph{Räuberfestung:} Hast du bei Spielende zum Beispiel 3 Silber und 1 Gold in deinem Kartensatz, werden dir 8 Siegpunkte abgezogen. Die Gesamtpunktzahl kann damit auch negativ sein.

	\medskip

	\emph{Schlachtfeld:} Du erhältst 2 Siegpunkt-Marker von hier, egal ob du die Punktekarte (auch ggf. kombinierte) kaufst oder auf andere Art und Weise nimmst. Dies funktioniert auch außerhalb deines Zuges. Falls mehrere Spieler eine Punktekarte nehmen, wird dies in Spielerreihenfolge (beginnend bei dem Spieler links des aktuellen Spielers) getan.}
\end{tikzpicture}
\hspace{-0.6cm}
\begin{tikzpicture}
	\card
	\cardstrip
	\cardbanner{banner/green.png}
	\cardtitle{\footnotesize{Landmarken (8/8)}\quad}
	\cardcontent{\emph{Triumphbogen:} Wenn du bei Spielende beispielsweise 7 Villen und 4 Wilde Jagden (und keine andere (auch ggf. kombinierte) Aktionskarte häufiger) in deinem Kartensatz hast, erhältst du 12 Siegpunkte (d.h. 3 Siegpunkte für jede der 4 Wilde Jagden). Hast du neben 7 Villen auch 7 Wilde Jagden, erhältst du für beide zusammen 21 Siegpunkte. 
	\emph{Errata:} Es sollte lauten \enquote{bei Gleichstand zählt nur eine} statt \enquote{bei Gleichstand zählen beide}.

	\medskip

	\emph{Turm:} Der Vorratsstapel muss leer sein. Ein gemischter Stapel, bei dem nur eine Sorte Karten fehlt, zählt nicht. Die Vorratsstapel mit Punktekarten zählen ebenfalls nicht, ein leerer Fluch-Stapel aber schon.

	\medskip

	\emph{Wolfsbau:} Du bekommst keine Minuspunkte durch den Wolfsbau, wenn du von einer Karte gar keine bzw. zwei oder mehr Stück in deinem kompletten Kartensatz besitzt. Hast du zum Beispiel einen Fluch in deinem Nachziehstapel und einen in deinem Ablagestapel, hast du insgesamt zwei Flüche und erhältst keine Minuspunkte durch den Wolfsbau. Die Gesamtpunktzahl kann negativ sein.}
\end{tikzpicture}
\hspace{-0.6cm}
\begin{tikzpicture}
	\card
	\cardstrip
	\cardbanner{banner/white.png}
	\cardtitle{\scriptsize{Empfohlene 10er Sätze\qquad}}
	\cardcontent{\emph{Basis Einführung}\\
	Hochzeit (Ereignis), Turm (Landmarke), Bauernmarkt, Forum, Ingenieurin, Legionär, Patrizier/Handelsplatz, Opfer, Schlösser, Stadtviertel, Villa, Wagenrennen

	\smallskip

	\emph{Fortgeschrittene Einführung}\\
	Arena (Landmarke), Triumphbogen (Landmarke), Archiv, Gärtnerin, Gladiator/Reichtum, Katapult/Felsen, Königlicher Schmied, Krone, Siedler/Emsiges Dorf, Tempel, Vermögen, Zauberin

	\smallskip

	\emph{Alles in Maßen (mit Basisspiel)}\\
	Glücksfall (Ereignis), Obstgarten (Landmarke), Forum, Legionär, Lehnsherr, Tempel, Zauberin, Bibliothek, Dorf, Keller, Umbau, Werkstatt

	\smallskip

	\emph{Silberne Kugeln (mit Basisspiel)}\\
	Eroberung (Ereignis), Aquädukt (Landmarke), Bauernmarkt, Gärtnerin, Katapult/Felsen, Patrizier/Handelsplatz, Zauber, Bürokrat, Gärtner, Geldverleiher, Laboratorium, Markt

	\smallskip

	\emph{Köstliche Folter (mit Die Intrige)}\\
	Bankett (Ereignis), Arena (Landmarke), Gärtnerin, Krone, Opfer, Schlösser, Siedler/Emsiges Dorf, Baron, Brücke, Eisenhütte, Harem, Kerkermeister}
\end{tikzpicture}
\hspace{-0.6cm}
\begin{tikzpicture}
	\card
	\cardstrip
	\cardbanner{banner/white.png}
	\cardtitle{\scriptsize{Empfohlene 10er Sätze\qquad}}
	\cardcontent{\emph{Buddy-Prinzip (mit Die Intrige)}\\
	Versalztes Land (Ereignis), Wolfsbau (Landmarke), Archiv, Forum, Ingenieurin, Katapult/Felsen, Vermögen, Adlige, Bergwerk, Handelsposten, Handlanger, Maskerade

	\emph{Eingeengt (mit Seaside)}\\
	Steuer (Ereignis), Mauer (Landmarke), Feldlager/Diebesgut, Gladiator/Reichtum, Schlösser, Wagenrennen, Zauberin, Lagerhaus, Müllverwerter, Schmuggler, Taktiker, Werft

	\emph{König der Meere (mit Seaside)}\\
	Erforschen (Ereignis), Brunnen (Landmarke), Archiv, Bauernmarkt, Lehnsherr, Tempel, Wilde Jagd, Eingeborenendorf, Entdecker, Hafen, Piratenschi , Seehexe

	\emph{Sammler (mit Die Alchemisten)}\\
	Kolonnaden (Landmarke), Museum (Landmarke), Bauernmarkt, Feldlager/Diebesgut, Krone, Stadtviertel, Zauberin, Apotheker, Kräuterkundiger, Lehrling, Universität, Verwandlung

	\emph{Gewaltig (mit Blütezeit)}\\
	Beherrschen (Ereignis), Obelisk (Landmarke), Gladiator/Reichtum, Königlicher Schmied, Patrizier/Handelsplatz, Vermögen, Villa, Bank, Großer Markt, Königliches Siegel, Kunstschmiede, Lohn, Platin- und Kolonie-Karten}
\end{tikzpicture}
\hspace{-0.6cm}
\begin{tikzpicture}
	\card
	\cardstrip
	\cardbanner{banner/white.png}
	\cardtitle{\scriptsize{Empfohlene 10er Sätze\qquad}}
	\cardcontent{\emph{Vergoldete Pforten (mit Blütezeit)}\\
	Basilika (Landmarke), Palast (Landmarke), Feldlager/Diebesgut, Gärtnerin, Stadtviertel, Wagenrennen, Wilde Jagd, Bischof, Denkmal, Hausierer, Münzer, Talisman, Platin- und Kolonie-Karten

	\emph{Zoowärter (mit Reiche Ernte)}\\
	Schlacht (Ereignis), Kolonnaden (Landmarke), Lehnsherr, Opfer, Siedler/Emsiges Dorf, Villa, Wilde Jagd, Festplatz, Harlekin, Menagerie, Pferdehändler, Turnier
	
	\emph{Einfache Pläne (mit Hinterland)}\\
	Spende (Ereignis), Labyrinth (Landmarke), Forum, Katapult/Felsen, Patrizier/Handelsplatz, Tempel, Villa, Aufbau, Blutzoll, Feilscher, Grenzdorf, Stallungen
	
	\emph{Ausbreitung (mit Hinterland)}\\
	Brunnen (Landmarke), Schlachtfeld (Landmarke), Feldlager/Diebesgut, Ingenieurin, Legionär, Schlösser, Zauber, Fernstraße, Fruchtbares Land, Gewürzhändler, Schatztruhe, Tunnel
	
	\emph{Das Grabmal des Rattenkönigs (mit Dark Ages)}\\
	Aufstieg (Ereignis), Grabmal (Landmarke), Katapult/Felsen, Legionär, Schlösser, Stadtviertel, Wagenrennen, Festung, Lagerraum, Leichenkarren, Ratten, Raubzug, Unterschlupf-Karten}
\end{tikzpicture}
\hspace{-0.6cm}
\begin{tikzpicture}
	\card
	\cardstrip
	\cardbanner{banner/white.png}
	\cardtitle{\scriptsize{Empfohlene 10er Sätze\qquad}}
	\cardcontent{\emph{Der Triumph des Banditenkönigs (mit Dark Ages)}\\
	Siegeszug (Ereignis), Entweihter Schrein (Landmarke), Gärtnerin, Ingenieurin, Legionär, Vermögen, Zauber, Banditenlager, Jagdgründe, Katakomben, Marktplatz, Prozession
	
	\emph{Das Ritual des Knappen (mit Dark Ages)}\\
	Ritual (Ereignis), Museum (Landmarke), Archiv, Katapult/Felsen, Krone, Patrizier/Handelsplatz, Siedler/Emsiges Dorf, Eisenhändler, Eremit, Knappe, Lehen, Schurke
	
	\emph{Geldfluss (mit Die Gilden)}\\
	Badehaus (Landmarke), Gebirgspass (Landmarke), Gladiator/Reichtum, Ingenieurin, Königlicher Schmied, Schlösser, Stadtviertel, Arzt, Bäcker, Herold, Metzger, Wahrsager
	
	\emph{Kontrollbereich (mit Abenteuer)}\\
	Bankett (Ereignis), Bollwerk (Landmarke), Bauernmarkt, Katapult/Felsen, Krone, Vermögen, Zauber, Königliche Münzen, Page (+ Eintauschkarten), Relikt, Schatz, Weinhändler
	
	\emph{Kein Geld, keine Probleme (mit Abenteuer)}\\
	Mission (Ereignis), Räuberfestung (Landmarke), Archiv, Feldlager/Diebesgut, Königlicher Schmied, Tempel, Villa, Duplikat, Gefolgsmann, Kleinbauer (+ Eintauschkarten), Transformation, Verlies}
\end{tikzpicture}
\hspace{0.6cm}
