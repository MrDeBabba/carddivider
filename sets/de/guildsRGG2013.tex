% Basic settings for this card set
\renewcommand{\cardcolor}{guilds}
\renewcommand{\cardextension}{Erweiterung VII}
\renewcommand{\cardextensiontitle}{Die Gilden}

\clearpage
\newpage
\section{\cardextension \ - \cardextensiontitle (Rio Grande Games 2013)}

\begin{tikzpicture}
	\card
	\cardstrip
	\cardbanner{banner/white.png}
	\cardicon{banner/coin.png}
	\cardprice{2}
	\cardtitle{\footnotesize{Leuchtenmacher}}
	\cardcontent{Du darfst in der Aktionsphase eine weitere Aktionskarte ausspielen. Du darfst in der Kaufphase einen zusätzlichen Kauf tätigen. Nimm dir eine Münze.}
\end{tikzpicture}
\hspace{-1cm}
\begin{tikzpicture}
	\card
	\cardstrip
	\cardbanner{banner/white.png}
	\cardicon{banner/coin.png}
	\cardprice{2+}
	\cardtitle{Steinmetz}
	\cardcontent{\tiny{Wenn du die Karte \emph{STEINMETZ} kaufst, darfst du mehr dafür bezahlen. Wenn du das tust, nimmst du dir zwei Aktionskarten, die beide jeweils genau so viel kosten, wie du überzahlt hast. Du kannst dir zweimal die gleiche oder zwei verschiedene Karten nehmen. Wenn du zum Beispiel den \emph{STEINMETZ} für \coin{6} kaufst, könntest du dir zwei \emph{HEROLDE} nehmen. Die Aktionskarten stammen aus dem Vorrat und werden auf deinen Ablagestapel gelegt. Sollten keine Karten mit den entsprechenden Kosten im Vorrat sein, bekommst du keine. Wenn du mit einem \emph{TRANK} (\emph{Die Alchemisten}) überzahlst, bekommst du Karten im Wert des \emph{TRANKS}. Karten, die mehreren Kartentypen angehören, von denen einer AKTION ist (wie die \emph{GROSSE HALLE} aus \emph{Die Intrige}), sind Aktionskarten. Wenn du beschließt, nicht mehr als die normalen Kosten zu zahlen, bekommst du keine Karten. Es ist nicht möglich, dann Aktionskarten zu nehmen, die \coin{0}	kosten.
	\\
	\medskip
	\\
	Wenn du diese Karte spielst, entsorgst du eine Karte aus deiner Hand und nimmst dir zwei Karten, die jeweils weniger kosten als die Karte, die du entsorgt hast. Wenn du keine Karte zum Entsorgen mehr auf der Hand hast, bekommst du keine Karten. Du kannst dir zweimal die gleiche oder zwei verschiedene Karten aus dem Vorrat nehmen. Diese werden auf deinen Ablagestapel gelegt. Falls es keine preiswerteren Karten im Vorrat gibt (weil du beispielsweise ein Kupfer entsorgst), bekommst du keine Karten. Wenn es im Vorrat nur noch eine einzige Karte gibt, die preiswerter ist als die entsorgte Karte, nimmst du dir diese. Du nimmst dir die Karten einzeln nacheinander; das kann bei Karten eine Rolle spielen, die beim Nehmen eine Auswirkung haben (wie das \emph{GASTHAUS} aus \emph{Hinterland}).\\}}
\end{tikzpicture}
\hspace{-1cm}
\begin{tikzpicture}
	\card
	\cardstrip
	\cardbanner{banner/white.png}
	\cardicon{banner/coin.png}
	\cardprice{3+}
	\cardtitle{Arzt}
	\cardcontent{\tiny{Wenn du diese Karte kaufst, darfst du mehr dafür zahlen. Pro \coin{1}, das du zusätzlich zahlst (d.h. \enquote{überzahlst}), darfst du dir die oberste Karte deines Nachziehstapels ansehen und sie entsorgen, ablegen oder auf den Nachziehstapel zurücklegen. Wenn dein Nachziehstapel aufgebraucht ist, mischst du deinen Ablagestapel und machst ihn zum neuen Nachziehstapel. Wenn dann noch immer nicht genügend Karten im Nach- ziehstapel sind, siehst du dir keine Karten an. Auch wenn du mehr als \coin{1} überzahlst, wird jede oberste Karte vom Nachziehstapel einzeln komplett abgehandelt, bevor du die nächste Karte ziehst.
	\\
	\medskip
	\\
	Wenn du diese Karte in der Aktionsphase spielst, nennst du den Namen einer beliebigen Karte, deckst die obersten drei Karten deines Nachziehstapels auf und entsorgst jede Karte mit dem von dir genannten Namen. Die übrigen Karten legst du in beliebiger Reihenfolge wieder oben auf deinen Nachziehstapel. Du musst keine Karte nennen, die in diesem Spiel verwendet wird. Wenn dein Nachziehstapel keine drei Karten mehr umfasst, deckst du die darin noch enthaltenen Karten auf, mischst dann deinen Ablagestapel (der die bereits aufgedeckten Karten nicht umfasst) und machst ihn zum neuen Nachziehstapel, von dem du die noch fehlenden Karten aufdeckst. Sind dann noch immer nicht genügend Karten aufgedeckt, belässt du es bei den bereits aufgedeckten Karten.\\}}
\end{tikzpicture}
\hspace{-1cm}
\begin{tikzpicture}
	\card
	\cardstrip
	\cardbanner{banner/white.png}
	\cardicon{banner/coin.png}
	\cardprice{4}
	\cardtitle{Berater}
	\cardcontent{Wenn dein Nachziehstapel keine drei Karten mehr umfasst, deckst du die darin noch enthaltenen Karten auf, mischst dann deinen Ablagestapel und machst ihn zum neuen Nachziehstapel, von dem du die noch fehlenden Karten aufdeckst. Sind dann noch immer nicht genügend Karten aufgedeckt, belässt du es bei den bereits aufgedeckten Karten. Unabhängig davon, wie viele Karten du aufgedeckt hast, wählt dein linker Nachbar eine davon aus, die du ablegst. Die verbleibenden Karten nimmst du auf die Hand.}
\end{tikzpicture}
\hspace{-1cm}
\begin{tikzpicture}
	\card
	\cardstrip
	\cardbanner{banner/white.png}
	\cardicon{banner/coin.png}
	\cardprice{4}
	\cardtitle{Platz}
	\cardcontent{Zuerst ziehst du eine Karte. Du erhältst zwei weitere Aktionen, die du spielen darfst, nachdem du alle Anweisungen dieser Karte (soweit möglich) erfüllt hast. Dann darfst du eine Geldkarte (auch ein \emph{MEISTERSTÜCK}) ablegen. Du kannst die Karte ablegen, die du eben gezogen hast, falls es sich um eine Geldkarte handelt. Wenn du eine Geldkarte abgelegt hast, nimmst du dir eine Münze. Karten, die mehreren Kartentypen angehören, von denen einer \emph{GELD} ist (wie der \emph{HAREM} aus Die Intrige), sind Geldkarten.}
\end{tikzpicture}
\hspace{-1cm}
\begin{tikzpicture}
	\card
	\cardstrip
	\cardbanner{banner/white.png}
	\cardicon{banner/coin.png}
	\cardprice{4}
	\cardtitle{\scriptsize{Steuereintreiber}}
	\cardcontent{Du darfst eine Geldkarte aus deiner Hand entsorgen. Karten, die mehreren Kartentypen angehören, von denen einer GELD ist (wie der \emph{HAREM} aus \emph{Die Intrige}), sind Geldkarten. Wenn du eine Geldkarte entsorgst, muss jeder Mitspieler, der mindestens fünf Karten in der Hand hat, die gleiche Karte aus seiner Hand auf seinen Ablagestapel legen oder seine Handkarten vorzeigen, damit die anderen Spieler sehen, dass er diese Karte nicht auf der Hand hat. Nimm dir für die entsorgte Karte eine Geldkarte, die bis zu \coin{3} mehr kostet als die entsorgte Geldkarte, und lege sie oben auf deinen Nachziehstapel. Wenn dein Nachziehstapel aufgebraucht ist, wird dies die einzige Karte deines Nachziehstapels. Du musst dir keine teurere Geldkarte nehmen, sondern darfst dir auch eine gleich teure oder preiswertere Geldkarte nehmen.}
\end{tikzpicture}
\hspace{-1cm}
\begin{tikzpicture}
	\card
	\cardstrip
	\cardbanner{banner/white.png}
	\cardicon{banner/coin.png}
	\cardprice{4+}
	\cardtitle{Herold}
	\cardcontent{\tiny{Wenn du diese Karte kaufst, darfst du mehr dafür zahlen. Wenn du das tust legst du pro \coin{1}, das du überzahlst, eine beliebige Karte deines Ablagestapels oben auf deinen Nachziehstapel. Du darfst dir dafür die Karten deines Ablagestapels ansehen, was normalerweise nicht möglich ist. Allerdings darfst du nicht zuerst deinen Ablagestapel durchsehen, um zu entscheiden, wie viel du überzahlen willst. Sobald du überzahlt hast, musst du die entsprechende Anzahl an Karten in beliebiger Reihenfolge oben auf deinen Nachziehstapel legen, sofern möglich. Wenn du so viel überzahlst, dass du mehr Karten auf deinen Nachziehstapel legen müsstest, als sich in deinem Ablagestapel be nden, legst du einfach alle Karten deines Ablagestapels in beliebiger Reihenfolge auf deinen Nachziehstapel. Falls du den \emph{HEROLD} kaufst, ohne zu überzahlen, darfst du deinen Ablagestapel nicht durchsehen.
	\\
	\medskip
	\\
	Wenn du diese Karte in der Aktionsphase spielst, ziehst du zuerst eine Karte. Du erhältst dann eine weitere Aktion, die du spielen darfst, nachdem du alle Anweisungen dieser Karte (soweit möglich) erfüllt hast. Dann deckst du die oberste Karte deines Nachziehstapels auf. Wenn es sich um eine Aktionskarte handelt, \emph{musst} du sie spielen. Die Karte zu spielen, verbraucht keine Aktion. Karten, die mehreren Kartentypen angehören, von denen einer AKTION ist (z. B. \emph{GROSSE HALLE} aus \emph{Die Intrige}), sind Aktionskarten. Alle anderen Kartentypen werden auf den Nachziehstapel zurückgelegt, ohne sie zu spielen.
	\\
	\medskip
	\\
	\emph{Hinweis:} Wenn durch den \emph{HEROLD} eine Dauer-Karte (aus \emph{Seaside}) ausgespielt wird, wird der \emph{HEROLD} dennoch am Ende der Runde wie gewohnt abgelegt, da er nicht gebraucht wird, um an etwas zu erinnern.\\}}
\end{tikzpicture}
\hspace{-1cm}
\begin{tikzpicture}
	\card
	\cardstrip
	\cardbanner{banner/gold.png}
	\cardicon{banner/coin.png}
	\cardprice{3+}
	\cardtitle{Meisterstück}
	\cardcontent{Dies ist eine Geldkarte mit dem Wert \coin{1}, wie Kupfer. Wenn du sie kaufst, nimmst du dir ein Silber pro \coin{1}, das du überzahlst. Falls du zum Beispiel \coin{6} für das \emph{MEISTERSTÜCK} zahlst, erhältst du drei Silber. Das \emph{MEISTERSTÜCK} ist eine Geldkarte und wird regeltechnisch auch so behandelt.}
\end{tikzpicture}
\hspace{-1cm}
\begin{tikzpicture}
	\card
	\cardstrip
	\cardbanner{banner/white.png}
	\cardicon{banner/coin.png}
	\cardprice{5}
	\cardtitle{Bäcker}
	\cardcontent{Wenn du diese Karte spielst, ziehst du eine Karte, darfst eine weitere Aktionskarte ausspielen und nimmst dir eine Münze.
	\\
	\medskip
	\\
	Wird ein Spiel mit dieser Karte gespielt, erhält jeder Spieler zu Beginn des Spiels eine Münze. Das gilt auch für Partien mit dem \emph{SCHWARZMARKT}, bei denen der \emph{BÄCKER} sich im \emph{SCHWARZMARKT}-Stapel befindet.\\}
\end{tikzpicture}
\hspace{-1cm}
\begin{tikzpicture}
	\card
	\cardstrip
	\cardbanner{banner/white.png}
	\cardicon{banner/coin.png}
	\cardprice{5}
	\cardtitle{\footnotesize{Kaufmannsgilde}}
	\cardcontent{Wenn diese Karte im Spiel ist (d.h. du hast sie in der Aktionsphase gespielt), darfst du eine weitere Karte kaufen und hast zusätzlich \coin{1} zur Verfügung. Jedes Mal, wenn du eine Karte kaufst, nimmst du dir eine Münze (1 Münze, wenn du eine Karte kaufst; 2 Münzen, wenn du zwei Karten kaufst etc.). Denke daran, dass du Münzen nur einsetzen kannst, bevor du Karten kaufst: Du darfst also diese Münze nicht sofort aufwenden. Diese Anweisung ist kumulativ: Wenn du zwei \emph{KAUFMANNSGILDEN} ausgespielt hast, bringt dir jede Karte, die du kaufst, zwei Münzen ein. Wenn du jedoch eine \emph{KAUFMANNSGILDE} mehrfach spielst, aber nur eine im Spiel hast – wie beim \emph{THRONSAAL} (\emph{DOMINION®}) oder \emph{KÖNIGSHOF} (\emph{Blütezeit}) –, bekommst du beim Kauf einer Karte nur eine Münze.}
\end{tikzpicture}
\hspace{-1cm}
\begin{tikzpicture}
	\card
	\cardstrip
	\cardbanner{banner/white.png}
	\cardicon{banner/coin.png}
	\cardprice{5}
	\cardtitle{Metzger}
	\cardcontent{Zuerst nimmst du dir 2 Münzen. Dann darfst du eine Karte aus deiner Hand entsorgen und eine beliebige Anzahl an Münzen einsetzen (auch 0 Münzen). Da du den \emph{METZGER} nicht mehr auf der Hand hast, kannst du diese Karte nicht entsorgen. Allerdings kannst du eine andere \emph{METZGER}-Karte entsorgen. Wenn du eine Karte entsorgt hast, nimmst du dir eine Karte, deren Kosten höchstens der Summe aus den Kosten der entsorgten Karte und der Anzahl der eingesetzten Münzen entsprechen darf. So könntest du beispielsweise ein \emph{ANWESEN} entsorgen und sechs Münzen zahlen, um dir eine \emph{PROVINZ} zu nehmen; oder du könntest einen weiteren \emph{METZGER} entsorgen und null Münzen zahlen, um dir ein \emph{HERZOGTUM} zu nehmen. Die Münzen, die du einsetzt, werden in den Vorrat zurückgelegt und dürfen in der Kaufphase nicht mehr benutzt werden, um andere Karten zu kaufen.}
\end{tikzpicture}
\hspace{-1cm}
\begin{tikzpicture}
	\card
	\cardstrip
	\cardbanner{banner/white.png}
	\cardicon{banner/coin.png}
	\cardprice{5}
	\cardtitle{Hellseherin}
	\cardcontent{Das Gold und die Fluchkarten stammen aus dem Vorrat und werden auf den Ablagestapel gelegt. Wenn kein Gold mehr vorhanden ist, erhältst du keins. Sind nicht mehr genügend Fluchkarten da, teilst du sie reihum aus, beginnend mit deinem linken Nachbarn. Jeder Spieler, der eine Fluchkarte erhalten hat, muss eine Karte vom Nach- ziehstapel ziehen. Falls ein Spieler keine Fluchkarte bekommen hat – weil nicht genügend Fluchkarten da waren oder aus einem anderen Grund –, zieht er keine Karte. Nutzt ein Spieler den \emph{WACHTURM} (\emph{Blütezeit}), um den Fluch zu entsorgen, hat er dennoch eine Fluchkarte erhalten und zieht deshalb eine Karte. Nutzt ein Spieler den \emph{FAHRENDEN HÄNDLER} (\emph{Hinterland}), um stattdessen ein Silber zu nehmen, hat er keine Fluchkarte erhalten und zieht infolgedessen auch keine Karte.}
\end{tikzpicture}
\hspace{-1cm}
\begin{tikzpicture}
	\card
	\cardstrip
	\cardbanner{banner/white.png}
	\cardicon{banner/coin.png}
	\cardprice{5}
	\cardtitle{\footnotesize{Wandergeselle}}
	\cardcontent{Zunächst nennst du eine Karte. Dabei muss es sich nicht um den Namen einer Karte handeln, die in diesem Spiel verwendet wird. Dann deckst du solange Karten vom Nachziehstapel auf, bis du drei Karten aufgedeckt hast, die \emph{nicht} dem von dir genannten Namen entsprechen. Nimm diese Karten auf die Hand und lege die anderen ab. Sollte dabei der Nachziehstapel aufgebraucht werden, bevor du auf drei entsprechende Karten gestoßen bist, mischst du deinen Ablagestapel und nutzt ihn als neuen Nachziehstapel. Wenn du beim weiteren Aufdecken noch immer nicht auf drei Karten mit einem anderen als dem genannten Namen kommst, beendest du das Aufdecken. Nimm die gefundenen Karten anderen Namens auf die Hand und lege den Rest auf deinen Ablagestapel.}
\end{tikzpicture}
\hspace{-1cm}
\begin{tikzpicture}
	\card
	\cardstrip
	\cardbanner{banner/white.png}
	\cardicon{}
	\cardprice{}
	\cardtitle{\scriptsize{Empfohlene 10er Sätze\qquad}}
	\cardcontent{\emph{Kunsthandwerk} (Die Gilden + \textit{Basisspiel}:)
	\\
	Steinmetz, Berater, Bäcker, Wandergeselle, Kaufmannsgilde, \textit{Laboratorium}, \textit{Keller}, \textit{Werkstatt}, \textit{Jahrmarkt}, \textit{Geldverleiher}
	\\
	\smallskip
	\\
	\emph{Rechtschaffen und anständig} (Die Gilden + \textit{Basisspiel}:)
	\\
	Metzger, Bäcker, Leuchtenmacher, Arzt, Wahrsager, \textit{Miliz}, \textit{Dieb}, \textit{Geldverleiher}, \textit{Gärten}, \textit{Dorf}
	\\
	\smallskip
	\\
	\emph{Des Guten zuviel} (Die Gilden + \textit{Basisspiel}:)
	\\
	Platz, Meisterstück, Leuchtenmacher, Steuereintreiber, Herold, \textit{Bibliothek}, \textit{Umbau}, \textit{Abenteurer}, \textit{Markt}, \textit{Kanzler}
	\\
	\smallskip
	\\
	\emph{Nenne diese Karte} (Die Gilden + \textit{Die Intrige}:)
	\\
	Bäcker, Arzt, Platz, Berater, Meisterstück, \textit{Burghof}, \textit{Wunschbrunnen}, \textit{Harem}, \textit{Tribut}, \textit{Adelige}
	\\
	\smallskip
	\\
	\emph{Geschäftstricks} (Die Gilden + \textit{Die Intrige}:)
	\\
	Steinmetz, Herold, Wahrsager, Wandergeselle, Metzger, \textit{Große Halle}, \textit{Adelige}, \textit{Verschwörer}, \textit{Maskerade}, \textit{Kupferschmied}
	\\
	\smallskip
	\\
	\emph{Entscheidungen, Entscheidungen} (Die Gilden + \textit{Die Intrige}:)
	\\
	Kaufmannsgilde, Leuchtenmacher, Meisterstück, Steuereintreiber, Metzger, \textit{Brücke}, \textit{Handlanger}, \textit{Bergwerk}, \textit{Anbau}, \textit{Herzog}
	\\
}
\end{tikzpicture}
\hspace{-1cm}
\begin{tikzpicture}
	\card
	\cardstrip
	\cardbanner{banner/white.png}
	\cardtitle{Neue Regeln\qquad}
	\cardcontent{\tiny{\emph{Die Münzen:} Einige Karten in \emph{Die Gilden} erlauben den Spielern, sich Münzen zu nehmen. Münzen werden immer vom Vorrat genommen, nicht von anderen Spielern. Diese Münzen können – anders als die Geldkarten – über den Zug, in dem ein Spieler sie erhält, hinaus aufbewahrt werden. In der Kaufphase eines Spielers kann dieser, \emph{bevor} er Karten kauft, eine beliebige Anzahl an Münzen einsetzen; jede eingesetzte Münze bringt dem Spieler +\coin{1}. Eingesetzte Münzen werden in den Vorrat zurückgelegt. Die Anzahl der Münzen im Vorrat ist nicht begrenzt: Sollten einmal nicht ausreichend Münzen vorhanden sein, verwenden die Spieler einen beliebigen Ersatz. Zwar handelt es sich um die gleichen Marker/Münzen wie in \emph{Seaside} und \emph{Blütezeit}, allerdings können die durch Anweisungen auf den Karten aus \emph{Die Gilden} erworbenen Münzen ausschließlich in der Kaufphase ausgegeben werden. Sie dürfen weder zum Kauf einer Karte mithilfe der Karte \emph{SCHWARZMARKT} benutzt werden, noch dürfen sie auf ein Piratenschiff-Tableau (\emph{Seaside}) oder die \emph{HANDELSROUTE} (\emph{Blütezeit}) gelegt werden.
	\\
	\emph{Überzahlen:} Für einige Karten in \emph{Die Gilden} darf man mehr zahlen, als sie eigentlich kosten. Bei diesen Karten steht hinter den Kosten ein \enquote{+}, z.B. \coin{2\textsuperscript{+}}. Wenn ein Spieler vor dem Kauf einen \emph{beliebigen zusätzlichen Betrag} für eine solche Karte zahlt, tritt ein auf der jeweiligen Karte genannter Effekt ein, der von der Höhe der Überzahlung abhängt. Überzahlt werden kann mit allen Geldwerten, mit denen Karten gekauft werden: Geldkarten, Münzen und Geldwerte auf Aktionskarten. Die Karten \emph{TRANK} (\emph{Die Alchemisten}) können ebenfalls zum Überzahlen verwendet werden (allerdings ist das nicht immer sinnvoll). Das zum Überzahlen verwendete Geld ist ausgegeben und kann dann nicht mehr eingesetzt werden. Geldkarten werden abgelegt und Münzen kommen zurück in den Vorrat. Spieler können eine Karte ausschließlich beim Kauf überzahlen, nicht wenn sie diese auf eine andere Weise erhalten. Das \enquote{+} ist lediglich eine Erinnerung: Für eine Karte, die hinter den Kosten das Symbol \enquote{+} aufweist, gelten nach wie vor für alle Zwecke die normalen Kosten.
	Grundsätzlich gibt es keine Wechselwirkungen zwischen Karten, welche die Kartenkosten reduzieren – wie die \emph{BRÜCKE} (\emph{Die Intrige}) oder die \emph{FERNSTRASSE} (\emph{Hinterland}) – und dem \enquote{Überzahle}.
	\\
	\emph{Zwei Dinge passieren zur gleichen Zeit:} Wenn einem Spieler zwei Dinge gleichzeitig passieren, entscheidet er selbst, in welcher Reihenfolge sie eintreten. Wenn unterschiedlichen Spielern zwei Dinge gleichzeitig passieren, werden diese reihum in Spielreihenfolge abgehandelt, beginnend mit dem Spieler, der gerade an der Reihe ist.\\}}
\end{tikzpicture}
\hspace{1cm}