% Basic settings for this card set
\renewcommand{\cardcolor}{intrigue}
\renewcommand{\cardextension}{Edition II}
\renewcommand{\cardextensiontitle}{Die Intrige}

\clearpage
\newpage
\section{\cardextension \ - \cardextensiontitle}

\begin{tikzpicture}
	\card
	\cardstrip
	\cardbanner{banner/whitegreen.png}
	\cardicon{banner/coin.png}
	\cardprice{6}
	\cardtitle{Adelige}
	\cardcontent{Diese Karte ist \emph{zugleich} eine Aktions- und eine Punktekarte. Wenn du sie ausspielst, kannst du wählen, 3 Karten nachzuziehen \emph{oder} 2 zusätzliche Aktionen zu erhalten. Die beiden Anweisungen können jedoch nicht geteilt und gemischt werden. Bei Spielende sind die Adeligen 2 Punkte wert. Im Spiel zu 3. und zu 4. werden 12 Karten verwendet, im Spiel zu 2. werden 8 Karten verwendet.}
\end{tikzpicture}
\hspace{-1cm}
\begin{tikzpicture}
	\card
	\cardstrip
	\cardbanner{banner/white.png}
	\cardicon{banner/coin.png}
	\cardprice{5}
	\cardtitle{Anbau}
	\cardcontent{Du ziehst zuerst eine Karte. Danach \emph{musst} du eine Karte aus deiner Hand entsorgen und dann eine Karte nehmen, die genau 1 Geld mehr kostet als die entsorgte Karte. Ist keine solche Karte im Vorrat, erhältst du keine Karte, musst jedoch trotzdem eine entsorgen. Wenn du keine Karte zum Entsorgen hast, entsorgst du keine und nimmst dir keine Karte.}
\end{tikzpicture}
\hspace{-1cm}
\begin{tikzpicture}
	\card
	\cardstrip
	\cardbanner{banner/white.png}
	\cardicon{banner/coin.png}
	\cardprice{3}
	\cardtitle{Armenviertel}
	\cardcontent{Du erhältst 2 zusätzliche Aktionen. Dann \emph{musst} du deine Kartenhand vorzeigen. Wenn du keine Aktionskarten auf der Hand hast (auch kombinierte Aktions-/Punktekarten sind Aktionskarten), musst du 2 Karten nachziehen. Sollte die erste der gezogenen Karten eine Aktionskarte sein, ziehst du trotzdem eine zweite Karte.}
\end{tikzpicture}
\hspace{-1cm}
\begin{tikzpicture}
	\card
	\cardstrip
	\cardbanner{banner/white.png}
	\cardicon{banner/coin.png}
	\cardprice{5}
	\cardtitle{Austausch}
	\cardcontent{Entsorge zuerst eine Karte aus deiner Hand. Dann nimmst du dir eine Karte vom Vorrat, die maximal 2 mehr kostet als die entsorgte Karte. Eine Karte kostet nur dann maximal 2 mehr, wenn die restlichen Kosten (z. B. TRANK aus Empires oder SCHULDEN aus Alchemie) gleich oder niedriger sind. Wenn die genommene Karte eine Aktions- und/oder Geldkarte ist, legst du die Karte oben auf deinen Nachziehstapel. Ansonsten legst du die Karte auf den Ablagestapel. Ist die genommene Karte eine Punktekarte, nimmt sich jeder Mitspieler – beginnend bei deinem linken Mitspieler – einen Fluch. Ist die genommene Karte eine Punktekarte sowie eine Aktions- oder Geldkarte (z. B. MÜHLE), legst du die Karte oben auf deinen Nachziehstapel und jeder Mitspieler muss sich einen Fluch nehmen.}
\end{tikzpicture}
\hspace{-1cm}
\begin{tikzpicture}
	\card
	\cardstrip
	\cardbanner{banner/white.png}
	\cardicon{banner/coin.png}
	\cardprice{4}
	\cardtitle{Baron}
	\cardcontent{Du musst kein Anwesen ablegen, auch wenn du eines auf der Hand hast. Wenn du jedoch keines ablegst, musst du dir ein Anwesen nehmen, so lange noch welche im Vorrat sind. Du kannst nicht nur den +1 Kauf nutzen und die übrigen Anweisungen ignorieren.}
\end{tikzpicture}
\hspace{-1cm}
\begin{tikzpicture}
	\card
	\cardstrip
	\cardbanner{banner/white.png}
	\cardicon{banner/coin.png}
	\cardprice{4}
	\cardtitle{Bergwerk}
	\cardcontent{Du ziehst immer eine Karte nach und erhältst 2 zusätzliche Aktionen. Dann musst du entscheiden, ob du das Bergwerk entsorgst, bevor du weitere Aktionen ausspielst oder in die anderen Phasen übergehst. Wenn du das Bergwerk auf einen Thronsaal spielst, kannst du die Karte nur einmal entsorgen (d. h., du erhältst insgesamt +2 Karten, +4 Aktionen aber nur +2 Geld ).}
\end{tikzpicture}
\hspace{-1cm}
\begin{tikzpicture}
	\card
	\cardstrip
	\cardbanner{banner/white.png}
	\cardicon{banner/coin.png}
	\cardprice{4}
	\cardtitle{Brücke}
	\cardcontent{Die Kosten sind für alle Belange um 1 Geld reduziert. Wenn du z. B. ein Bergwerk ausspielst, danach eine Brücke, dann eine Eisenhütte, könntest du dir für die Eisen- hütte ein Herzogtum nehmen (kostet durch die Brücke nur noch 4 Geld). Die Karten der Spieler (Handkarten, Nachziehstapel und Ablagestapel) sind auch betroffen. Der Effekt ist kumulativ. Wenn du die Brücke auf einen Thronsaal spielst, sind die Kosten der Karten für diesen Zug um 2 Geld reduziert. Die Kosten sinken niemals unter 0 Geld. Wenn du eine Brücke und dann einen Anbau ausspielst, kannst du ein Kupfer entsorgen (das immer noch 0 Geld kostet) und dir einen Handlanger dafür nehmen (kostet durch die Brücke nur noch 1 Geld).}
\end{tikzpicture}
\hspace{-1cm}
\begin{tikzpicture}
	\card
	\cardstrip
	\cardbanner{banner/white.png}
	\cardicon{banner/coin.png}
	\cardprice{2}
	\cardtitle{Burghof}
	\cardcontent{Du ziehst 3 Karten und nimmst diese auf die Hand bevor du eine Karte auf den Nachziehstapel legst. Die Karte, die du auf den Nachziehstapel legst, muss keine der 3 gerade gezogenen Karten sein.}
\end{tikzpicture}
\hspace{-1cm}
\begin{tikzpicture}
	\card
	\cardstrip
	\cardbanner{banner/blue.png}
	\cardicon{banner/coin.png}
	\cardprice{4}
	\cardtitle{Diplomatin}
	\cardcontent{Diese Karte ist eine Aktions- und Reaktionskarte. Wird sie als Aktion in der Aktionsphase ausgespielt, nimmst du 2 Karten. Hast du dann 5 oder weniger Karten auf der Hand, erhältst du außerdem + 2 Aktionen. Spielt ein Mitspieler eine Angriffskarte aus und du hast zu diesem Zeitpunkt 5 oder mehr Karten auf der Hand, darfst du diese Karte – bevor der ausgespielte Angriff ausgeführt wird – aus der Hand aufdecken. Wenn du das tust, nimmst du diese DIPLOMATIN wieder auf die Hand, ziehst 2 Karten und legst dann 3 Karten (auch möglich inklusive dieser DIPLOMATIN) ab. Hast du dann immer noch 5 oder mehr Karten sowie eine DIPLOMATIN auf der Hand, darfst du die DIPLOMATIN noch einmal aufdecken – und dies so oft wiederholen wie du möchtest und die Bedingung der 5 oder mehr Karten auf der Hand erfüllt ist. Erst dann wird der Angriff ausgeführt. Hast du mehrere Reaktionskarten auf der Hand, mit denen du auf das Ausspielen einer Angriffskarte reagieren kannst, darfst du diese nacheinander in beliebiger Reihenfolge aufdecken.}
\end{tikzpicture}
\hspace{-1cm}
\begin{tikzpicture}
	\card
	\cardstrip
	\cardbanner{banner/white.png}
	\cardicon{banner/coin.png}
	\cardprice{4}
	\cardtitle{Eisenhütte}
	\cardcontent{Du nimmst dir eine Karte vom Vorrat und legst sie auf deinen Ablagestapel. Je nach Kartentyp der genommenen Karte erhältst du einen Bonus. Nimmst du eine Karte mit kombiniertem Kartentyp, z. B. Große Halle erhältst du +1 Aktion (weil die Große Halle eine Aktionskarte ist) und +1 Karte (weil die Große Halle auch eine Punktekarte ist).}
\end{tikzpicture}
\hspace{-1cm}
\begin{tikzpicture}
	\card
	\cardstrip
	\cardbanner{banner/white.png}
	\cardicon{banner/coin.png}
	\cardprice{4}
	\cardtitle{Geheimgang}
	\cardcontent{Du ziehst 2 Karten und erhältst + 1 Aktion. Dann nimmst du eine beliebige Karte aus deiner Hand (auch ggf. eine, die du gerade gezogen hast) und legst sie an eine beliebige Stelle in deinen Nachziehstapel. Du darfst sie oben drauf, unten drunter oder irgendwo in die Mitte legen. Du darfst dabei die Karten deines Nachziehstapels zählen, aber nicht ansehen. Befinden sich keine Karten in deinem Nachziehstapel, wird die zurückgelegte Karte zur einzigen Karte in deinem Nachziehstapel.}
\end{tikzpicture}
\hspace{-1cm}
\begin{tikzpicture}
	\card
	\cardstrip
	\cardbanner{banner/white.png}
	\cardicon{banner/coin.png}
	\cardprice{5}
	\cardtitle{Handelsposten}
	\cardcontent{Wenn du 2 oder mehr Karten auf der Hand hast, \emph{musst} du 2 Karten entsorgen und dir dafür ein Silber nehmen. Du nimmst das Silber direkt auf die Hand und kannst es auch in der Kaufphase verwenden. Wenn kein Silber mehr im Vorrat ist, erhältst du kein Silber, musst jedoch trotzdem 2 Karten entsorgen. Wenn du nur 1 Karte auf der Hand hast, musst du diese entsorgen, erhältst jedoch kein Silber. Wenn du keine Karte mehr auf der Hand hast, kannst du nichts entsorgen und erhältst auch kein Silber.}
\end{tikzpicture}
\hspace{-1cm}
\begin{tikzpicture}
	\card
	\cardstrip
	\cardbanner{banner/white.png}
	\cardicon{banner/coin.png}
	\cardprice{2}
	\cardtitle{Handlanger}
	\cardcontent{Wähle 2 verschiedene Anweisungen. Du darfst nicht eine Anweisung zweimal wählen. Du musst zuerst beide Anweisungen auswählen und sie dann erst (in jeder möglichen Reihenfolge) ausführen. Du kannst nicht eine Karte nachziehen und dann erst die zweite Anweisung wählen.}
\end{tikzpicture}
\hspace{-1cm}
\begin{tikzpicture}
	\card
	\cardstrip
	\cardbanner{banner/goldgreen.png}
	\cardicon{banner/coin.png}
	\cardprice{6}
	\cardtitle{Harem}
	\cardcontent{Diese Karte ist zugleich eine Geld- und eine Punktekarte. Du kannst sie in der Kaufphase spielen, genau wie ein Silber. Bei Spielende ist der Harem 2 Punkte wert. Im Spiel zu 3. und zu 4. werden 12 Karten verwendet, im Spiel zu 2. werden 8 Karten verwendet.}
\end{tikzpicture}
\hspace{-1cm}
\begin{tikzpicture}
	\card
	\cardstrip
	\cardbanner{banner/white.png}
	\cardicon{banner/coin.png}
	\cardprice{2}
	\cardtitle{Herumtreiberin}
	\cardcontent{Die Karte, die du entsorgst oder vom Müllstapel nimmst, muss den Typ AKTION beinhalten, d. h. sie kann auch eine kombinierte Aktionskarte (z. B. MÜHLE) sein. Genommene Karten werden auf den Ablagestapel gelegt – es sei denn, auf der Karte steht etwas anderes. Wird eine Karte entsorgt, die einen speziellen Effekt beim Entsorgen hat, tritt dieser ein.}
\end{tikzpicture}
\hspace{-1cm}
\begin{tikzpicture}
	\card
	\cardstrip
	\cardbanner{banner/green.png}
	\cardicon{banner/coin.png}
	\cardprice{5}
	\cardtitle{Herzog}
	\cardcontent{Diese Karte hat bis zum Ende des Spiels keine Funktion. Bei Spielende ist der Herzog 1 Punkt pro Herzogtum in Handkarten, Nachziehstapel und Ablagestapel wert. Im Spiel zu 3. und zu 4. werden 12 Karten verwendet, im Spiel zu 2. werden 8 Karten verwendet.}
\end{tikzpicture}
\hspace{-1cm}
\begin{tikzpicture}
	\card
	\cardstrip
	\cardbanner{banner/white.png}
	\cardicon{banner/coin.png}
	\cardprice{5}
	\cardtitle{Höflinge}
	\cardcontent{Decke eine Karte aus deiner Hand auf. Zähle dann die Typen, denen diese Karte angehört – also AKTION, GELD, REAKTION, ANGRIFF, PUNKTE, FLUCH etc. Pro Typ, dem die Karte angehört, entscheidest du dich für eine der vier angegebenen Optionen. Dabei darfst du keine der Optionen doppelt auswählen. Wenn du zum Beispiel eine PATROUILLE (AKTION) aufdeckst, darfst du eine Option auswählen, deckst du einen KARAWANENWÄCHTER aus Abenteuer (AKTION – DAUER – REAKTION) auf, darfst du 3 unterschiedliche Optionen wählen. Entscheidest du dich für das Gold, legst du dieses auf den Ablagestapel. Kannst du keine Handkarte aufdecken, erhältst du nichts.}
\end{tikzpicture}
\hspace{-1cm}
\begin{tikzpicture}
	\card
	\cardstrip
	\cardbanner{banner/white.png}
	\cardicon{banner/coin.png}
	\cardprice{5}
	\cardtitle{Kerkermeister}
	\cardcontent{Jeder Mitspieler, beginnend mit dem Spieler links vom Angreifer, muss sich eine der beiden Anweisungen wählen und diese dann ausführen. Ein Spieler kann wählen, 2 Karten abzulegen, auch wenn er weniger als 2 Karten auf der Hand hat. Hat er nur eine Karte auf der Hand legt er diese ab. Hat er keine Karte mehr auf der Hand, muss er auch keine ablegen. Ein Spieler kann wählen einen Fluch zu nehmen, auch wenn keine Fluchkarten mehr im Vorrat sind. In diesem Fall nimmt er keinen Fluch. Fluchkarten nehmen die Spieler direkt auf die Hand.}
\end{tikzpicture}
\hspace{-1cm}
\begin{tikzpicture}
	\card
	\cardstrip
	\cardbanner{banner/white.png}
	\cardicon{banner/coin.png}
	\cardprice{5}
	\cardtitle{Lakai}
	\cardcontent{Zunächst entscheidest du dich für eine der der beiden Anweisungen. Entweder erhältst du +2 virtuelles Geld \emph{oder} du wählst die zweite Anweisung für den Angriff. In diesem Fall sind nur Spieler mit 5 oder mehr Karten auf der Hand betroffen. Wehrt ein Spieler den Angriff mit einem Burggraben ab, darf er weder Karten nachziehen, noch muss er Karten ablegen. Ein Spieler kann auf den Angriff mit der Geheimkammer reagieren, auch wenn er weniger als 5 Karten auf der Hand hat. Anschließend hast du +1 Aktion, unabhängig davon, welche der beiden Anweisungen du gewählt hast.}
\end{tikzpicture}
\hspace{-1cm}
\begin{tikzpicture}
	\card
	\cardstrip
	\cardbanner{banner/white.png}
	\cardicon{banner/coin.png}
	\cardprice{3}
	\cardtitle{Maskerade}
	\cardcontent{Du ziehst zuerst 2 Karten. Dann wählen alle Spieler gleichzeitig eine Karte aus ihrer Hand und legen diese verdeckt zwischen sich und den Spieler zu ihrer Linken. Erst dann nehmen alle Spieler die Karten, die sie vom Spieler rechts bekommen haben auf. Die Spieler wählen also zuerst, welche Karte sie weiter geben, bevor sie sehen, welche Karte sie bekommen. Am Ende darfst nur du eine Karte aus deiner Hand entsorgen. Die Maskerade ist kein Angriff. Die übrigen Spieler dürfen also keine Reaktionskarten aus ihrer Hand vorzeigen um sich zu schützen.}
\end{tikzpicture}
\hspace{-1cm}
\begin{tikzpicture}
	\card
	\cardstrip
	\cardbanner{banner/whitegreen.png}
	\cardicon{banner/coin.png}
	\cardprice{4}
	\cardtitle{Mühle}
	\cardcontent{Diese Karte ist eine kombinierte Aktions- und Punktekarte. Als Punktekarte bringt sie beim Zählen der Punkte 1. Spielst du die MÜHLE als Aktionskarte aus, ziehst du 1 Karte und erhältst + 1 Aktion. Du darfst 2 Karten aus deiner Hand ablegen. Wenn du das tust, erhältst du + 2 . Tust du das nicht (weil du zum Beispiel nicht genügend Karten auf der Hand hast), erhältst du nichts. Nur, wenn du nicht mehr als eine Karte auf der Hand hast, darfst du genau eine Karte ablegen, erhältst dafür aber kein GELD.}
\end{tikzpicture}
\hspace{-1cm}
\begin{tikzpicture}
	\card
	\cardstrip
	\cardbanner{banner/white.png}
	\cardicon{banner/coin.png}
	\cardprice{5}
	\cardtitle{Patrouille}
	\cardcontent{Ziehe zuerst 3 Karten. Decke dann die obersten 4 Karten deines Nachziehstapels auf. So aufgedeckte Punktekarten (auch ggf. kombinierte) und Flüche nimmst du alle auf die Hand. Die restlichen Karten legst du in beliebiger Reihenfolge zurück auf den Nachziehstapel.}
\end{tikzpicture}
\hspace{-1cm}
\begin{tikzpicture}
	\card
	\cardstrip
	\cardbanner{banner/white.png}
	\cardicon{banner/coin.png}
	\cardprice{3}
	\cardtitle{Trickser}
	\cardcontent{Jeder Mitspieler, beginnend mit dem Spieler links vom Angreifer, muss die oberste Karte von seinem Nachziehstapel aufdecken. Er muss diese Karte entsorgen und du wählst eine Karte aus dem Vorrat, die das gleiche kostet. Diese Karte nimmt sich der Spieler und legt sie bei sich ab. Ist im Vorrat keine Karte, mit den gleichen Kosten, erhält der Spieler nichts, muss jedoch die Karte trotzdem entsorgen. Entsorgt er z. B. ein Kupfer, kannst du einen Fluch auswählen, den er nehmen muss. Du kannst auch die selbe Karte wählen, die er entsorgt hat. Die gewählte Karte muss im Vorrat zur Verfügung stehen. Du kannst also keine Karte aus einem leeren Stapel oder vom Müll wählen. Sind keine Karten mehr im Vorrat, die genau so viel kosten, wie die entsorgte Karte, erhält der Spieler nichts. Deckt ein Spieler den Burggraben aus seiner Hand auf, muss er keine Karte vom Nachziehstapel aufdecken und entsorgen und erhält auch keine Karte.}
\end{tikzpicture}
\hspace{-1cm}
\begin{tikzpicture}
	\card
	\cardstrip
	\cardbanner{banner/white.png}
	\cardicon{banner/coin.png}
	\cardprice{4}
	\cardtitle{Verschwörer}
	\cardcontent{Du überprüfst die Bedingung ob du +1 Karte und +1 Aktion erhältst wenn du den Verschwörer ausgespielt hast. Wenn die Bedingung später im Zug erfüllt wird, überprüfst du die Bedingung nicht rückwirkend. Wird eine Karte auf den Thronsaal gespielt, zählt der Thronsaal selbst als gespielte Aktionskarte und die darauf gespielte Aktionskarte zusätzlich zweimal als gespielte Aktionskarte. Wenn du z. B. den Verschwörer auf den Thronsaal spielst, ist der Thronsaal die erste Aktionskarte, der zuerst ausgespielte Verschwörer ist die zweite Aktionskarte (du erhältst also keine +1 Karte und keine +1 Aktion). Wenn du den Verschwörer zum zweiten mal ausspielst, hast du 3 Aktionskarten ausgespielt und erhältst +1 Karte und +1 Aktion.}
\end{tikzpicture}
\hspace{-1cm}
\begin{tikzpicture}
	\card
	\cardstrip
	\cardbanner{banner/white.png}
	\cardicon{banner/coin.png}
	\cardprice{3}
	\cardtitle{Verwalter}
	\cardcontent{Wenn du dich entscheidest, 2 Karten zu entsorgen und 2 oder mehr Karten auf der Hand hast, musst du genau 2 Karten entsorgen. Wenn du dich entscheidest, 2 Karten zu entsorgen, aber aber nur 1 Karte auf der Hand hast musst du diese Karte entsorgen. Du kannst die verschiedenen Anweisungen nicht mischen, du musst wählen: \emph{entweder} +2 Karten \emph{oder} +2 Geld \emph{oder} 2 Karten entsorgen.}
\end{tikzpicture}
\hspace{-1cm}
\begin{tikzpicture}
	\card
	\cardstrip
	\cardbanner{banner/white.png}
	\cardicon{banner/coin.png}
	\cardprice{3}
	\cardtitle{Wunschbrunnen}
	\cardcontent{Du ziehst zuerst eine Karte nach. Dann benennst du eine Karte (z. B. „Kupfer“, nicht „Geld“) und deckst die oberste Karte von deinem Nachziehstapel auf. Wenn es sich um die benannte Karte handelt, nimmst du sie auf die Hand. Wenn nicht, legst du sie zurück auf den Nachziehstapel.}
\end{tikzpicture}
\hspace{-1cm}
\begin{tikzpicture}
	\card
	\cardstrip
	\cardbanner{banner/white.png}
	\cardtitle{\scriptsize{Empfohlene 10er Sätze\qquad}}
	\cardcontent{\emph{Siegestanz:}
		\\
		Adlige, Austausch, Baron, Eisenhütte, Harem, Herzog, Höflinge, Maskerade, Mühle, Patrouille
		\\
		\smallskip
		\\
		\emph{Verschwörung:}
		\\
		Bergwerk, Eisenhütte, Geheimgang, Handelsposten, Handlanger, Herumtreiberin, Kerkermeister, Trickser, Verschwörer, Verwalter 
		\\
		\smallskip
		\\
		\emph{Beste Wünsche:}
		\\
		Anbau, Armenviertel, Baron, Burghof, Diplomatin, Geheimgang, Herzog, Kerkermeister, Verschwörer, Wunschbrunnen
		\\
	}
\end{tikzpicture}
\hspace{-1cm}