% Basic settings for this card set
\renewcommand{\cardcolor}{intrigue}
\renewcommand{\cardextension}{Erweiterung}
\renewcommand{\cardextensiontitle}{Die Intrige}

\clearpage
\newpage
\section{\cardextension \ - \cardextensiontitle \ (Rio Grande Games 2014)}

\begin{tikzpicture}
	\card
	\cardstrip
	\cardbanner{banner/white.png}
	\cardicon{banner/coin.png}
	\cardprice{2}
	\cardtitle{Burghof}
	\cardcontent{Du ziehst 3 Karten von deinem Nachziehstapel und nimmst sie auf die Hand. Dann wählst du eine beliebige Karte aus deiner Hand und legst sie verdeckt auf den Nachziehstapel.}
\end{tikzpicture}
\hspace{-0.6cm}
\begin{tikzpicture}
	\card
	\cardstrip
	\cardbanner{banner/blue.png}
	\cardicon{banner/coin.png}
	\cardprice{2}
	\cardtitle{\footnotesize{Geheimkammer}}
	\cardcontent{Wenn du diese Karte in deiner eigenen Aktionsphase ausspielst, legst du eine beliebige Anzahl Karten (auch 0 Karten) aus deiner Hand ab. \emph{Danach} erhältst du +\coin{1} pro abgelegte Karte. 

	\medskip

	Spielt ein anderer Spieler eine Angriffskarte aus, darfst du diese Karte vorzeigen, wenn du sie in diesem Moment auf der Hand hast, auch wenn dich der Angriff nicht betrifft. Wenn du das tust, ziehst du zuerst 2 Karten nach und legst dann 2 beliebige Karten aus deiner Hand verdeckt zurück auf den Nachziehstapel. Du darfst auch die \emph{GEHEIMKAMMER} selbst auf den Nachziehstapel legen, da sich die Karte auch nach dem Vorzeigen auf deiner Hand befindet. Du darfst pro Angriff so viele Reaktionskarten vorzeigen, wie du möchtest. So kannst du zum Beispiel zuerst die \emph{GEHEIMKAMMER} abwickeln und danach einen \emph{BURGGRABEN} vorzeigen, um den Angriff abzuwehren. Die \emph{GEHEIMKAMMER} selbst wehrt einen Angriff nicht ab.}
\end{tikzpicture}
\hspace{-0.6cm}
\begin{tikzpicture}
	\card
	\cardstrip
	\cardbanner{banner/white.png}
	\cardicon{banner/coin.png}
	\cardprice{2}
	\cardtitle{Handlanger}
	\cardcontent{Du darfst von den vier Anweisungen der Karte \emph{genau 2} auswählen und diese für deinen Zug nutzen. Du musst zwei verschiedene Anweisungen wählen und darfst nicht z. B. zwei Karten ziehen oder 2 zusätzliche Käufe tätigen. Du musst dich sofort entscheiden, welche zwei Anweisungen du nutzen möchtest. Du darfst nicht eine Karte nachziehen und dich erst dann entscheiden, welche zweite Anweisung du ausführen möchtest.}
\end{tikzpicture}
\hspace{-0.6cm}
\begin{tikzpicture}
	\card
	\cardstrip
	\cardbanner{banner/white.png}
	\cardicon{banner/coin.png}
	\cardprice{3}
	\cardtitle{Armenviertel}
	\cardcontent{Du darfst in deiner Aktionsphase zwei zusätzliche Aktionen ausführen. Zeige deine Kartenhand vor. Wenn du keine Aktionskarte (oder Aktions-/Punktekarte) auf der Hand hast, ziehst du zwei Karten nach. Sollten sich darunter Aktionskarten befinden, darfst du diese auch gleich nutzen.}
\end{tikzpicture}
\hspace{-0.6cm}
\begin{tikzpicture}
	\card
	\cardstrip
	\cardbanner{banner/whitegreen.png}
	\cardicon{banner/coin.png}
	\cardprice{3}
	\cardtitle{Große Halle}
	\cardcontent{Diese Karte ist eine kombinierte Aktions- und Punktekarte. Sie kann in der Aktionsphase eingesetzt werden und bringt \emph{zusätzlich} bei Spielende 1 Siegpunkt. Wenn du diese Karte ausspielst, ziehst du eine Karte und darfst eine zusätzliche Aktion ausspielen.}
\end{tikzpicture}
\hspace{-0.6cm}
\begin{tikzpicture}
	\card
	\cardstrip
	\cardbanner{banner/white.png}
	\cardicon{banner/coin.png}
	\cardprice{3}
	\cardtitle{Maskerade}
	\cardcontent{Ziehe zwei Karten. Anschließend wählen alle Spieler (einschließlich dir selbst) eine beliebige Karte aus ihrer Hand und legen sie verdeckt neben ihrem linken Nachbarn ab. Wenn alle Karten verteilt sind, nimmt jeder Spieler die erhaltene Karte auf die Hand. Da die \emph{MASKERADE} keine Angriffskarte ist, dürfen die anderen Spieler keine Reaktionskarten vorzeigen. Danach darfst du eine Karte aus deiner Hand entsorgen.}
\end{tikzpicture}
\hspace{-0.6cm}
\begin{tikzpicture}
	\card
	\cardstrip
	\cardbanner{banner/white.png}
	\cardicon{banner/coin.png}
	\cardprice{3}
	\cardtitle{Trickser}
	\cardcontent{Alle Mitspieler müssen die oberste Karte ihres Nachziehstapels aufdecken und diese entsorgen. Du wählst für jeden Mitspieler jeweils eine Karte aus dem Vorrat, die genauso viel kostet, wie die entsorgte Karte und gibst sie dem Mitspieler. Dieser legt die neue Karte auf seinen Ablagestapel. Befindet sich keine Karte mit den gleichen Kosten im Vorrat, muss der Mitspieler seine Karte trotzdem entsorgen, erhält dafür aber keine neue Karte. Du darfst dem Mitspieler auch eine gleiche Karte wie die, die er entsorgt hat, zurückgeben. }
\end{tikzpicture}
\hspace{-0.6cm}
\begin{tikzpicture}
	\card
	\cardstrip
	\cardbanner{banner/white.png}
	\cardicon{banner/coin.png}
	\cardprice{3}
	\cardtitle{Verwalter}
	\cardcontent{Du wählst von den drei Anweisungen der Karte \emph{genau 1} aus und führst diese dann komplett aus. Wenn du dich entscheidest, zwei Karten zu entsorgen, du aber nur eine Karte auf der Hand hast, musst du diese entsorgen.}
\end{tikzpicture}
\hspace{-0.6cm}
\begin{tikzpicture}
	\card
	\cardstrip
	\cardbanner{banner/white.png}
	\cardicon{banner/coin.png}
	\cardprice{3}
	\cardtitle{\scriptsize{Wunschbrunnen}}
	\cardcontent{Zuerst ziehst du eine Karte. Du darfst in der Aktionsphase eine zusätzliche Aktion ausführen. Nenne eine Karte (z.B. \emph{KUPFER}) und decke die oberste Karte deines Nachziehstapels auf. Handelt es sich dabei um die von dir genannte Karte, nimmst du sie auf die Hand. Ansonsten legst du sie zurück auf den Nachziehstapel.}
\end{tikzpicture}
\hspace{-0.6cm}
\begin{tikzpicture}
	\card
	\cardstrip
	\cardbanner{banner/white.png}
	\cardicon{banner/coin.png}
	\cardprice{4}
	\cardtitle{Baron}
	\cardcontent{Du \emph{darfst} ein Anwesen (sofern du gerade eins auf der Hand hast) ablegen und erhältst dafür +\coin{4} für die Kaufphase. Wenn du das nicht tun kannst (weil du kein Anwesen auf der Hand hast) oder willst, musst du dir ein Anwesen nehmen, solange noch welche im Vorrat sind.}
\end{tikzpicture}
\hspace{-0.6cm}
\begin{tikzpicture}
	\card
	\cardstrip
	\cardbanner{banner/white.png}
	\cardicon{banner/coin.png}
	\cardprice{4}
	\cardtitle{Bergwerk}
	\cardcontent{Du ziehst zuerst eine Karte nach und \emph{darfst} dann diese Karte entsorgen, bevor du ggf. weitere Aktionen ausspielst. Du erhältst dafür +\coin{2}. Wenn du das \emph{BERGWERK} auf einen \emph{THRONSAAL} spielst, erhältst du den Bonus für das Entsorgen nur einmal, da du die Karte nur einmal entsorgen kannst. Die anderen Anweisungen (+ 1 Karte sowie + 2 Aktionen) werden durch den \emph{THRONSAAL} dagegen verdoppelt.}
\end{tikzpicture}
\hspace{-0.6cm}
\begin{tikzpicture}
	\card
	\cardstrip
	\cardbanner{banner/white.png}
	\cardicon{banner/coin.png}
	\cardprice{4}
	\cardtitle{Brücke}
	\cardcontent{Die Kosten aller Karten (auch Handkarten, Karten aus den Nachzieh- und Ablagestapeln) werden in diesem Spielzug für alle Belange um \coin{1} reduziert (nicht aber unter \coin{0}). Dieser Effekt ist kumulativ, d. h. die Kosten pro Karte können durch das Ausspielen bestimmter Karten (z. B. den \emph{THRONSAAL}) auch um \coin{2} oder mehr reduziert werden.}
\end{tikzpicture}
\hspace{-0.6cm}
\begin{tikzpicture}
	\card
	\cardstrip
	\cardbanner{banner/white.png}
	\cardicon{banner/coin.png}
	\cardprice{4}
	\cardtitle{Eisenhütte}
	\cardcontent{Du nimmst dir eine beliebige Karte vom Vorrat, die maximal \coin{4} kostet. Durch das Ausspielen bestimmter Aktionskarten (z. B. die \emph{BRÜCKE}) können die Kosten der Karten reduziert werden.

	\medskip

	Je nachdem, ob du dich für eine Aktions-, Geld- oder Punktekarte entschieden hast, erhältst du einen anderen Bonus. Solltest du dich für eine kombinierte Karte entscheiden, erhältst du die Boni beider Kartentypen.}
\end{tikzpicture}
\hspace{-0.6cm}
\begin{tikzpicture}
	\card
	\cardstrip
	\cardbanner{banner/white.png}
	\cardicon{banner/coin.png}
	\cardprice{4}
	\cardtitle{\footnotesize{Kupferschmied}}
	\cardcontent{Mit dieser Karte erhöhst du den Wert aller in diesem Zug gespielten \emph{KUPFER} um +\coin{1}. Der Effekt ist kumulativ, d. h. durch das Ausspielen anderer Aktionskarten (z. B. den \emph{THRONSAAL} oder einen weiteren \emph{KUPFERSCHMIED}) kann der Wert weiter erhöht werden.}
\end{tikzpicture}
\hspace{-0.6cm}
\begin{tikzpicture}
	\card
	\cardstrip
	\cardbanner{banner/white.png}
	\cardicon{banner/coin.png}
	\cardprice{4}
	\cardtitle{Späher}
	\cardcontent{Sollten für das Aufdecken der vier Karten nicht genügend Karten im Nachziehstapel sein, ziehst du zunächst die restlichen Karten und mischst dann deinen Ablagestapel neu, ohne die bereits aufgedeckten Karten mit einzumischen. Sollten auch dann nicht genügend Karten zur Verfügung stehen, ziehst du nur so viele Karten wie möglich. Du musst alle Punktekarten auf die Hand nehmen, die restlichen Karten legst du in beliebiger Reihenfolge auf den Nachziehstapel. Diese musst du deinen Mitspielern nicht zeigen. Kombinierte Aktions-/Punktekarten sind auch Punktekarten. }
\end{tikzpicture}
\hspace{-0.6cm}
\begin{tikzpicture}
	\card
	\cardstrip
	\cardbanner{banner/white.png}
	\cardicon{banner/coin.png}
	\cardprice{4}
	\cardtitle{Verschwörer}
	\cardcontent{Wenn du zu dem Zeitpunkt an dem du den \emph{VERSCHWÖRER} spielst, bereits mindestens 3 Aktionskarten (inklusive diesem \emph{VERSCHWÖRER}) ausgespielt hast, erhältst du den Bonus. Wenn du erst im weiteren Verlauf deiner Aktionsphase diese Bedingung erfüllst, erhältst du den Bonus nicht. Aktionskarten, die z. B. durch den \emph{THRONSAAL} doppelt ausgespielt werden dürfen, gelten als 2 ausgespielte Karten.}
\end{tikzpicture}
\hspace{-0.6cm}
\begin{tikzpicture}
	\card
	\cardstrip
	\cardbanner{banner/white.png}
	\cardicon{banner/coin.png}
	\cardprice{5}
	\cardtitle{Anbau}
	\cardcontent{Nachdem du dir eine Karte genommen und eine zusätzliche Aktion erhalten hast, musst du eine Karte aus deiner Hand entsorgen sofern du noch Handkarten hast. Du nimmst dir dafür eine Karte vom Vorrat, die \emph{genau} \coin{1} mehr kostet als die entsorgte Karte. Wenn keine solche Karte vorhanden ist, musst du zwar eine Karte entsorgen, erhältst aber keine Karte vom Vorrat. }
\end{tikzpicture}
\hspace{-0.6cm}
\begin{tikzpicture}
	\card
	\cardstrip
	\cardbanner{banner/white.png}
	\cardicon{banner/coin.png}
	\cardprice{5}
	\cardtitle{\footnotesize{Handelsposten}}
	\cardcontent{Wenn du nur eine Karte auf der Hand hast, musst du sie entsorgen, erhältst dafür aber kein Silber. Wenn du zwei oder mehr Karten auf der Hand hast, musst du genau zwei Karten entsorgen und nimmst dir dafür ein Silber direkt auf die Hand. Sollte kein Silber mehr im Vorrat sein, musst du die Karten trotzdem entsorgen, erhältst aber kein Silber. }
\end{tikzpicture}
\hspace{-0.6cm}
\begin{tikzpicture}
	\card
	\cardstrip
	\cardbanner{banner/green.png}
	\cardicon{banner/coin.png}
	\cardprice{5}
	\cardtitle{Herzog}
	\cardcontent{Diese Karte ist die einzige reine Punktekarte unter den Königreichkarten. Sie hat bis zum Ende des Spiels keine Funktion. Bei Spielende erhält der Spieler, der diese Karte in seinem Kartensatz (Nachziehstapel, Handkarten und Ablagestapel) hat, für jedes \emph{HERZOGTUM} im Kartensatz 1 Siegpunkt. Wer mehrere \emph{HERZÖGE} besitzt, erhält für jeden \emph{HERZOG} die entsprechende Anzahl Siegpunkte.} 
\end{tikzpicture}
\hspace{-0.6cm}
\begin{tikzpicture}
	\card
	\cardstrip
	\cardbanner{banner/white.png}
	\cardicon{banner/coin.png}
	\cardprice{5}
	\cardtitle{\footnotesize{Kerkermeister}}
	\cardcontent{Jeder Mitspieler (beginnend bei deinem linken Nachbarn) muss entweder zwei Karten ablegen oder einen \emph{FLUCH} vom Stapel nehmen. Ein Spieler kann sich entscheiden, die Karten abzulegen, auch wenn er nur eine oder gar keine Karte auf der Hand hat. Er legt dann nur so viele Karten ab, wie er kann. Er kann sich auch entscheiden, einen \emph{FLUCH} zu nehmen, wenn es keine \emph{FLÜCHE} mehr im Vorrat gibt.}
\end{tikzpicture}
\hspace{-0.6cm}
\begin{tikzpicture}
	\card
	\cardstrip
	\cardbanner{banner/white.png}
	\cardicon{banner/coin.png}
	\cardprice{5}
	\cardtitle{Lakai}
	\cardcontent{Du entscheidest dich für eine der beiden Optionen: Entweder erhältst du +\coin{2} in diesem Zug oder du legst alle deine Handkarten ab und ziehst vier neue Karten nach. Wenn du die zweite Option wählst, müssen außerdem alle Mitspieler, die fünf oder mehr Karten auf der Hand haben (alle anderen sind nicht betroffen), diese ablegen und ebenfalls vier Karten nachziehen. Jeder Spieler (auch wenn er von dem Angriff nicht betroffen ist) kann eine oder mehrere Reaktionskarten vorzeigen, wenn du den \emph{LAKAIEN} spielst.}
\end{tikzpicture}
\hspace{-0.6cm}
\begin{tikzpicture}
	\card
	\cardstrip
	\cardbanner{banner/white.png}
	\cardicon{banner/coin.png}
	\cardprice{5}
	\cardtitle{Saboteur}
	\cardcontent{Jeder Mitspieler (beginnend bei deinem linken Nachbarn) muss solange Karten von seinem Nachziehstapel aufdecken, bis er eine Karte aufdeckt, die mindestens \emph{3} kostet (\emph{Achtung:} Die Kosten einer Karte können durch die \emph{BRÜCKE} verringert werden). Er muss diese Karte sofort entsorgen und darf sich dafür eine Karte aus dem Vorrat nehmen, die mindestens \emph{2} weniger kostet. Die restlichen aufgedeckten Karten werden abgelegt. Sollte im gesamten restlichen Nachziehstapel keine Karte mit passendem Wert vorhanden sein, wird der Ablagestapel ohne die bereits aufgedeckten Karten gemischt und zum neuen Nachziehstapel. Sollte dann immer noch keine passende Karte zu finden sein, legt der Spieler alle Karten ab und nichts passiert.}
\end{tikzpicture}
\hspace{-0.6cm}
\begin{tikzpicture}
	\card
	\cardstrip
	\cardbanner{banner/white.png}
	\cardicon{banner/coin.png}
	\cardprice{5}
	\cardtitle{Tribut}
	\cardcontent{Dein linker Mitspieler muss die obersten beiden Karten seines Nachziehstapels aufdecken und ablegen. Da der \emph{TRIBUT} keine Angriffskarte ist, kann sich der Mitspieler nicht gegen diese Anweisung wehren. Der Spieler, der ein \emph{TRIBUT} ausgespielt hat, erhält für die erste Karte den genannten Bonus. Für die zweite Karte erhält der Spieler nur dann ein weiteres Mal den Bonus, wenn nicht die gleiche Karte wie zuvor aufgedeckt wurde. Kombinierte Karten bringen dem Spieler auch doppelte Boni.}
\end{tikzpicture}
\hspace{-0.6cm}
\begin{tikzpicture}
	\card
	\cardstrip
	\cardbanner{banner/whitegreen.png}
	\cardicon{banner/coin.png}
	\cardprice{6}
	\cardtitle{Adelige}
	\cardcontent{Diese Karte ist eine kombinierte Aktions- und Punktekarte. Sie kann in der Aktionsphase eingesetzt werden und bringt außerdem bei Spielende 2 Siegpunkte. Wenn du diese Karte ausspielst, musst du dich entscheiden, ob du entweder 3 Karten ziehst oder  2 weitere Aktionen ausspielen willst. Du darfst die Anweisungen aber nicht mischen oder aufteilen. Wenn du die Karte das nächste Mal auf der Hand hast und ausspielst, darfst du natürlich eine andere Wahl treffen.}
\end{tikzpicture}
\hspace{-0.6cm}
\begin{tikzpicture}
	\card
	\cardstrip
	\cardbanner{banner/goldgreen.png}
	\cardicon{banner/coin.png}
	\cardprice{6}
	\cardtitle{Harem}
	\cardcontent{Diese Karte ist eine kombinierte Geld- und Punktekarte. Sie wird während des Zugs wie eine normale Geldkarte eingesetzt und bringt außerdem bei Spielende 2 Siegpunkte.}
\end{tikzpicture}
\hspace{-0.6cm}
\begin{tikzpicture}
	\card
	\cardstrip
	\cardbanner{banner/white.png}
	\cardtitle{\scriptsize{Empfohlene 10er Sätze\qquad}}
	\cardcontent{\emph{Siegestanz:}\\
	Adlige, Anbau, Brücke, Eisenhütte, Große Halle, Handlanger, Harem, Herzog, Maskerade, Späher 

	\smallskip

	\emph{Geheime Pläne:}\\
	Armenviertel, Eisenhütte, Handelsposten, Handlanger, Harem, Saboteur, Tribut, Trickser, Verschwörer, Verwalter

	\smallskip

	\emph{Beste Wünsche:}\\
	Anbau, Armenviertel, Burghof, Handelsposten, Kerkermeister, Kupferschmied, Maskerade, Späher, Verwalter, Wunschbrunnen

	\smallskip

	\emph{Demontage} (Intrige + \textit{Basisspiel}):\\
	Bergwerk, Brücke, Geheimkammer, Kerkermeister, Trickser, Saboteur, \textit{Dieb}, \textit{Spion}, \textit{Thronsaal}, \textit{Umbau}

	\smallskip

	\emph{Eine Hand voll} (Intrige + \textit{Basisspiel}):\\
	Adlige, Burghof, Kerkermeister, Lakai, Verwalter, \textit{Bürokrat}, \textit{Kanzler}, \textit{Miliz}, \textit{Mine}, \textit{Ratsversammlung}

	\smallskip

	\emph{Untergebene} (Intrige + \textit{Basisspiel}):\\
	Adlige, Baron, Lakai, Maskerade, Verwalter, Handlanger, \textit{Bibliothek}, \textit{Hexe}, \textit{Jahrmarkt}, \textit{Keller}}
\end{tikzpicture}
\hspace{-0.6cm}
