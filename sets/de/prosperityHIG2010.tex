% Basic settings for this card set
\renewcommand{\cardcolor}{prosperity}
\renewcommand{\cardextension}{Erweiterung III}
\renewcommand{\cardextensiontitle}{Blütezeit}
\renewcommand{\seticon}{prosperity.png}

\clearpage
\newpage
\section{\cardextension \ - \cardextensiontitle \ (Hans Im Glück 2010)}

\begin{tikzpicture}
	\card
	\cardstrip
	\cardbanner{banner/gold.png}
	\cardicon{icons/coin.png}
	\cardprice{5}
	\cardtitle{Abenteuer}
	\cardcontent{\emph{Errata:} Die Reihenfolge der beiden letzten Anweisungen sollte lauten: \enquote{Lege die übrigen aufgedeckten Karten ab. Lege diese Geldkarte aus.} 

	\smallskip

	Diese Geldkarte hat den Wert \coin[1], wie ein Kupfer. Wenn du diese Karte auslegst, deckst du solange Karten von deinem Nachziehstapel auf, bis du eine Geldkarte aufgedeckt hast. Die auf diese Weise aufgedeckte Geldkarte legst du sofort aus. Die übrigen gerade aufgedeckten Karten legst du auf deinen Ablagestapel. Wenn du auch nach dem Mischen keine Geldkarte aufdecken kannst, legst du alle aufgedeckten Karten ab. Enthält die gerade ausgelegte Geldkarte zusätzliche Anweisungen, so führst du diese Anweisungen nun aus. Wenn du z. B. durch das Abenteuer ein weiteres Abenteuer aufdeckst, deckst du erneut Karten auf, bis du eine weitere Geldkarte \enquote{findest}. Denke daran, dass du die Geldkarten aus deiner Hand in beliebiger Reihenfolge auslegen kannst. Hast du z. B. das Abenteuer und den Lohn auf der Hand, darfst du wählen, welche der Karten du zuerst auslegst.}
\end{tikzpicture}
\hspace{-0.6cm}
\begin{tikzpicture}
	\card
	\cardstrip
	\cardbanner{banner/white.png}
	\cardicon{icons/coin.png}
	\cardprice{4}
	\cardtitle{Arbeiterdorf}
	\cardcontent{Du ziehst zuerst eine Karte nach, dann darfst du 2 weitere Aktionskarten auslegen und in der Kaufphase eine weitere Karte kaufen.}
\end{tikzpicture}
\hspace{-0.6cm}
\begin{tikzpicture}
	\card
	\cardstrip
	\cardbanner{banner/white.png}
	\cardicon{icons/coin.png}
	\cardprice{7}
	\cardtitle{Ausbau}
	\cardcontent{Du kannst den ausgelegten Ausbau selbst nicht entsorgen, da du die Karte nicht mehr auf der Hand hast, wenn du die Anweisung ausführst. Hast du einen weiteren Ausbau auf der Hand, kannst du diesen jedoch entsorgen. Hast du keine Karte auf der Hand, die du entsorgen könntest, darfst du dir auch keine neue Karte nehmen. Die neue Karte darf bis zu \coin[3] mehr kosten, als die entsorgte Karte. Du darfst keine Geldkarten oder virtuelles Geld verwenden, um den Betrag zu erhöhen. Du darfst auch eine Karte nehmen, die weniger kostet oder eine der entsorgten identische Karte nehmen. Die Karte, die du nimmst, legst du auf deinen Ablagestapel.}
\end{tikzpicture}
\hspace{-0.6cm}
\begin{tikzpicture}
	\card
	\cardstrip
	\cardbanner{banner/gold.png}
	\cardicon{icons/coin.png}
	\cardprice{7}
	\cardtitle{Bank}
	\cardcontent{Diese Geldkarte hat einen variablen Wert. Die Bank hat einen Wert von \coin[1] für jede Geldkarte, die zu diesem Zeitpunkt im Spiel ist. Die Bank selbst wird dabei mitgezählt. Hierfür ist es wichtig, dass du Geldkarten in beliebiger Reihenfolge auslegen kannst. Wenn du die Bank alleine ausliegen hast, hat sie einen Wert von \coin[1]. Wenn du 2 Banken hintereinander auslegst, hat die erste einen Wert von \coin[1], die zweite Bank einen Wert von \coin[2]. Der Wert jeder Bank wird sofort nach dem Auslegen bestimmt und ändert sich nicht mehr z. B. durch das Auslegen weiterer Geldkarten.}
\end{tikzpicture}
\hspace{-0.6cm}
\begin{tikzpicture}
	\card
	\cardstrip
	\cardbanner{banner/white.png}
	\cardicon{icons/coin.png}
	\cardprice{4}
	\cardtitle{Bischof}
	\cardcontent{Für diese Karte werden die Punkte-Marker benötigt (siehe neue Regeln). Zuerst erhältst du +\coin[1] und 1 Punkte-Marker auf dein Tableau. Dann musst du eine Karte entsorgen, wenn du noch eine Karte auf der Hand hast. Du darfst dir eine Anzahl Punkte-Marker nehmen, die der Hälfte der Geld-Kosten der entsorgten Karte entspricht. Es werden dabei nur die Geld-Kosten beachtet. Trank-Kosten werden ignoriert. Wenn du z. B. den Golem (\coin[4], \potion) (Dominion – Die Alchemisten) entsorgst, nimmst du dir 2 zusätzliche Punkte-Marker und legst diese auf dein Tableau. Danach dürfen deine Mitspieler reihum jeweils eine Karte entsorgen. Sie erhalten jedoch keine Punkte-Marker.}
\end{tikzpicture}
\hspace{-0.6cm}
\begin{tikzpicture}
	\card
	\cardstrip
	\cardbanner{banner/white.png}
	\cardicon{icons/coin.png}
	\cardprice{4}
	\cardtitle{Denkmal}
	\cardcontent{Für diese Karte werden die Punkte-Marker benötigt (siehe neue Regeln).}
\end{tikzpicture}
\hspace{-0.6cm}
\begin{tikzpicture}
	\card
	\cardstrip
	\cardbanner{banner/white.png}
	\cardicon{icons/coin.png}
	\cardprice{5}
	\cardtitle{Gesindel}
	\cardcontent{Du ziehst zunächst 3 Karten nach. Dann muss, beginnend mit dem Spieler links von dir, jeder Mitspieler 3 Karten von seinem Nachziehstapel aufdecken. Wenn er (auch nach dem eventuell nötigen Mischen seines Ablagestapels) nur weniger als 3 Karten aufdecken kann, deckt er nur so viele auf, wie möglich. Dann legt er alle auf diese Weise aufgedeckten Aktions- und Geldkarten auf seinen Ablagestapel. Karten mit kombinierten Kartentypen (z. B. Harem, Dominion – Die Intrige) sind davon auch betroffen. Die übrigen der gerade aufgedeckten Karten legt er in beliebiger Reihenfolge zurück auf seinen Nachziehstapel.}
\end{tikzpicture}
\hspace{-0.6cm}
\begin{tikzpicture}
	\card
	\cardstrip
	\cardbanner{banner/white.png}
	\cardicon{icons/coin.png}
	\cardprice{5}
	\cardtitle{Gewölbe}
	\cardcontent{Zuerst ziehst du 2 Karten nach. Dann legst du \enquote{eine beliebige Anzahl} Handkarten ab. Du kannst auch 0 Karten ablegen. Du kannst auch Karten ablegen, die du gerade nachgezogen hast. Beginnend mit dem Spieler links von dir, entscheidet sich dann jeder Mitspieler, ob er Handkarten ablegen möchte oder nicht. Wenn er sich dafür entscheidet Karten abzulegen, muss er genau 2 Karten ablegen und zieht dann eine Karte nach. Der Spieler kann sich auch für das Ablegen entscheiden, wenn er nur 1 Karte auf der Hand hat. In diesem Fall legt er diese ab, zieht dann allerdings keine Karte nach. Der Spieler darf sich nicht dafür entscheiden, nur 1 Karte abzulegen, wenn er 2 oder mehr Karten auf der Hand hat.}
\end{tikzpicture}
\hspace{-0.6cm}
\begin{tikzpicture}
	\card
	\cardstrip
	\cardbanner{banner/white.png}
	\cardicon{icons/coin.png}
	\cardprice{6}
	\cardtitle{\scriptsize{Großer Markt}}
	\cardcontent{Du musst nicht alle Geldkarten aus deiner Hand auslegen. Kupfer, die du auf der Hand hast, verbieten nicht, den Großen Markt zu kaufen. Kupfer, die in diesem Zug im Spiel waren, es nun aber nicht mehr sind, verbieten dir auch nicht, den Großen Markt zu kaufen. Hast zu z. B. 2 Käufe und 11 Kupfer auslegen und kaufst zuerst den Münzer, so entsorgst du nach diesem Kauf alle Geldkarten im Spiel. \emph{Nun hast du keine Kupfer mehr im Spiel und kannst den Großen Markt kaufen} (siehe auch: Münzer). Wenn du den Großen Markt auf eine andere Art nimmst, z. B. durch den Ausbau, so verhindern Kupfer, die im Spiel sind, dies auch nicht. Andere Geldkarten als Kupfer verbieten dir nicht, den Großen Markt zu kaufen, selbst wenn diese auch \coin[1] wert sind, wie z. B. Lohn.}
\end{tikzpicture}
\hspace{-0.6cm}
\begin{tikzpicture}
	\card
	\cardstrip
	\cardbanner{banner/white.png}
	\cardicon{icons/coin.png}
	\cardprice{6}
	\cardtitle{\scriptsize{Halsabschneider}}
	\cardcontent{Für diese Karte werden die Punkte-Marker benötigt (siehe neue Regeln). Zunächst erhältst du +1 Kauf und +\coin[2]. Dann müssen deine Mitspieler reihum ihre Handkarten auf 3 reduzieren. Für jede Karte, die du in dieser Runde kaufst, nimmst du dir einen Punkte-Marker und legst diesen auf dein Tableau. Hast du mehrere Handlanger im Spiel, erhältst du für jeden Handlanger einen Punkte-Marker pro gekaufter Karte. Hast du jedoch den Handlanger auf den Königshof gespielt, erhältst du dafür nur einen Punkte-Marker, weil die Karte nur einmal im Spiel ist.}
\end{tikzpicture}
\hspace{-0.6cm}
\begin{tikzpicture}
	\card
	\cardstrip
	\cardbanner{banner/white.png}
	\cardicon{icons/coin.png}
	\cardprice{3}
	\cardtitle{\footnotesize{Handelsroute}}
	\cardcontent{\tiny{\begin{Spacing}{1}
	\vspace{1em}
	Du erhältst einen zusätzlichen Kauf in deiner Kaufphase und +\coin[1] für jeden Geld-Marker, der auf dem Tableau Handelsroute liegt, wenn du die Karte auslegst. Dann musst du eine Karte aus deiner Hand entsorgen (wenn du noch eine Karte auf der Hand hast). 
 
	\emph{Spielvorbereitung:} Wenn die Handelsroute im Spiel verwendet wird (entweder als eine der 10 Königreichkarten oder im Schwarzmarktstapel), wird bei der Spielvorbereitung das Tableau Handelsroute bereit gelegt. Zusätzlich wird auf jeden Punktekartenstapel im gesamten Vorrat ein Geld-Marker gelegt. Punktekarten sind: die Basiskarten: Anwesen, Herzogtum, Provinz und Kolonie, die Königreichkarten: z. B. Gärten (Dominion – Basisspiel) und Herzog (Dominion – Die Intrige), kombinierte Karten: z. B. Harem und Adelige (Dominion – Die Intrige). Punktekarten, die nicht im Vorrat sind, werden nicht beachtet, z. B. aus dem Schwarzmarktstapel (Promokarte Schwarzmarkt). Auf den Ritter-Stapel (Dominion – Dark Ages) wird kein Marker gelegt, auch wenn die oberste Karte eine Punktekarte ist.
 
	\emph{Tableau Handelsroute:} Wird im Spielverlauf jeweils die erste Karte eines Punktekartenstapels genommen oder gekauft, legt der Spieler den Geld-Marker von diesem Stapel auf das Tableau Handelsroute. Legt ein Spieler die Handelsroute aus, so erhält er +1 virtuelles Geld für jeden Geld-Marker, der zu diesem Zeitpunkt auf dem Tableau liegt. Dabei ist egal, welcher Spieler den Geld-Marker auf das Tableau gelegt hat. Es wird nur beim Kaufen oder Nehmen der ersten Karte jedes Punktekartenstapels ein Geld-Marker auf das Tableau Handelsroute gelegt, also nur wenn der Geld-Marker, der bei Spielaufbau auf den Stapel gelegt wurde, noch dort liegt. Die Geld-Marker bleiben für den Rest des Spiels auf dem Tableau. Geld-Marker werden nicht vom Tableau entfernt und während des Spiels werden keine neuen Geld-Marker auf die Punktekartenstapel gelegt. Z. B. wenn ein Punktekartenstapel im Vorrat wieder durch den Botschafter (Dominion – Seaside) aufgefüllt wird.
	\end{Spacing}}}
\end{tikzpicture}
\hspace{-0.6cm}
\begin{tikzpicture}
	\card
	\cardstrip
	\cardbanner{banner/white.png}
	\cardicon{icons/coin.png}
	\cardprice{8*}
	\cardtitle{Hausierer}
	\cardcontent{Normalerweise kostet diese Karte \coin[8]. In der Kaufphase kostet der Hausierer pro Aktionskarte, die du selbst im Spiel hast, \coin[2] weniger. Das betrifft auch Hausierer auf der Hand oder in den Stapeln aller Spieler. Die Kosten sinken niemals unter \coin[0]. Wenn du z. B. das Arbeiterdorf auf den Königshof spielst, hast du 2 Aktionskarten im Spiel, obwohl du das Arbeiterdorf dreimal ausgespielt hast. Wenn du den Hausierer ausserhalb der üblichen Kaufphase (z. B. durch den Schwarzmarkt) kaufst oder nimmst, kostet der Hausierer \coin[8].}
\end{tikzpicture}
\hspace{-0.6cm}
\begin{tikzpicture}
	\card
	\cardstrip
	\cardbanner{banner/gold.png}
	\cardicon{icons/coin.png}
	\cardprice{6}
	\cardtitle{Hort}
	\cardcontent{Diese Geldkarte hat einen Wert von \coin[2], wie ein Silber. Wenn diese Karte im Spiel ist, und du eine Punktekarte kaufst, nimmst du dir zusätzlich ein Gold und legst es auf deinen Ablagestapel. Wenn kein Gold mehr im Vorrat ist, nimmst du nichts. Wenn du mehrere Karten Hort im Spiel hast, erhältst du mehrere Gold für den Kauf einer Punktekarte. Wenn du z. B. +1 Kauf hast und 2 Karten Hort auslegst, könntest du 2 Anwesen kaufen und dazu 4 Gold bekommen. Auch kombinierte Karten, wie z. B. Adlige und Harem (Dominion – Die Intrige) sind Punktekarten. Du nimmst dir ein Gold, auch wenn du z. B. den Wachturm vorzeigst und die gekaufte Punktekarte sofort entsorgst. Du nimmst dir nur ein Gold, wenn du eine Punktekarte tatsächlich kaufst, wenn du sie auf eine andere Weise nimmst, bekommst du kein Gold.}
\end{tikzpicture}
\hspace{-0.6cm}
\begin{tikzpicture}
	\card
	\cardstrip
	\cardbanner{banner/gold.png}
	\cardicon{icons/coin.png}
	\cardprice{5}
	\cardtitle{\tiny{Königliches Siegel}}
	\cardcontent{Diese Geldkarte hat den Wert \coin[2], wie ein Silber. Wenn du mehrere Karten nimmst oder kaufst, kannst du für jede Karte getrennt entscheiden, ob du sie auf den Ablagestapel oder auf den Nachziehstapel legst. Ist dein Nachziehstapel leer und du entscheidest dich, eine Karte auf den Nachziehstapel zu legen, so wird dies die einzige Karte deines Nachziehstapels. Karten, die du durch die Besessenheit (Dominion – Die Alchemisten) im Zug eines anderen Spielers erhältst, darfst du nicht auf den Nachziehstapel legen, da das Königliche Siegel nur für den besessenen Spieler gilt.}
\end{tikzpicture}
\hspace{-0.6cm}
\begin{tikzpicture}
	\card
	\cardstrip
	\cardbanner{banner/white.png}
	\cardicon{icons/coin.png}
	\cardprice{7}
	\cardtitle{Königshof}
	\cardcontent{\emph{Errata:} Der Text auf der Karte sollte heißen: \enquote{Du darfst eine Aktionskarte aus deiner Hand wählen. Spiele diese Aktionskarte dreimal aus.} 

	\smallskip
	
	Diese Karte funktioniert ähnlich wie der Thronsaal (Dominion – Basisspiel), mit dem Unterschied, dass du die gewählte Karte 3mal spielst. Du wählst also eine Aktionskarte aus deiner Hand, legst sie aus, führst sie komplett aus, nimmst sie zurück auf die Hand, legst sie erneut aus, führst die Anweisungen nochmals aus, nimmst sie wieder zurück auf die Hand und legst sie dann ein drittes Mal aus. Dieses dreimalige Auslegen verbraucht keine Aktionen. (Das Auslegen des Königshofs verbraucht eine Aktion.) Du darfst keine anderen Aktionskarten auslegen, bis du die gewählte Karte 3mal ausgelegt und ausgeführt hast, ausser die Anweisung auf der gewählte Karte erlaubt es explizit, wie es z. B. der Königshof selbst tut. Wenn du eine Karte wählst, die +1 Aktion gibt, hast du am Ende +3 Aktionen. Legst du einen Königshof auf einen anderen Königshof aus, so wählst du nacheinander 3 Karten aus und legst jede gewählte Karte 3mal aus. Du legst also nicht eine Karte 9mal aus.}
\end{tikzpicture}
\hspace{-0.6cm}
\begin{tikzpicture}
	\card
	\cardstrip
	\cardbanner{banner/white.png}
	\cardicon{icons/coin.png}
	\cardprice{7}
	\cardtitle{\footnotesize{Kunstschmiede}}
	\cardcontent{Du darfst eine beliebige Anzahl Karten aus deiner Hand entsorgen. Das bedeutet, du darfst auch 0 Karten entsorgen. Dafür musst du dir eine Karte nehmen, die 0 Geld kostet. Dies Anweisung unterscheidet sich von Karten wie z. B. Ausbau, weil hier die Summe der Kosten aller entsorgten Karten betrachtet wird, nicht die Anzahl oder der Wert der einzelnen Karten. Ist im Vorrat keine Karte, die exakt soviel kostet, wie die entsorgten Karten, darfst du dir keine Karte nehmen. Es werden dabei nur die Geld-Kosten beachtet. Trank-Kosten (Dominion – Die Alchemisten) werden ignoriert. Du zählst weder Trank zur Summe, noch darfst du eine Karte mit Trank-Kosten nehmen.}
\end{tikzpicture}
\hspace{-0.6cm}
\begin{tikzpicture}
	\card
	\cardstrip
	\cardbanner{banner/white.png}
	\cardicon{icons/coin.png}
	\cardprice{5}
	\cardtitle{Leihaus}
	\cardcontent{Diese Karte erlaubt dir, deinen Ablagestapel durchzusehen (was normalerweise nicht erlaubt ist). Du darfst deinen Ablagestapel nur durchsehen, wenn du das Leihhaus gerade ausgelegt hast. Du musst deinen Mitspielern nicht die gesamten Karten deines Ablagestapels zeigen, nur die Kupfer, die du auf die Hand nimmst. Nachdem du die Kupfer auf die Hand genommen hast, legst du die übrigen Karten in beliebiger Reihenfolge auf den Ablagestapel zurück.}
\end{tikzpicture}
\hspace{-0.6cm}
\begin{tikzpicture}
	\card
	\cardstrip
	\cardbanner{banner/gold.png}
	\cardicon{icons/coin.png}
	\cardprice{3}
	\cardtitle{Lohn}
	\cardcontent{(Wir haben bei der Übersetzung von \enquote{Loan} aus dem englischen Original bewusst \enquote{Lohn} gewählt, da uns dieser Name als passender erscheint, als eine sprachlich korrekte Übersetzung.) Diese Geldkarte hat einen Wert von \coin[1], wie Kupfer. Wenn du sie auslegst, deckst du solange Karten von deinem Nachziehstapel auf, bis du eine Geldkarte aufgedeckt hast. Dann entscheidest du dich, ob du die aufgedeckte Geldkarte entsorgst oder ablegst. Danach legst du alle übrigen aufgedeckten Karten ab. Wenn du (auch nach dem Mischen deines Ablagestapels) keine Karten mehr aufdecken kannst, legst du alle aufgedeckten Karten ab. Beachte, dass du deine Geldkarten in beliebiger Reihenfolge auslegen kannst und nicht alle Geldkarten aus deiner Hand auslegen musst.}
\end{tikzpicture}
\hspace{-0.6cm}
\begin{tikzpicture}
	\card
	\cardstrip
	\cardbanner{banner/white.png}
	\cardicon{icons/coin.png}
	\cardprice{5}
	\cardtitle{Münzer}
	\cardcontent{Wenn du diese Karte kaufst, entsorgst du alle Geldkarten, die zu diesem Zeitpunkt im Spiel sind und nur diese (nicht aus deiner Hand oder sonst woher). Bedenke, dass du nicht alle Geldkarten auslegen musst. Wenn du in dieser Runde mehrere Karten kaufst, entsorgst du die Geldkarten im Spiel, direkt nachdem du den Münzer gekauft hast. Geldkarten, die du in dieser Runde ausgelegt hast, haben aber bereits Geld \enquote{produziert}, auch wenn du sie entsorgst. Du kannst also für den gesamten Wert der ausgelegten Geldkarten neue Karten kaufen. Du kannst allerdings keine weiteren Geldkarten mehr auslegen, sobald du eine Karte gekauft hast. Wenn du den Münzer auslegst, darfst du eine Geldkarte aus deiner Hand aufdecken. Dann nimmst du dir sofort eine identische Karte aus dem Vorrat und legst diese auf deinen Ablagestapel. Die aufgedeckte Geldkarte nimmst du zurück auf die Hand. Die aufgedeckte Geldkarte kann auch eine kombinierte Karte sein, z. B. Harem (Dominion – Die Intrige). Wenn du den Münzer kaufst und den Wachturm aus deiner Hand aufdeckst, darfst du den Münzer sofort auf deinen Nachziehstapel legen. Die ausgelegten Geldkarten müssen jedoch entsorgt werden. Wenn du den Münzer kaufst, während das Königliche Siegel im Spiel ist, wird das Königliche Siegel entsorgt, bevor du den Münzer zurück auf deinen Nachziehstapel legen dürftest.}
\end{tikzpicture}
\hspace{-0.6cm}
\begin{tikzpicture}
	\card
	\cardstrip
	\cardbanner{banner/white.png}
	\cardicon{icons/coin.png}
	\cardprice{5}
	\cardtitle{Quacksalber}
	\cardcontent{Beginnend mit dem Spieler links von dir muss jeder Mitspieler entscheiden, ob er einen Fluch aus seiner Hand ablegt oder sich ein Kupfer und einen Fluch vom Vorrat nimmt und auf seinen Ablagestapel legt. Er darf sich auch für die zweite Möglichkeit entscheiden, wenn einer oder beide Stapel leer sind. In diesem Fall nimmt er sich nur eine der noch vorhandenen Karten oder, wenn beide Stapel leer sind, keine Karte. Deckt ein Spieler einen Burggraben (Dominion – Basisspiel) aus seiner Hand auf, darf er weder eine Karte nehmen noch eine Karte ablegen. Er kann nicht nur einen Teil des Angriffs abwehren. Entscheidet sich ein Spieler für die zweite Möglichkeit und deckt einen Wachturm aus seiner Hand auf, darf er sofort eine oder beide Karten entsorgen.}
\end{tikzpicture}
\hspace{-0.6cm}
\begin{tikzpicture}
	\card
	\cardstrip
	\cardbanner{banner/gold.png}
	\cardicon{icons/coin.png}
	\cardprice{5}
	\cardtitle{\scriptsize{Schmuggelware}}
	\cardcontent{Diese Geldkarte hat einen Wert von \coin[3], wie ein Gold. Wenn du sie ausspielst, erhältst du zunächst +1 Kauf. Dann benennt der Spieler links von dir eine Karte, die du in dieser Runde nicht kaufen darfst. Er kann auch eine Karte benennen, die nicht im Vorrat ist, aber z. B. im Schwarzmarkt-Stapel. Wenn du mehrere Karten Schmuggelware auslegst, benennt der Spieler links von dir jedesmal eine Karte. Du darfst in diesem Zug keine der benannten Karten kaufen. Hierfür ist es wichtig, dass du Geldkarten in beliebiger Reihenfolge auslegen kannst. Du kannst also z. B. zuerst eine Schmuggelware auslegen, dann benennt der Spieler eine Karte, danach kannst du weitere Geldkarten auslegen. Die Anzahl der Karten, die ein Spieler noch auf der Hand hält, ist für die übrigen Spieler sichtbar. Du darfst die benannte Karte nicht kaufen, du darfst sie aber nehmen, wenn dir dies eine Anweisung einer anderen Karte erlaubt, z. B. Hort. Beachte, dass du in dieser Runde keine weiteren Geldkarten mehr auslegen darfst, sobald du eine Karte gekauft hast.}
\end{tikzpicture}
\hspace{-0.6cm}
\begin{tikzpicture}
	\card
	\cardstrip
	\cardbanner{banner/white.png}
	\cardicon{icons/coin.png}
	\cardprice{5}
	\cardtitle{Stadt}
	\cardcontent{\emph{Errata:} Auf der Karte Stadt hat sich ein Tippfehler eingeschlichen. Statt \enquote{1+ Kauf} sollte es heißen \enquote{+1 Kauf}.

	\smallskip
	
	Du ziehst zuerst eine Karte nach und erhältst +2 Aktionen. Wenn mindestens ein Stapel im Vorrat leer ist, ziehst du eine weitere Karte nach. Wenn genau 1 Stapel im Vorrat leer ist, erhältst du nur +1 Karte. Oder wenn 2 oder mehr Stapel im Vorrat leer sind, erhältst du +1 Karte, +\coin[1] und +1 Kauf. Es gibt keine weiteren Boni, wenn 3 oder mehr Stapel im Vorrat leer sind. Die jeweiligen Bedingungen müssen beim Ausspielen der Karte erfüllt sein. 

	\smallskip
	 
	Der Effekt der Karte wird nicht rückwirkend verändert, wenn ein Stapel im Vorrat später leer wird (z. B. durch den Ausbau) oder auch wieder aufgefüllt wird (z. B. durch den Botschafter, Dominion-Seaside). 

	\smallskip
	
	Der Müllstapel ist nicht Teil des Vorrats und wird somit nicht beachtet. Sind beim Ausspielen der Karte also z. B. 2 Stapel leer, so erhältst du insgesamt +2 Karten, +2 Aktionen, +\coin[1] und +1 Kauf.}
\end{tikzpicture}
\hspace{-0.6cm}
\begin{tikzpicture}
	\card
	\cardstrip
	\cardbanner{banner/gold.png}
	\cardicon{icons/coin.png}
	\cardprice{4}
	\cardtitle{Steinbruch}
	\cardcontent{Diese Geldkarte hat der Wert 1 Geld, wie ein Kupfer. Wenn diese Karte im Spiel ist, kosten Aktionskarten 2 Geld weniger. Dieser Effekt ist kumulativ. Wenn du z. B. in der Kaufphase 2 Steinbrüche ausspielst, kosten Aktionskarten um 4 Geld weniger. Der Effekt kann auch durch andere Karten ergänzt werde. Spielst du z. B. in der Aktionsphase ein Arbeiterdorf und in der Kaufphase 2 Steinbrüche, kostet der Hausierer nur noch 2 Geld. Handkarten, Karten im Nachziehstapel und im Ablagestapel sind auch durch diesen Effekt betroffen. Kombinierte Karten, wie z. B. Adelige (Dominion – Die Intrige) sind auch Aktionskarten.}
\end{tikzpicture}
\hspace{-0.6cm}
\begin{tikzpicture}
	\card
	\cardstrip
	\cardbanner{banner/gold.png}
	\cardicon{icons/coin.png}
	\cardprice{4}
	\cardtitle{Talisman}
	\cardcontent{Diese Geldkarte hat den Wert 1 Geld, wie ein Kupfer. Wenn diese Karte im Spiel ist und du eine Karte kaufst, die 4 Geld oder weniger kostet und die keine Punktekarte ist, nimmst du dir zusätzlich eine weitere identische Karte. Du nimmst diese Karte vom Vorrat und legst sie auf deinen Ablagestapel. Gibt es keine weitere Karte mit diesem Namen im Vorrat, nimmst du dir nichts. Hast du mehrere Talismane im Spiel, nimmst du dir für jeden Talisman eine zusätzliche Karte. Wenn du mehrere Karten kaufst, auf diese Bedingungen zutreffen (4 Geld oder weniger, keine Punktekarte), nimmst du dir von jeder eine weitere Karte. Wenn du z. B. 2 Talismane, 4 Kupfer und 2 Käufe hast und dir ein Silber und eine Handelsroute kaufst, nimmst du dir zusätzlich 2 weitere Silber und 2 weitere Handelsrouten. Der Talisman wirkt nur für Karten, die du kaufst. Nimmst du dir eine Karte auf eine andere Weise, z. B. durch den Ausbau, erhältst du keine weitere. Kombinierte Karten, wie z. B. die Große Halle (Dominion – Die Intrige) sind auch Punktekarten. Bei den Kosten von 4 Geld oder weniger werden die aktuell geltenden Kosten betrachtet. Das sind nicht unbedingt die aufgedruckten Kosten. Wenn du z. B. in deiner Aktionsphase 2 Aktionskarten ausgelegt hast, kostet der Hausierer nur noch 4 Geld und du darfst dir somit einen weiteren Hausierer nehmen.}
\end{tikzpicture}
\hspace{-0.6cm}
\begin{tikzpicture}
	\card
	\cardstrip
	\cardbanner{banner/blue.png}
	\cardicon{icons/coin.png}
	\cardprice{3}
	\cardtitle{Wachturm}
	\cardcontent{\tiny{\begin{Spacing}{1}
	Wenn du diese Karte in deinem Zug auslegst, ziehst du solange Karten von deinem Nachziehstapel, bis du 6 Karten auf der Hand hast. Hast du nach dem Ausspielen des Wachturmes bereits 6 oder mehr Karten auf der Hand, ziehst du keine Karte nach. Immer wenn du eine Karte nimmst oder kaufst, egal ob in deinem eigenen Zug oder im Zug eines Mitspielers, darfst du den Wachturm aus deinem Hand aufdecken und dann entscheiden, ob du die neue Karte entsorgst oder oben auf deinen Nachziehstapel legst. Du darfst den Wachturm jedesmal, wenn du eine Karte nimmst oder kaufst aus deiner Hand aufdecken. Wie üblich bei Reaktionskarten, deckst du den Wachturm nur auf und nimmst ihn dann zurück auf deine Hand. Spielt ein Mitspieler z. B. den Quacksalber, kannst du den Wachturm nur bei einer oder bei beiden Karten (Kupfer und Fluch) aufdecken und für jede Karte getrennt entscheiden. Du kannst mit dem Wachturm auch nacheinander auf mehrere Angriffe unterschiedlicher Mitspieler reagieren und ihn dann in deinem eigenen Zug nochmals einsetzen. Auch wenn du dich dafür entscheidest eine Karte, die du gerade genommen hast oder nehmen musstest, zu entsorgen, musst du diese Karte zuerst nehmen. Die Karte ist also nicht mehr im Vorrat, und andere Karten, die auf genommene Karten Bezug nehmen, wie z. B. die Schmuggler (Dominion – Seaside), können auch darauf angewandt werden. Wenn während des Extrazuges durch die Besessenheit (Dominion – Die Alchemisten) Karten genommen oder gekauft werden, kannst du den Wachturm nicht aus deiner Hand aufdecken, da der besessene Spieler die Karte nimmt. Du kannst den Wachturm auch aus deiner Hand aufdecken, wenn du eine Karte nimmst, die du nicht wie üblich auf deinen Ablagestapel legst, wie z. B. die Mine (Dominion – Basisspiel).
	\end{Spacing}}}
\end{tikzpicture}
\hspace{-0.6cm}
\begin{tikzpicture}
	\card
	\cardstrip
	\cardbanner{banner/gold.png}
	\cardicon{icons/coin.png}
	\cardprice{9}
	\cardtitle{Platin}
	\cardcontent{}
\end{tikzpicture}
\hspace{-0.6cm}
	\begin{tikzpicture}
	\card
	\cardstrip
	\cardbanner{banner/green.png}
	\cardicon{icons/coin.png}
	\cardprice{11}
	\cardtitle{Kolonien}
	\cardcontent{}
\end{tikzpicture}
\hspace{-0.6cm}
\begin{tikzpicture}
	\card
	\cardstrip
	\cardbanner{banner/white.png}
	\cardtitle{\scriptsize{Empfohlene 10er Sätze\qquad}}
	\cardcontent{\emph{Blütezeit und Basisspiel:}

	\smallskip 
	
	\emph{Haufenweise Geld:} \\ 
	Abenteuer, Bank, Großer Markt, Königliches Siegel, Münzer, Abenteurer, Geldverleiher, Laboratorium, Mine, Spion 

	\smallskip 
	
	\emph{Die Armee des Königs:} \\ 
	Ausbau, Gesindel, Gewölbe, Handlanger, Königshof, Bürokrat, Burggraben, Dorf, Ratsversammlung, Spion 

	\smallskip 
	
	\emph{Ein gutes Leben:} \\ 
	Denkmal, Hort, Leihaus, Quacksalber, Schmuggelware, Bürokrat, Dorf, Gärten, Kanzler, Keller}
\end{tikzpicture}
\hspace{-0.6cm}
\begin{tikzpicture}
	\card
	\cardstrip
	\cardbanner{banner/white.png}
	\cardtitle{\scriptsize{Empfohlene 10er Sätze\qquad}}
	\cardcontent{\emph{Blütezeit und Die Intrige:}

	\smallskip 
	
	\emph{Pfade zum Sieg:} \\ 
	Bischof, Denkmal, Halsabschneider, Hausierer, Leihaus, Anbau, Armenviertel, Baron, Handlanger, Harem
	
	\smallskip 
	
	\emph{All along the watchtower:} \\ 
	Gewölbe, Handelsroute, Hort, Talisman, Wachturm, Bergwerk, Brücke, Große Halle, Handlanger, Kerkermeister

	\smallskip 
	
	\emph{Glücksritter:} \\ 
	Ausbau, Bank, Gewölbe, Königshof, Kunstschmiede, Brücke, Kupferschmied, Tribut, Trickser, Wunschbrunnen}
\end{tikzpicture}
\hspace{-0.6cm}
\begin{tikzpicture}
	\card
	\cardstrip
	\cardbanner{banner/white.png}
	\cardtitle{\scriptsize{Empfohlene 10er Sätze\qquad}}
	\cardcontent{\emph{Blütezeit:}

	\smallskip 
	
	\emph{Anfänger:} \\ 
	Abenteuer, Arbeiterdorf, Ausbau, Bank, Denkmal, Gesindel, Halsabschneider, Königliches Siegel, Leihaus, Wachturm 

	\smallskip 
	
	\emph{Freundliche Interaktion:} \\ 
	Arbeiterdorf, Bischof, Gewölbe, Handelsroute, Hausierer, Hort, Königliches Siegel, Kunstschmiede, Schmuggelware, Stadt 

	\smallskip 
	
	\emph{Große Aktionen:} \\ 
	Ausbau, Gesindel, Gewölbe, Großer Markt, Königshof, Lohn, Münzer, Stadt, Steinbruch, Talisman}
\end{tikzpicture}
\hspace{0.6cm}
