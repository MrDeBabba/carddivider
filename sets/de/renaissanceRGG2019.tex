% Basic settings for this card set
\renewcommand{\cardcolor}{renaissance}
\renewcommand{\cardextension}{Erweiterung XI}
\renewcommand{\cardextensiontitle}{Renaissance}
\renewcommand{\seticon}{renaissance.png}

\clearpage
\newpage
\section{\cardextension \ - \cardextensiontitle \ (Rio Grande Games 2019)}

\begin{tikzpicture}
	\card
	\cardstrip
	\cardbanner{banner/white.png}
	\cardicon{icons/coin.png}
	\cardprice{2}
	\cardtitle{Diener}
	\cardcontent{Wenn du diese Karte nimmst, lege 2 \emph{Marker} vom Haufen auf dein Tableau auf die Dorfbewohner-Seite.}
\end{tikzpicture}
\hspace{-0.6cm}
\begin{tikzpicture}
	\card
	\cardstrip
	\cardbanner{banner/gold.png}
	\cardicon{icons/coin.png}
	\cardprice{2}
	\cardtitle{Goldmünze}
	\cardcontent{Wenn du diese Karte ausspielst, bekommst du kein \coin, sondern legst einen Marker auf die Talerseite deines Taler-/Dorfbewohner-Tableaus und erhältst +1 Kauf.
	
	\medskip

	Wenn du diese Karte nimmst, darfst du ein \emph{KUPFER} von deiner Hand entsorgen; dies ist optional. Die \emph{GOLDMÜNZE} ist trotzdem eine \emph{GELD}-Karte und in der Kaufphase auszuspielen, obwohl sie Anweisungen ähnlich einer Aktionskarte enthält. Sie hat einen \emph{GELD}-Wert von 0.}
\end{tikzpicture}
\hspace{-0.6cm}
\begin{tikzpicture}
	\card
	\cardstrip
	\cardbanner{banner/white.png}
	\cardicon{icons/coin.png}
	\cardprice{2}
	\cardtitle{Grenzposten}
	\cardcontent{Wenn ihr den \emph{GRENZPOSTEN} verwendet, sucht zu Spielbeginn das \emph{HORN} und die \emph{LATERNE} heraus und legt sie neben dem Vorrat bereit. 
	
	\medskip
	
	Wenn du einen \emph{GRENZPOSTEN} ausspielst und nicht die \emph{LATERNE} hast, deckst du die obersten 2 Karten deines Nachziehstapels auf, wählst eine davon und nimmst sie zu deinen Handkarten; die andere legst du ab. Wenn beide Karten Aktionskarten sind, erhältst du danach die \emph{LATERNE} oder das \emph{HORN}. Wenn du nicht 2 Karten aufdecken kannst bzw. nicht 3 Karten, wenn du die \emph{LATERNE} hast, erhältst du kein Artefakt. 
	
	\medskip
	
	\emph{Artefakt Horn:} Solange du das \emph{HORN} hast, darfst du einmal pro eigenem Zug beim Ablegen eines \emph{GRENZPOSTENS} aus dem Spiel diesen auf deinen Nachziehstapel legen anstatt auf deinen Ablagestapel. Das \emph{HORN} wirkt schon in dem Zug, in dem du es bekommst.

	\medskip
	\emph{Artefakt Laterne:} Wenn du die \emph{LATERNE} hast, musst du in deinen Zügen bei jedem ausgespielten Grenzposten 3 Karten vom Nachziehstapel aufdecken und 2 ablegen. In diesem Fall müssen alle 3 Karten Aktionskarten sein, damit du das \emph{HORN} erhältst. Die \emph{LATERNE} wirkt schon in dem Zug, in dem du sie bekommst.}
\end{tikzpicture}
\hspace{-0.6cm}
\begin{tikzpicture}
	\card
	\cardstrip
	\cardbanner{banner/white.png}
	\cardicon{icons/coin.png}
	\cardprice{3}
	\cardtitle{Experiment}
	\cardcontent{Wenn du diese Karte nimmst, nimmst du dir ein weiteres \emph{EXPERIMENT} aus dem Vorrat; dies gilt unabhängig davon, ob du die Karte kaufst oder auf irgendeine andere Art und Weise nimmst. Ist kein \emph{EXPERIMENT} mehr im Vorrat, nimmst du keine zusätzliche Karte für diese Karte. Wenn du ein \emph{EXPERIMENT} für einen anderen Ort als deinen Ablagestapel nimmst, legst du das weitere genommene \emph{EXPERIMENT} ab. Wenn du zum Beispiel eine \emph{BILDHAUERIN} benutzt, um ein \emph{EXPERIMENT} zu nehmen, nimmst du eins auf die Hand und legst eins auf deinen Ablagestapel. Wenn du \emph{VOGELFREIE} (aus \emph{Dark Ages}) oder \emph{LEHNSHERR} (aus \emph{Empires}) aus \emph{EXPERIMENT} ausspielst, legst du die Karte jeweils auf ihren eigenen Stapel zurück, nicht auf den \emph{EXPERIMENT}-Stapel. Falls das \emph{EXPERIMENT} aus irgendeinem Grund nicht im Spiel ist (zum Beispiel weil es mit \emph{TOTENBESCHWÖRER} (aus \emph{Nocturne}) aus dem Müll gespielt wurde), wird es nicht auf seinen Stapel zurückgelegt.}
\end{tikzpicture}
\hspace{-0.6cm}
\begin{tikzpicture}
	\card
	\cardstrip
	\cardbanner{banner/white.png}
	\cardicon{icons/coin.png}
	\cardprice{3}
	\cardtitle{Fortschritt}
	\cardcontent{Du darfst nur eine Aktionskarte entsorgen, die in diesem Zug abgelegt werden würde. Du darfst keine Nicht-Aktionskarte wie \emph{SILBER} oder eine Dauerkarte entsorgen, die nicht in diesem Zug abgelegt wird. Den ausgespielten \emph{FORTSCHRITT} selbst darfst du entsorgen. Die Karte, die du nimmst, muss keine Aktionskarte sein, sie muss nur genau \coin[1] mehr kostet als die entsorgte Aktionskarte. Die Benutzung dieser Fähigkeit ist optional, aber wenn du eine Karte entsorgst, musst du eine Karte nehmen, wenn möglich.}
\end{tikzpicture}
\hspace{-0.6cm}
\begin{tikzpicture}
	\card
	\cardstrip
	\cardbanner{banner/orange.png}
	\cardicon{icons/coin.png}
	\cardprice{3}
	\cardtitle{Frachtschiff}
	\cardcontent{Die Karte, die du zur Seite legst, muss nicht die nächste Karte sein, die du nimmst; du könntest eine oder mehrere Karten nehmen und dann eine, die du zur Seite legst. Wenn du keine Karte zur Seite legst, wird das \emph{FRACHTSCHIFF} in diesem Zug abgelegt.}
\end{tikzpicture}
\hspace{-0.6cm}
\begin{tikzpicture}
	\card
	\cardstrip
	\cardbanner{banner/white.png}
	\cardicon{icons/coin.png}
	\cardprice{3}
	\cardtitle{\scriptsize{Schauspieltruppe}}
	\cardcontent{Wenn du es nicht schaffst, diese Karte zu entsorgen (zum Beispiel, weil du sie durch einen \emph{THRONSAAL} aus dem \emph{Basisspiel} zweimal ausspielst), bekommst du trotzdem +4 Dorfbewohner, d.~h. du legst 4 Marker auf die Dorfbewohner-Seite deines Tableaus.}
\end{tikzpicture}
\hspace{-0.6cm}
\begin{tikzpicture}
	\card
	\cardstrip
	\cardbanner{banner/white.png}
	\cardicon{icons/coin.png}
	\cardprice{4}
	\cardtitle{Bergdorf}
	\cardcontent{Wenn dein Ablagestapel Karten enthält, musst du eine von ihnen aufnehmen. Du kannst dich nicht bewusst dafür entscheiden, stattdessen eine Karte zu ziehen. Du darfst deinen Ablagestapel durchschauen und eine Karte heraussuchen, die du nimmst. Die Reihenfolge der Karten in deinem Ablagestapel spielt danach keine Rolle.}
\end{tikzpicture}
\hspace{-0.6cm}
\begin{tikzpicture}
	\card
	\cardstrip
	\cardbanner{banner/white.png}
	\cardicon{icons/coin.png}
	\cardprice{4}
	\cardtitle{Erfinder}
	\cardcontent{Zuerst nimmst du eine Karte, die bis zu \coin[4] kostet; danach sinken die Kosten den Rest deine Zuges um \coin[1]. Die Kostensenkung betrifft alle Karten, einschließlich zur Seite gelegter Karten, aller Karten im Vorrat, in den Händen sowie den Nachzieh- und Ablagestapeln aller Spieler. Sie betrifft auch Karten auf den Sonderstapeln (nicht im Vorrat), z.~B. die Erscheinungen (aus \emph{Nocturne}) und die Reisenden (aus \emph{Abenteuer}), sowie das Schwarzmarktdeck. Nicht betroffen sind Projekte und Ereignisse, weil sie regeltechnisch keine Karten sind. Die Kostensenkung wirkt kumulativ: wenn du zum Beispiel zwei \emph{ERFINDER} ausspielst, wirkt die Kostensenkung des ersten für das Nehmen beim zweiten (du könntest zum Beispiel ein \emph{HERZOGTUM}) nehmen, das dann \coin[4] kosten würde), und danach kosten alle Karten für den Rest des Zuges \coin[2] weniger.}
\end{tikzpicture}
\hspace{-0.6cm}
\begin{tikzpicture}
	\card
	\cardstrip
	\cardbanner{banner/white.png}
	\cardicon{icons/coin.png}
	\cardprice{4}
	\cardtitle{\footnotesize{Fahnenträger}}
	\cardcontent{Wenn du den \emph{FAHNENTRÄGER} nimmst oder entsorgst und nicht die \emph{FAHNE} hast, erhalte die \emph{FAHNE} und lege sie neben dir ab. Du behältst dieses Artefakt (und dessen Effekt), bis ein anderer Spieler die \emph{FAHNE} erhält. Wird der \emph{FAHNENTRÄGER} entsorgt, erhält der Spieler, der ihn entsorgt, die \emph{FAHNE}, unabhängig davon, wer am Zug ist.
	
	\medskip
	
	\emph{Artefakt Fahne:} Die \emph{FAHNE} veranlasst dich, eine weitere Karte zu ziehen, wenn du in der Aufräumphase deine Karten vom Nachziehstapel ziehst. Dies gilt auch, wenn auf deiner Hand normalerweise weniger oder mehr als 5 Karten wären; wenn du zum Beispiel einen \emph{AUSSENPOSTEN} (aus \emph{Seaside}) ausgespielt hast, würdest du statt 3 Karten für den \emph{AUSSENPOSTEN}-Zug 4 Karten ziehen.}
\end{tikzpicture}
\hspace{-0.6cm}
\begin{tikzpicture}
	\card
	\cardstrip
	\cardbanner{banner/blue.png}
	\cardicon{icons/coin.png}
	\cardprice{4}
	\cardtitle{Patron}
	\cardcontent{Wirst du durch eine Anweisung mit der Formulierung \enquote{aufdecken} dazu veranlasst, einen \emph{PATRON} aufzudecken, erhältst du +1 Taler und legst einen Marker auf die Taler-Seite deines Tableaus. Spielst du zum Beispiel einen \emph{GRENZPOSTEN} aus und deckst zwei \emph{PATRONE} auf, erhältst du +2 Taler. Es reicht nicht, dass andere Spieler eine Karte sehen, ohne dass die Formulierung \enquote{aufdecken} benutzt wird. Wenn zum Beispiel ein anderer Spieler einen \emph{SCHWARZEN MEISTER} ausspielt und du einen \emph{PATRON} ablegst, bekommst du nicht +1 Taler. Wenn ein Spieler aber zum Beispiel 3 \emph{ERFINDER} ausgespielt hat, sodass die Kosten der Karten um \coin[3] reduziert sind, kostet der \emph{PATRON} auf deiner Hand nur noch \coin[1]. Spielt dieser Spieler dann einen \emph{SCHWARZEN MEISTER} aus und du hast keine andere Karten mit Kosten von \coin[2] oder mehr auf der Hand, musst du den \emph{PATRON} \enquote{aufdecken} und erhältst +1 Taler.}
\end{tikzpicture}
\hspace{-0.6cm}
\begin{tikzpicture}
	\card
	\cardstrip
	\cardbanner{banner/orange.png}
	\cardicon{icons/coin.png}
	\cardprice{4}
	\cardtitle{Forscherin}
	\cardcontent{Für jedes \coin[1], das die entsorgte Karte kostet, legst du die oberste Karte deines Nachziehstapels für den nächsten Zug zur Seite (auf diese \emph{FORSCHERIN}). Wenn du zum Beispiel ein \emph{SILBER} entsorgst, legst du die obersten 3 Karten deines Nachziehstapels für deinen nächsten Zug zur Seite. Sind nicht genügend Karten im Nachziehstapel, mischst du zuerst deinen Ablagestapel und legst die Karten verdeckt als Nachziehstapel bereit. Sind dann immer noch nicht genügend Karten im Nachziehstapel, legst du so viele Karten zur Seite, wie du kannst. Die Karten werden verdeckt zur Seite gelegt. Du darfst sie anschauen, die anderen Spieler nicht.}
\end{tikzpicture}
\hspace{-0.6cm}
\begin{tikzpicture}
	\card
	\cardstrip
	\cardbanner{banner/white.png}
	\cardicon{icons/coin.png}
	\cardprice{4}
	\cardtitle{Priester}
	\cardcontent{Wenn du diese Karte ausspielst, bekommst du +\coin[2] und musst eine Karte von deiner Hand entsorgen; für den Rest deines Zuges gilt dann: +\coin[2] für jede Karte, die du entsorgst. Der Bonus wirkt sogar dann, wenn die Karte nicht aus deiner Hand entsorgt wurde; entsorgst du zum Beispiel eine \emph{SCHAUSPIELTRUPPE} aus dem Spiel (aufgrund ihrer eigenen Anweisung), erhältst du den Bonus, ebenso für das Entsorgen einer Karte aus dem Vorrat mit der \emph{HERUMTREIBERIN} (aus \emph{Intrige}). Ein \emph{PRIESTER} wirkt kumulativ, auch wenn der gleiche \emph{PRIESTER} mehrmals gespielt wird (z.~B. mit dem \emph{ZEPTER}).}
\end{tikzpicture}
\hspace{-0.6cm}
\begin{tikzpicture}
	\card
	\cardstrip
	\cardbanner{banner/white.png}
	\cardicon{icons/coin.png}
	\cardprice{4}
	\cardtitle{\scriptsize{Seidenhändlerin}}
	\cardcontent{Wenn du eine \emph{SEIDENHÄNDLERIN} entsorgst oder nimmst (egal ob du am Zug bist oder nicht), bekommst du +1 Taler sowie +1 Dorfbewohner und legst je 1 Marker auf die beiden Seiten deines Tableaus. Die \emph{SEIDENHÄNDLERIN} selbst gibt dir nicht das Recht, sich selbst zu entsorgen.}
\end{tikzpicture}
\hspace{-0.6cm}
\begin{tikzpicture}
	\card
	\cardstrip
	\cardbanner{banner/white.png}
	\cardicon{icons/coin.png}
	\cardprice{4}
	\cardtitle{Versteck}
	\cardcontent{Du musst eine Karte entsorgen. Flüche sind keine Punktekarten. Auch für kombinierte Karten, von denen ein Typ \emph{PUNKTE} ist, musst du einen Fluch nehmen.}
\end{tikzpicture}
\hspace{-0.6cm}
\begin{tikzpicture}
	\card
	\cardstrip
	\cardbanner{banner/white.png}
	\cardicon{icons/coin.png}
	\cardprice{5}
	\cardtitle{Alte Hexe}
	\cardcontent{Auch wenn der Fluch-Stapel leer wird oder ist und deshalb ein oder mehrere Spieler keinen Fluch nehmen können, dürfen trotzdem alle Mitspieler einen Fluch aus ihrer Hand entsorgen. Ist ein Spieler vom Effekt der \emph{ALTEN HEXE} nicht betroffen, z.~B. weil er als Reaktion einen \emph{BURGGRABEN} aufgedeckt hat, nimmt er weder einen Fluch, noch darf er einen entsorgen.}
\end{tikzpicture}
\hspace{-0.6cm}
\begin{tikzpicture}
	\card
	\cardstrip
	\cardbanner{banner/white.png}
	\cardicon{icons/coin.png}
	\cardprice{5}
	\cardtitle{Anwerber}
	\cardcontent{Zuerst ziehst du 2 Karten, dann musst du eine Karte aus deiner Hand entsorgen. Pro \coin[1], das die entsorgte Karte kostet, bekommst du +1 Dorfbewohner. Wenn du zum Beispiel ein \emph{SILBER} entsorgst, bekommst du +3 Dorfbewohner. Du bekommst nichts für \potion und \hex, nur für \coin.}
\end{tikzpicture}
\hspace{-0.6cm}
\begin{tikzpicture}
	\card
	\cardstrip
	\cardbanner{banner/white.png}
	\cardicon{icons/coin.png}
	\cardprice{5}
	\cardtitle{Bildhauerin}
	\cardcontent{Die Karte musst du auf die Hand nehmen. Ist die genommen Karte eine Geldkarte (ggf. auch eine kombinierte), lege 1 Marker auf die Dorfbewohner-Seite deines Tableaus. Die \emph{BILDHAUERIN} hat Vorrang, d.~h. enthält die genommene Karte eine Anweisung, dass sie beim Nehmen nicht auf die Hand genommen werden soll (wie z.~B. beim \emph{NOMADENCAMP} aus \emph{Hinterland}), nimmst du sie trotzdem auf die Hand.}
\end{tikzpicture}
\hspace{-0.6cm}
\begin{tikzpicture}
	\card
	\cardstrip
	\cardbanner{banner/white.png}
	\cardicon{icons/coin.png}
	\cardprice{5}
	\cardtitle{Freibeuterin}
	\cardcontent{Zuerst ziehest du 3 Karten, dann prüfst du, ob dein Ablagestapel Karten enthält. Wenn dich das Ziehen der 3 Karten veranlasst hat zu mischen, ist dein Ablagestapel leer. Wenn dein Ablagestapel mindestens eine Karte enthält, legst du 1 Marker auf die Taler-Seite deines Tableaus, und wenn denn 4 oder mehr Marker auf der Taler-Seite deines Tableaus liegen, erhältst du die \emph{SCHATZKISTE}. Du erhältst die \emph{SCHATZKISTE} nur, wenn du gerade einen Marker aufgrund der Anweisung auf dieser \emph{FREIBEUTERIN} auf die Taler-Seite deines Tableaus gelegt hast. Hast du das nicht getan, erhältst du die \emph{SCHATZKISTE} nicht, selbst wenn 4 Marker oder mehr auf der Taler-Seite deines Tableaus liegen.
	
	\medskip

	\emph{Artefakt Schatzkiste:} Solang du die \emph{SCHATZKISTE} hast, nimmst du in jedem eigenen Zug zu Beginn deiner Kaufphase ein \emph{GOLD}, einschließlich des Zuges, in dem du sie erhältst (aber nicht, wenn du sie erst nach Beginn deiner Kaufphase erhältst).}
\end{tikzpicture}
\hspace{-0.6cm}
\begin{tikzpicture}
	\card
	\cardstrip
	\cardbanner{banner/white.png}
	\cardicon{icons/coin.png}
	\cardprice{5}
	\cardtitle{Gelehrter}
	\cardcontent{Wenn du durch das Ziehen der 7 Karten veranlasst wirst zu mischen, ischst du die abglegten Karten mit ein.}
\end{tikzpicture}
\hspace{-0.6cm}
\begin{tikzpicture}
	\card
	\cardstrip
	\cardbanner{banner/gold.png}
	\cardicon{icons/coin.png}
	\cardprice{5}
	\cardtitle{Gewürze}
	\cardcontent{Diese Karte ist eine Geldkarte und bringt dir \coin[2] sowie +1 Kauf. Wenn du sie nimmst, nimmst du 2 Marker und legst sie auf die Taler-Seite deines Tableaus.}
\end{tikzpicture}
\hspace{-0.6cm}
\begin{tikzpicture}
	\card
	\cardstrip
	\cardbanner{banner/white.png}
	\cardicon{icons/coin.png}
	\cardprice{5}
	\cardtitle{\scriptsize{Schatzmeisterin}}
	\cardcontent{Wenn du die \emph{SCHATZMEISTERIN} ausspielst, erhältst du auf jeden Fall +\coin[3]. Dann hast du die Wahl zwischen drei verschiedenen Optionen: Du kannst auch eine Option wählen, die du nicht erfüllen kannst, oder du kannst die Option \enquote{\emph{SCHLÜSSEL} erhalten} wählen., wenn du den \emph{SCHLÜSSEL} bereits hast. Entschiedest du dich dafür, eine Geldkarte aus dem Müll auf deine Hand zu nehmen, achte darauf, dass Geldkarten mit zusätzlichen Anweisungen möglicherweise Anweisungen enthalten, die beim Nehmen der Karte ausgeführt werden müssen.
	
	\medskip
	
	\emph{Artefakt Schlüssel:} Solange du den \emph{SCHLÜSSEL} hast, erhältst du zu Beginn des Zuges +\coin[1], also noch nicht in dem Zug, in dem du ihn erhältst.}
\end{tikzpicture}
\hspace{-0.6cm}
\begin{tikzpicture}
	\card
	\cardstrip
	\cardbanner{banner/white.png}
	\cardicon{icons/coin.png}
	\cardprice{5}
	\cardtitle{Seher}
	\cardcontent{Karten mit \potion (aus \emph{Alchemie}) oder \hex (aus \emph{Empires}) in ihren Kosten kosten nicht \coin[2] bis \coin[4]. Kannst du keine 3 Karten aufdecken, deckst du so viele auf wie möglich und führst die Aktion damit durch.}
\end{tikzpicture}
\hspace{-0.6cm}
\begin{tikzpicture}
	\card
	\cardstrip
	\cardbanner{banner/white.png}
	\cardicon{icons/coin.png}
	\cardprice{5}
	\cardtitle{\tiny{Schwarzer Meister}}
	\cardcontent{Hat ein Mitspieler keine Karten auf der Hand, die \coin[2] oder mehr kosten, muss er dies \enquote{beweisen}, indem er seine Handkarten aufdeckt.}
\end{tikzpicture}
\hspace{-0.6cm}
\begin{tikzpicture}
	\card
	\cardstrip
	\cardbanner{banner/gold.png}
	\cardicon{icons/coin.png}
	\cardprice{5}
	\cardtitle{Zepter}
	\cardcontent{Diese Karte ist eine Geldkarte und wird wie alle Geldkarten in der entsprechenden Spielphase ausgespielt. Wenn du sie ausspielst, kannst du wählen, ob sie in diesem Zug +\coin[2] bringt oder ob sie \coin[0] bringt und du dafür eine in diesem Zug ausgespielte Aktionskarte, die du noch im Spiel hast, erneut ausspielst. Dauerkarten, die in vergangenen Zügen ausgespielt wurden, können nicht noch einmal ausgespielt werden, in diesem Zug ausgespielte Dauerkarten aber schon (in diesem Fall bleibt das \emph{ZEPTER} auch so lange im Spiel, bis die Dauerkarte abgelegt wird).
	
	\medskip
	
	Anweisungen auf der erneut ausgespielten Aktionskarte, die nur in der Aktionsphase nutzbar sind (z.~B. \enquote{+x Aktionen}) verfallen. Anweisungen wie \enquote{+x Karten} oder \enquote{+\coin[X]} können genutzt werden. Wenn unter den gezogenen Karten Geldkarten sind, darfst du sie noch in diesem Zug ausspielen.}
\end{tikzpicture}
\hspace{-0.6cm}
\begin{tikzpicture}
	\card
	\cardstrip
	\cardbanner{banner/lightred.png}
	\cardicon{icons/coin.png}
	\cardprice{\scriptsize{3-8}}
	\cardtitle{Projekte (1/8)}
	\cardcontent{\emph{Projekte} können wie Königreich-Karten in der Kaufphase gekauft werden. Dies verbraucht jeweils 1 Kauf und kostet einen bestimmten Betrag an \coin, der in der linken oberen Ecke des jeweiligen \emph{Projektes} steht. Wenn du ein \emph{Projekt} kaufst, lege einen deiner Holzsteine darauf. Für den Rest des Spiels hast du diese Fähigkeit. Mehrere Spieler können das gleiche Projekt kaufen und haben dann alle diese Fähigkeit bis zum Ende des Spiels. Sind mehr als 2 Projekte im Spiel, hast du trotzdem nur 2 Steine zum Platzieren. Du darfst nicht beide Steine auf das gleiche \emph{Projekt} legen und keinen Stein von einem \emph{Projekt} zurücklegen, nachdem du ihn dort hingelegt hast.
	
	\bigskip
	
	\emph{Abwassertunnel} (\coin[3]): Es ist egal, was dich dazu veranlasst, eine Karte zu entsorgen, es darf nur nicht der \emph{ABWASSERTUNNEL} sein. Die Fähigkeit des \emph{ABWASSERTUNNELS} kommt z.~B. zum Einsatz, wenn du eine Karte mit einem \emph{PRIESTER} entsorgst, wenn du einen Fluch mit einer \emph{ALTEN HEXE} entsorgst, wenn du eine \emph{SCHAUSPIELTRUPPE} entsorgst, nachdem du sie ausgespielt hast, und wenn du mit einer \emph{HERUMTREIBERIN} (aus \emph{Intrige}) eine Karte vom Vorrat entsorgst. Die mit Hilfe des \emph{ABWASSERTUNNELS} entsorgte Karte kann eine beliebige Karte aus deiner Hand sein, auch wenn der \emph{ABWASSERTUNNEL} durch etwas ausgelöst wurde,  das nur das Entsorgen bestimmter Karten erlaubt.}
\end{tikzpicture}
\hspace{-0.6cm}
\begin{tikzpicture}
	\card
	\cardstrip
	\cardbanner{banner/lightred.png}
	\cardicon{icons/coin.png}
	\cardprice{\scriptsize{3-8}}
	\cardtitle{Projekte (2/8)}
	\cardcontent{\emph{Festzug} (\coin[3]): Wenn du mindestens \coin[1] hast, das du nicht ausgegeben hast, kannst du genau \coin[1] bezahlen, um 1 Marker zu nehmen und auf die Taler-Seite deines Tableaus zu legen. Dies ist nur einmal pro Zug möglich.
	
	\medskip
	
	\emph{Kathedrale} (\coin[3]): Sobald du diese Fähigkeit hast, musst du sie zu Beginn jedes deiner Züge einsetzen. Es gibt keine Möglichkeit, deinen Holzstein zurückzulegen.
	
	\medskip
	
	\emph{Stadttor} (\coin[3]): Zuerst ziehst du eine Karte. Dann legst du eine beliebige Karte aus deiner Hand auf den Nachziehstapel. Das kann auch die gerade gezogene Karte sein. Du musst auch eine Karte zurücklegen, wenn du zuvor keine ziehen konntest.
	
	\medskip
	
	\emph{Sternenkarte} (\coin[3]): Jedes Mal, wenn du mischst, darfst du die zu mischenden Karten durchsehen und eine auswählen, die du nach dem Mischen der restlichen Karten oben auf die gemischten Karten legst.}
\end{tikzpicture}
\hspace{-0.6cm}
\begin{tikzpicture}
	\card
	\cardstrip
	\cardbanner{banner/lightred.png}
	\cardicon{icons/coin.png}
	\cardprice{\scriptsize{3-8}}
	\cardtitle{Projekte (3/8)}
	\cardcontent{\emph{Erkundung} (\coin[4]): Diese Fähigkeit kannst du nur nutzen, wenn du in deiner Kaufphase keine Karte gekauft hast. \emph{Ereignisse} (aus \emph{Abenteuer} und \emph{Empires}) und \emph{Projekte} gelten nicht als Karten im spieltechnischen Sinn. Entsprechend kannst du ein \emph{Ereignis} oder ein \emph{Projekt} kaufen bzw. eine Karte auf eine andere Art und Weise nehmen und kannst die Fähigkeit trotzdem nutzen. Wenn du z.~B. in deinem Zug nur eine \emph{ERKUNDUNG} kaufst, bekommst du +1 Taler und +1 Dorfbewohner in diesem Zug. Die Fähigkeit der \emph{ERKUNDUNG} kannst du sofort in dem Zug nutzen, in dem du die \emph{ERKUNDUNG} kaufst - falls du nur die \emph{ERKUNDUNG} und ggf. andere \emph{Ereignisse} oder \emph{Projekte} gekauft hast und keine Karte.
	
	\medskip
	
	\emph{Finsterer Plan} (\coin[4]): Wenn du den \emph{FINSTEREN PLAN} kaufst, legst du einen deiner Holzsteine darauf und zu Beginn deines nächsten Zuges einen Marker neben den Holzstein. Haltet die Marker-Stapel der einzelnen Spieler auf diesem Projekt gut voneinander getrennt.
	\linebreak
	Zu Beginn jedes deiner Züge legst du einen Marker hierher (einen unbenutzten, nicht einen vom Tableau) oder legst alle deiner Marker von hier zurück und ziehst die entsprechende Anzahl Karten.
	
	\medskip
	
	\emph{Kleiner Markt} (\coin[4]): Du hast +1 Kauf in jedem deiner Züge.}
\end{tikzpicture}
\hspace{-0.6cm}
\begin{tikzpicture}
	\card
	\cardstrip
	\cardbanner{banner/lightred.png}
	\cardicon{icons/coin.png}
	\cardprice{\scriptsize{3-8}}
	\cardtitle{Projekte (4/8)}
	\cardcontent{\emph{Speicher} (\coin[4]): Zuerst legst du die \emph{KUPFER} ab, dann ziehst du die gleiche Anzahl Karten nach. Wenn du durch das Ziehen mischen musst, mischst du die \emph{KUPFER} mit ein.
	
	\medskip
	
	\emph{Akademie} (\coin[5]): Es ist egal, auf welche Art und Weise du die Aktionskarte nimmst (also ob du sie kaufst, aufgrund einer Kartenanweisung nehmen kannst o. ä.).
	
	\medskip
	
	\emph{Flotte} (\coin[5]): Wenn mindestens ein Spieler die \emph{FLOTTE} gekauft hat, gibt es eine zusätzliche Runde, nachdem das Spiel normalerweise enden würde. Nur Spieler mit der \emph{FLOTTE} führen in der zusätzlichen Runde einen \enquote{normalen} Zug aus. Ansonsten wird gehandelt, als wäre da Spiel um eine normale Runde verlängert. Spieler ohne die \emph{FLOTTE} führen nur Extrazüge (\emph{AUSSENPOSTEN}, \emph{MISSION}, \emph{BESESSENHEIT}) aus. Nach dem letzten durch die \emph{FLOTTE} ermöglichten, normalen Zug werden keine Extrazüge mehr ausgeführt.
	\linebreak
	Die zusätzliche Runde beginnt der Spieler, der im Uhrzeigersinn folgend am nächsten zu dem Spieler sitzt, der den letzten Zug gemacht hat.
	\linebreak
	Bis die letzte Runde beendet ist, darf kein Spieler (auch nicht die Spieler ohne die \emph{FLOTTE}) seine Karten durchsehen oder bereits zählen. 
	\linebreak
	Sind die Spielende-Bedingungen nach der zusätzlichen Runde nicht mehr erfüllt, ist das Spiel trotzdem beendet.}
\end{tikzpicture}
\hspace{-0.6cm}
\begin{tikzpicture}
	\card
	\cardstrip
	\cardbanner{banner/lightred.png}
	\cardicon{icons/coin.png}
	\cardprice{\scriptsize{3-8}}
	\cardtitle{Projekte (5/8)}
	\cardcontent{\emph{Gildenhaus} (\coin[5]): es ist egal, auf welche Art und Weise du die Geldkarte nimmst (also ob du sie kaufst, aufgrund einer Kartenanweisung nehmen kannst o. ä.).
	
	\medskip 
	
	\emph{Kapitalismus} (\coin[5]): Nur Aktionskarten mit einer +\coin-Anweisung in diesem Kartentext sind von diesem Projekt betroffen (z.~B. verwandelt der \emph{KAPITALISMUS} einen \emph{FORTSCHRITT} in eine Geldkarte, hat aber keinen Einfluss auf einen \emph{ERFINDER}). Diese gelten ab sofort während deiner Züge in allen Belangen als kombinierte Geld-/Aktionskarten und können entweder wie gewohnt als Aktionskarte oder auch als Geldkarte ausgespielt werden. In der Kaufphase kannst du beliebig viele Aktionskarten ausspielen, die durch \emph{KAPITALISMUS} zu Geldkarten werden. Auch wenn du sie als Geldkarte ausspielst, müssen trotzdem alle Anweisungen dieser Aktionskarte ausgeführt werden.
	\linebreak
	Dein \emph{KAPITALISMUS} wirkt sich nur während deiner Züge aus, betrifft aber alle Karten überall, also auch die Karten anderer Spieler. Ebenso wirkt sich auch die \emph{KAPITALISMUS}-Fähigkeit eines anderen Spielers während dessen Zügen auf deine Aktionskarten mit +\coin-Karten aus.
	\linebreak
	Immer wenn du eine kombinierte Aktions-/Geldkarte ausspielst, ist sie sowohl Aktion als auch Geld, unabhängig von der Phase, in der du dich gerade befindest. Dass du +1 Aktion in deiner Kaufphase bekommst, erlaubt dir nicht, dass du in der Phase andere Aktionskarten ausspielen darfst.}
\end{tikzpicture}
\hspace{-0.6cm}
\begin{tikzpicture}
	\card
	\cardstrip
	\cardbanner{banner/lightred.png}
	\cardicon{icons/coin.png}
	\cardprice{\scriptsize{3-8}}
	\cardtitle{Projekte (6/8)}
	\cardcontent{\emph{Piazza} (\coin[5]): Sobald du diese Fähigkeit hast, musst du sie zu Beginn jedes deiner Züge einsetzen. Wenn die aufgedeckte Karte keine Aktion ist, lege sie auf deinen Nachziehstapel zurück.
	
	\medskip
	
	\emph{Straßennetz} (\coin[5]): Jedes Mal, wenn ein anderer Spieler eine Punktekarte nimmt (auch während deines Zuges), entweder gekauft oder auf andere Art und Weise, ziehst du eine Karte von deinem Nachziehstapel.
	
	\medskip
	
	\emph{Fruchtwechsel} (\coin[6]): Wenn du durch das Ziehen veranlasst wirst zu mischen, mischst du die abgelegte Punktekarte mit ein.}
\end{tikzpicture}
\hspace{-0.6cm}
\begin{tikzpicture}
	\card
	\cardstrip
	\cardbanner{banner/lightred.png}
	\cardicon{icons/coin.png}
	\cardprice{\scriptsize{3-8}}
	\cardtitle{Projekte (7/8)}
	\cardcontent{\emph{Innovation} (\coin[6]): Dies ist optional betrifft aber nur die erste Aktionskarte, die du pro Zug genommen hast. Unabhängig davon, ob du die \emph{INNOVATION} bei dieser ersten Karte benutzt, darfst du sie für folgende Karten, die du in dieser Kaufphase nimmst, nicht mehr benutzen. Dies gilt für Karten, die du kaufst oder auf andere Art und Weise nimmst. Wenn du die erste Aktionskarte deines Zuges in der Kaufphase nimmst, bedeutet das, dass du diese Karte ausspielen darfst, obwohl du es in deiner Kaufphase tust. Auch wenn die Karte dir +x Aktionen gibt, darfst du diese zusätzlichen Aktionen trotzdem in deiner Kaufphase nicht verwenden. Wenn du durch sie Karten ziehen darfst und sich unter den gezogenen Karten Geldkarten befinden, darfst du diese nun ausspielen, wenn du in der Kaufphase noch nichts gekauft hast (außer mit dem \emph{SCHWARZMARKT}, der mit dem \emph{KAPITALISMUS} als Geldkarte gespielt wurde).
	
	\medskip
	
	\emph{Kaserne} (\coin[6]): Du hast +1 Aktion in jedem deiner Züge.}
\end{tikzpicture}
\hspace{-0.6cm}
\begin{tikzpicture}
	\card
	\cardstrip
	\cardbanner{banner/lightred.png}
	\cardicon{icons/coin.png}
	\cardprice{\scriptsize{3-8}}
	\cardtitle{Projekte (8/8)}
	\cardcontent{\emph{Kanal} (\coin[7]): Während der Züge aller Spieler, die dieses \emph{Projekt} gekauft haben, kosten alle Karten \coin[1] weniger, aber nicht weniger als \coin[0]. Dies betrifft alle Karten, einschließlich zur Seite gelegter Karten, aller Karten im Vorrat, in den Händen sowie den Nachzieh- und Ablagestapeln aller Spieler. Es betrifft auch Karten auf allen Sonderstapeln (nicht im Vorrat), z.~B. die Erscheinungen (aus \emph{Nocturne}) und die Reisenden (aus \emph{Abenteuer}), sowie das Schwarzmarktdeck.
	\linebreak
	Wenn du den \emph{KANAL} hast und z.~B. einen \emph{SCHWARZEN MEISTER} ausspielst, legen die übrigen Spieler jeweils eine Karte ab, die mindestens \coin[2] kostet. Dies darf aber kein \emph{ANWESEN} sein, denn ein \emph{ANWESEN} kostet während deiner Züge nur \qquad \coin[1]. Kosten von \emph{Projekten} und \emph{Ereignissen} (aus \emph{Abenteuer} und \emph{Empires}) werden durch den \emph{KANAL} nicht reduziert, weil es sich nicht um Karten im regeltechnischen Sinne handelt.
	
	\smallskip 
	
	\emph{Zitadelle} (\coin[8]): Sobald due diese Fähigkeit hast, musst du sie in jedem deiner Züge einsetzen. Dies kann eine Aktion betreffen, die außerhalb der Aktionsphase gespielt wurde, wenn die deine erste ausgespielte Aktionskarte in diesem Zug ist. Wenn du z.~B. auch den \emph{KAPITALISMUS} hast, könntest du einen \emph{FAHNENTRÄGER} in deiner Kaufphase als erste Aktionskarte in diesem Zug ausspielen und erhältst \coin[2]. Dann spielst du den \emph{FAHNENTRÄGER} zum zweiten Mal aus und erhältst erneut \qquad \coin[2]. Ist die erste ausgespielte Aktionskarte eine Dauerkarte, musst du dich in deinem nächsten Zug selbstständig an die \enquote{Verdoppelung} erinnern.}
\end{tikzpicture}
\hspace{-0.6cm}
\begin{tikzpicture}
	\card
	\cardstrip
	\cardbanner{banner/gold.png}
	\cardtitle{Artefakte\qquad}
	\cardcontent{\tiny{\emph{Horn (GRENZPOSTEN):} Solange du das \emph{HORN} hast, darfst du einmal pro eigenem Zug beim Ablegen eines \emph{GRENZPOSTENS} aus dem Spiel diesen auf deinen Nachziehstapel legen anstatt auf deinen Ablagestapel. Das \emph{HORN} wirkt schon in dem Zug, in dem du es bekommst.

	\medskip

	\emph{Laterne (GRENZPOSTEN):} Wenn du die \emph{LATERNE} hast, musst du in deinen Zügen bei jedem ausgespielten Grenzposten 3 Karten vom Nachziehstapel aufdecken und 2 ablegen. In diesem Fall müssen alle 3 Karten Aktionskarten sein, damit du das \emph{HORN} erhältst. Die \emph{LATERNE} wirkt schon in dem Zug, in dem du sie bekommst.
	
	\medskip
	
	\emph{Fahne (FAHNENTRÄGER):} Die \emph{FAHNE} veranlasst dich, eine weitere Karte zu ziehen, wenn du in der Aufräumphase deine Karten vom Nachziehstapel ziehst. Dies gilt auch, wenn auf deiner Hand normalerweise weniger oder mehr als 5 Karten wären; wenn du zum Beispiel einen \emph{AUSSENPOSTEN} (aus \emph{Seaside}) ausgespielt hast, würdest du statt 3 Karten für den \emph{AUSSENPOSTEN}-Zug 4 Karten ziehen.
	
	\medskip
	
	\emph{Schatzkiste (FREIBEUTERIN):} Solang du die \emph{SCHATZKISTE} hast, nimmst du in jedem eigenen Zug zu Beginn deiner Kaufphase ein \emph{GOLD}, einschließlich des Zuges, in dem du sie erhältst (aber nicht, wenn du sie erst nach Beginn deiner Kaufphase erhältst).
	
	\medskip
	
	\emph{Schlüssel (SCHATZMEISTERIN):} Solange du den \emph{SCHLÜSSEL} hast, erhältst du zu Beginn des Zuges +\coin[1], also noch nicht in dem Zug, in dem du ihn erhältst.}}
\end{tikzpicture}
\hspace{-0.6cm}
\begin{tikzpicture}
	\card
	\cardstrip
	\cardbanner{banner/white.png}
	\cardtitle{\scriptsize{Spielvorbereitung (1/2)}\qquad}
	\cardcontent{Zum Spielen mit \emph{DOMINION Renaissance} benötigt ihr ein \emph{DOMINION-Basisspiel} oder das \emph{Basiskarten-Set}. Legt alle Basiskarten (\emph{KUPFER}, \emph{SILBER}, \emph{GOLD} (+ ggf. \emph{PLATIN}), \emph{ANWESEN}, \emph{HERZOGTUM}, \emph{PROVINZ} (+ ggf. \emph{KOLONIE}) sowie die \emph{FLÜCHE} und die Müllkarte (bzw. das Mülltableau aus der \emph{DOMINION 2. Edition})) wie gewohnt als Teil des Vorrats in die Tischmitte.
	
	\medskip

	\underline{\emph{Taler/Dorfbewohner}}
	\linebreak
	Wenn ihr mindestens eine Karte verwendet, die sich auf Taler oder Dorfbewohner bezieht, erhält jeder Spieler ein Taler-/Dorfbewohner-Tableau. Legt dann die 35 Marker als Haufen neben dem Vorrat bereit.
	\linebreak
	Die Taler-/Dorfbewohner-Tableaus haben als Rahmen die gleichen 6 Farben wie die Holzsteine. Jeder Spieler nimmt sich ein Tableau und die beiden Holzstein in der gleichen Farbe.

	\medskip
	
	\underline{\emph{Artefakte}}
	\linebreak
	Wenn ihr \emph{GRENZPOSTEN}, \emph{FAHNENTRÄGER}, \emph{FREIBEUTERIN} oder \emph{SCHATZMEISTERIN} verwendet, sucht die \emph{Artefakte} heraus, auf die sie sich beziehen, und legt sie neben dem Vorrat bereit. \emph{Artefakte} gehören nicht zum Vorrat.}
\end{tikzpicture}
\hspace{-0.6cm}
\begin{tikzpicture}
	\card
	\cardstrip
	\cardbanner{banner/white.png}
	\cardtitle{\scriptsize{Spielvorbereitung (2/2)}\qquad}
	\cardcontent{\underline{\emph{Projekte}}
		\linebreak
		Zusätzlich zu den \emph{KÖNIGREICH}-Karten gibt es \emph{Projekte}, die während des Spiels gekauft werden können und deren Effekt bis zum Spielende für den Spieler in Kraft bleiben, der sie gekauft hat. Wir empfehlen, pro Spiel maximal insgesamt 2 \emph{Projekte}, \emph{Landmarken} (aus \emph{Empires}) und/oder \emph{Ereignisse} (aus \emph{Abenteuer}) zu verwenden. Verwendet ihr im Spiel mindestens 1 \emph{Projekt}, erhält jeder Spieler zu Beginn 2 \emph{Holzsteine} derselben Farbe und legt sie vor sich ab. Diese Holzsteine legt ihr im Spielverlauf jeweils auf Projekte, die ihr gekauft habt.
		\linebreak
		Zieht die \emph{Projekte} aus einem Stapel (dieser kann auch die \emph{Landmarken} (aus \emph{Empires}) und/oder \emph{Ereignisse} (aus \emph{Abenteuer}) enthalten) oder mischt sie (trotz ihrer unterschiedlichen Rückseite) in die Platzhalterkarten ein. Deckt ihr ein \emph{Projekt} auf, legt es neben den Vorrat bereit. \emph{Projekte} gehören nicht zum Vorrat. Deckt so lange Karten auf, bis ihr 10 \emph{Königreich}-Karten aufgedeckt habt. Wenn ihr mehr als 2 \emph{Projekte} (empfohlen) aufgedeckt habt, legt die überzähligen \emph{Projekte} in die Schachtel zurück -- sie kommen in diesem Spiel nicht zum Einsatz. \emph{Projekte} können nicht als Bannstapel für die \emph{JUNGE HEXE} (aus \emph{Reiche Ernte}) genutzt werden.}
\end{tikzpicture}
\hspace{-0.6cm}
\begin{tikzpicture}
	\card
	\cardstrip
	\cardbanner{banner/white.png}
	\cardtitle{\scriptsize{Neue Regeln (1/5)}\qquad}
	\cardcontent{\emph{Es gelten die Basisspielregeln mit folgenden Änderungen:}

	\medskip

	\emph{Taler \& Dorfbewohner:} In \emph{Renaissance} gibt es Tableaus, auf denen du \coin und Aktionen für später aufsparen kannst. Wenn ihr mindestens eine Karte verwendet, die sich auf Taler oder Dorfbewohner bezieht, erhält jeder Spieler ein Taler-/Dorfbewohner-Tableau. Legt dann die 35 Marker als Haufen neben dem Vorrat bereit.
	\begin{itemize}
		\item Karten mit \enquote{+\emph{1 Taler}}: Lege 1 Marker auf die Taler-Seite deines Tableaus. In der Kaufphase kannst du den Marker für +\coin[1] wieder zurücklegen, bevor du etwas kaufst.\\
		\item Karten mit \enquote{+\emph{1 Dorfbewohner}}: Lege 1 Marker auf die Dorfbewohner-Seite deines Tableaus. In der Aktionsphase kannst du den Marker für +1 Aktion wieder zurücklegen.
	\end{itemize}
	Du kannst beliebig viele Marker gleichzeitig von deinem Tableau zurücklegen. Jeder gibt zusätzlich +\coin[1] bzw. +1 Aktion.}
\end{tikzpicture}
\hspace{-0.6cm}
\begin{tikzpicture}
	\card
	\cardstrip
	\cardbanner{banner/white.png}
	\cardtitle{\scriptsize{Neue Regeln (2/5)}\qquad}
	\cardcontent{
		Die Anzahl der Marker im Vorrat gilt nicht als begrenzt: Sollten nicht ausreichend Marker vorhanden sein, verwendet einen beliebigen Einsatz. Es handelt sich um die gleichen Marker wie in \emph{Seaside}, \emph{Blütezeit} und \emph{Die Gilden}. Sie können beliebig gemischt werden.
		\smallskip
	
		Marker, die auf andere Art und Weise verwendet werden, wie z.~B. auf dem Piratenschiff-Tableau in \emph{Seaside}, können nicht für +\coin[1] oder +1 Aktion zurückgelegt werden. Dies ist nur mit Markern auf dem Taler/-Dorfbewohner-Tableau möglich.
	
		\smallskip
	
		Marker werden immer vom Marker-Haufen genommen und dorthin zurückgelegt, niemals von anderen Tableaus oder Mitspielern.
	
		\smallskip
	
		In \emph{Die Gilden} gibt es auch ein Taler-Tableau und dort werden die Marker auf die gleiche Art und Weise verwendet. In den ersten Auflagen von \emph{Die Gilden} stand stattdessen \enquote{Nimm eine Münze}; in den späteren Auflagen wurde das zu \enquote{+1 Taler}, und so sollte es auch gespielt werden.}
\end{tikzpicture}
\hspace{-0.6cm}
\begin{tikzpicture}
	\card
	\cardstrip
	\cardbanner{banner/white.png}
	\cardtitle{\scriptsize{Neue Regeln (3/5)}\qquad}
	\cardcontent{
		\emph{Projekte:} \emph{Projekte} sind Fähigkeiten, die Spieler kaufen können. Es gibt insgesamt 20 \emph{Projekte}.
		\linebreak
		Zu Spielbeginn wird von den Spielern entschieden, mit welchen \emph{Projekten} gespielt wird. Wir empfehlen, pro Spiel maximal 2 \emph{Projekte} zu verwenden. Die \emph{Projekte} werden neben den Vorrat bereitgelegt, gehören aber nicht zum Vorrat. Außerdem erhält jeder Spieler zu Beginn 2 Holzsteine derselben Farbe und legt sie vor sich ab. Diese Holzsteine legt ihr im Spielverlauf jeweils auf die \emph{Projekte}, die ihr gekauft habt.
		
		\smallskip
		
		Du kannst \emph{Projekte} in deiner Kaufphase kaufen. Dies verbraucht 1 Kauf und kostet einen bestimmten Betrag an \coin, der in der linken oberen Ecke des jeweiligen \emph{Projektes} steht. Wenn du ein \emph{Projekt} kaufst, lege einen deiner Holzsteine darauf. Für den Rest des Spiels hast du diese Fähigkeit. Wenn z.~B. der \emph{KLEINE MARKT} verwendet wird, kannst du \coin[4] bezahlen und einen deiner Holzsteine auf das \emph{Projekt} legen. Dann hast du für den Rest des Spiels +1 Kauf in jedem deiner Züge.
		
		\smallskip
		
		Wir empfehlen mit maximal 2 \emph{Projekten} zu spielen. Sind mehr im Spiel, hast du trotzdem nur 2 Steine zum Platzieren. Du darfst nicht beide Steine auf das gleiche \emph{Projekt} legen und keinen Stein von einem \emph{Projekt} zurücklegen, nachdem du ihn dort hingelegt hast.}
\end{tikzpicture}
\hspace{-0.6cm}
\begin{tikzpicture}
	\card
	\cardstrip
	\cardbanner{banner/white.png}
	\cardtitle{\scriptsize{Neue Regeln (4/5)}\qquad}
	\cardcontent{
		Beliebig viele Spieler können die gleiche Fähigkeit eines \emph{Projektes} zur gleichen Zeit haben.
		\linebreak
		Das Bezahlen eines \emph{Projektes} gilt nicht als \enquote{eine Karte kaufen}; es wird nicht billiger durch Karten wie \emph{ERFINDER} und spielt für Karten wie \emph{ERKUNDUNG} keine Rolle. Du kannst kein \emph{Projekt} kaufen, wenn du Schilden \hex hast (aus \emph{Empires}).
		
		\medskip
		
		\emph{Artefakte:} \emph{Artefakte} sind Fähigkeiten, die Spieler auf eine bestimmte Art und Weise durch eine einzelne Königliche-Karte erhalten können. Wenn ihr \emph{GRENZPOSTEN}, \emph{FAHNENTRÄGER}, \emph{FREIBEUTERIN} oder \emph{SCHATZMEISTERIN} verwendet, sucht die \emph{Artefakte} heraus, auf die sie sich beziehen, und legt sie neben dem Vorrat bereit.
		\linebreak
		Wenn du ein \emph{Artefakt} erhältst und ein anderer Spieler hat es, erhältst du es von ihm. Solange du ein \emph{Artefakt} hast, steht dir dessen Fähigkeit zur Verfügung, und du verlierst diese Fähigkeit, wenn ein anderer Spieler das \emph{Artefakt} erhält. Wenn du zum Beispiel eine \emph{SCHATZMEISTERIN} ausspielst, darfst du dich dafür entscheiden, den \emph{SCHLÜSSEL} zu erhalten, der dir +\coin[1] am Anfang von jedem deiner Züge gibt.}
\end{tikzpicture}
\hspace{-0.6cm}
\begin{tikzpicture}
	\card
	\cardstrip
	\cardbanner{banner/white.png}
	\cardtitle{\scriptsize{Neue Regeln (5/5)}\qquad}
	\cardcontent{
		\emph{Die Dauerkarten:} In \emph{Renaissance} gibt es zwei Dauerkarten (wie schon in \emph{Seaside} und \emph{Abenteuer}). Die orangefarbenen Dauerkarten beinhalten Anweisungen, die in späteren Zügen umgesetzt werden. Sie werden nicht grundsätzlich in der Aufräumphase des Zuges abgelegt, in dem sie ausgespielt wurden, sondern bleiben bis zur Aufräumphase des Zuges, in dem die letzte Anweisung ausgeführt wird, im Spiel. Wir eine Dauerkarte mehrfach ausgespielt (z.~B. durch den \emph{THRONSAAL} aus dem \emph{Basisspiel}), bleibt die verursachende Karte ebenfalls so lange im Spiel, bis die Dauerkarte abgelegt wird. Um anzuzeigen, dass eine Dauerkarte in der aktuellen Aufräumphase noch nicht abgelegt wird, wird sie in eine eigene Reihe oberhalb der restlichen ausgespielten Karten gelegt. Bei mehreren im Spiel befindlichen Dauerkarten darf der Spieler die Reihenfolge selbst bestimmen, in der er sie abhandelt.}
\end{tikzpicture}
\hspace{-0.6cm}
\begin{tikzpicture}
	\card
	\cardstrip
	\cardbanner{banner/white.png}
	\cardtitle{\scriptsize{Anweisungen (1/2)}\qquad}
	\cardcontent{\emph{Nehmen/Erhalten}: \emph{Projekte} und \emph{Artefakte} in \emph{Renaissance} sind keine \enquote{Karten} im regeltechnischen Sinn und werden erhalten, niemals \enquote{genommen}, d.~h. sie werden niemals dem Kartensatz eines Spielers hinzugefügt.
	
	\begin{itemize}
		\item[\rightarrow] \emph{Nehmen:} Karten, die ein Spieler durch Kauf oder eine Anweisung auf einer anderen Karte nimmt, werden vom Spieler physisch an sich genommen und werden dadurch dem Kartensatz hinzugefügt.\\
		\item[\rightarrow] \emph{Erhalten:} Erhält ein Spieler ein \emph{Artefakt}, legt er es für eine gewisse Zeit an den angewiesenen Ort. Kauf er ein \emph{Projekt}, legt er einen eigenen Holzstein auf das entsprechende Projekt, womit er dessen Effekt erhält. \emph{Artefakte} und \emph{Projekte} gehören ihm aber nicht und werden auch nicht berücksichtigt, wenn z.~B. die \emph{GÄRTEN} die Anzahl der Karten eines Spielers zählen.\\
		\item[\rightarrow] \emph{Zur Seite legen:} Karten, die durch eine Anweisung zur Seite gelegt werden, befinden sich nicht im Spiel.\\ 
		\item[\rightarrow] \emph{Im Spiel:} Im Spiel befinden sich die in diesem Zug ausgespielten Karten, Dauerkarten aus vorherigen Zügen sowie vom Wirtshaustableau aufgerufene Karten (aus \emph{Abenteuer}). Nicht im Spiel befinden sich bereits entsorgte Karten sowie zur Seite gelegte Karten.
	\end{itemize}}
\end{tikzpicture}
\hspace{-0.6cm}
\begin{tikzpicture}
	\card
	\cardstrip
	\cardbanner{banner/white.png}
	\cardtitle{\scriptsize{Anweisungen (2/2)}\qquad}
	\cardcontent{\begin{itemize} 
		\item[\rightarrow] \emph{In den Vorrat zurücklegen:} Die Karte wird sofort in den Vorrat auf den entsprechenden Stapel zurückgelegt.\\
		\item[\rightarrow] \emph{Diese Karte:} Enthält eine Karte eine Anweisung, die sich auf \enquote{diese Karte} bezieht, ist \emph{immer} die Karte gemeint, auf der die Anweisung steht, niemals eine andere Karte, auf die innerhalb der Anweisung Bezug genommen wird.\\
		Beispiel: Der Kartentext der \emph{FORSCHERIN} besagt: \enquote{Pro \coin[1], das sie kostet, lege eine Karte von deinem Nachziehstapel mit der Bildseite nach unten zur Seite (auf diese Karte).} Dies bedeutet, dass der Spieler die Karte von seinem Nachziehstapel auf die zur Seite gelegte \emph{FORSCHERIN} legen muss.\\ 
		\item[\rightarrow] \emph{Jene Karte:} Enthält eine Karte eine Anweisung, die sich auf \enquote{jene Karte} bezieht, sind immer die Karten gemeint, auf die auf einer Karte (\enquote{dieser Karte}) Bezug genommen wird -- es ist niemals die gerade genutzte Karte gemeint.\\ 
		\item[\rightarrow] \emph{Aufdecken:} Der Spieler deckt die Karte(n) auf, zeigt sie allen Mitspielern und legt sie dorthin zurück, wo er sie herhat.\\ 
		\item[\rightarrow] \emph{Wähle eins:} Der Spieler muss genau eine der aufgelisteten Anweisungen auswählen und sie -- soweit möglich -- ausführen.
	\end{itemize}}
\end{tikzpicture}
\hspace{-0.6cm}
\begin{tikzpicture}
	\card
	\cardstrip
	\cardbanner{banner/white.png}
	\cardtitle{\scriptsize{Empfohlene 10er Sätze\qquad}}
	\cardcontent{\emph{Ouvertüre} (+ \underline{Projekte}):\\
	\underline{Kleiner Markt}, Alte Hexe, Diener, Erfinder, Experiment, Fahnenträger, Fortschritt, Schatzmeisterin, Schauspieltruppe, Seher, Versteck

	\medskip

	\emph{Präludium} (+ \underline{Projekte}):\\
	\underline{Sternenkarte}, \underline{Zitadelle}, Anwerber, Bergdorf, Bildhauerin, Frachtschiff, Freibeuterin, Goldmünze, Grenzposten, Priester, Schwarzer Meister, Seidenhändlerin

	\medskip

	\emph{Her mit den Dorfbewohnern} (+ \textit{Basisspiel 2. Edition} + \underline{Projekte}):\\
	\underline{Straßennetz}, Anwerber, Frachtschiff, Schauspieltruppe, Schatzmeisterin, Seher, \textit{Markt}, \textit{Händlerin}, \textit{Mine}, \textit{Schmiede}, \textit{Vasall}

	\medskip

	\emph{Holt die Fahne} (+ \textit{Basisspiel 2. Edition} + \underline{Projekte}):\\
	\underline{Festzug}, \underline{Kaserne}, Diener, Fahnenträger, Freibeuterin, Gelehrter, Schwarzer Meister, \textit{Vorbotin}, \textit{Jahrmarkt}, \textit{Keller}, \textit{Umbau}, \textit{Werkstatt}}
\end{tikzpicture}
\hspace{-0.6cm}
\begin{tikzpicture}
	\card
	\cardstrip
	\cardbanner{banner/white.png}
	\cardtitle{\scriptsize{Empfohlene 10er Sätze\qquad}}
	\cardcontent{\emph{Freihandel} (+ \textit{Seaside} + \underline{Projekte}):\\
	\underline{Innovation}, Diener, Forscherin, Frachtschiff, Gewürze, Schauspieltruppe, \textit{Außenposten}, \textit{Embargo}, \textit{Insel}, \textit{Schmuggler}, \textit{Werft}

	\medskip

	\emph{Schatzsuche} (+ \textit{Seaside} + \underline{Projekte}):\\
	\underline{Fruchtwechsel}, \underline{Speicher}, Bildhauerin, Erfinder, Fahnenträger, Freibeuterin, Grenzposten, \textit{Karawane}, \textit{Eingeborenendorf}, \textit{Müllverwerter}, \textit{Taktiker}, \textit{Schatzkarte}

	\medskip

	\emph{Traum-Träumer} (+ \textit{Blütezeit (mit Platin und Kolonie)} + \underline{Projekte}):\\
	\underline{Akademie}, Alte Hexe, Frachtschiff, Gelehrter, Priester, Zepter, \textit{Arbeiterdorf}, \textit{Ausbau}, \textit{Denkmal}, \textit{Gewölbe}, \textit{Wachturm}

	\medskip

	\emph{Geld regiert die Welt} (+ \textit{Blütezeit (mit Platin und Kolonie)} + \underline{Projekte}):\\
	\underline{Kapitalismus}, \underline{Zitadelle}, Forscherin, Patron, Schatzmeisterin, Schwarzer Meister, Versteck, \textit{Bank}, \textit{Gesindel}, \textit{Großer Markt}, \textit{Lohn}, \textit{Stadt}}
\end{tikzpicture}
\hspace{-0.6cm}
\begin{tikzpicture}
	\card
	\cardstrip
	\cardbanner{banner/white.png}
	\cardtitle{\scriptsize{Empfohlene 10er Sätze\qquad}}
	\cardcontent{\emph{Sterndeutung} (+ \textit{Dark Ages (mit Unterschlüpfen)} + \underline{Projekte}):\\
	\underline{Sternenkarte}, Freibeuterin, Grenzposten, Patron, Seher, Seidenhändlerin, \textit{Barde}, \textit{Eremit}, \textit{Medium}, \textit{Prozession}, \textit{Weiser}

	\medskip

	\emph{Kanalratten} (+ \textit{Dark Ages (mit Unterschlüpfen)} + \underline{Projekte}):\\
	\underline{Abwassertunnel}, \underline{Fruchtwechsel}, Bergdorf, Diener, Fahnenträger, Forscherin, Fortschritt, \textit{Falschgeld}, \textit{Grabräuber}, \textit{Graf}, \textit{Kultist}, \textit{Ratte}

	\medskip

	\emph{Entwicklung}  (+ \textit{Abenteuer} + \underline{Projekte} + \underline{\textit{Ereignisse}}):\\
	\underline{Piazza}, \underline{\textit{Training}}, Anwerber, Experiment, Fortschritt, Seher, Seidenhändlerin, \textit{Gefolgsmann}, \textit{Sumpfhexe}, \textit{Transformation}, \textit{Wildhüter}, \textit{Zerstörung}

	\medskip

	\emph{Es war einmal}  (+ \textit{Abenteuer} + \underline{Projekte} + \underline{\textit{Ereignisse}}):\\
	\underline{Innovation}, \underline{\textit{Überfahrt}}, Bildhauerin, Diener, Gewürze, Priester, Schauspieltruppe, \textit{Duplikat}, \textit{Ferne Lande}, \textit{Geisterwald}, \textit{Geschichtenerzähler}, \textit{Königliche Kutsche}}
\end{tikzpicture}
\hspace{-0.6cm}
\begin{tikzpicture}
	\card
	\cardstrip
	\cardbanner{banner/white.png}
	\cardtitle{\scriptsize{Empfohlene 10er Sätze\qquad}}
	\cardcontent{\emph{Die Stadt erkunden} (+ \textit{Empires} + \underline{Projekte} + \underline{\textit{Ereignisse}}):\\
	\underline{Erkundung}, \underline{\textit{Schlachtfeld}}, Bergdorf, Bildhauerin, Experiment, Frachtschiff, Priester, \textit{Bauernmarkt}, \textit{Gärtnerin}, \textit{Opfer}, \textit{Stadtviertel}, \textit{Wilde Jagd}

	\medskip

	\emph{Durch die Kanalisation} (+ \textit{Empires} + \underline{Projekte} + \underline{\textit{Ereignisse}}):\\
	\underline{Abwassertunnel}, \underline{\textit{Ritual}}, Alte Hexe, Fahnenträger, Fortschritt, Schauspieltruppe, Zepter, \textit{Gladiator}, \textit{Patrizier}, \textit{Villa}, \textit{Wagenrennen}, \textit{Zauberin}

	\medskip

	\emph{Zum Monster werden} (+ \textit{Nocturne} + \underline{Projekte}):\\
	\underline{Erkundung}, Alte Hexe, Bergdorf, Experiment, Forscherin, Gewürze, \textit{Attentäter}, \textit{Kloster}, \textit{Schäferin}, \textit{Teufelswerkstatt}, \textit{Tragischer Held}

	\medskip

	\emph{Wahre Gläubige} (+ \textit{Nocturne} + \underline{Projekte}):\\
	\underline{Kathedrale}, \underline{Piazza}, Bildhauerin, Frachtschiff, Gelehrter, Grenzposten, Schwarzer Meister, \textit{Geheime Höhle}, \textit{Getreuer Hund}, \textit{Heiliger Hain}, \textit{Krypta}, \textit{Seliges Dorf}}
\end{tikzpicture}
\hspace{0.6cm}
