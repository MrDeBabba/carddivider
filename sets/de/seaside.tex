% Basic settings for this card set
\renewcommand{\cardcolor}{seaside}
\renewcommand{\cardextension}{Erweiterung I}
\renewcommand{\cardextensiontitle}{Seaside}

\clearpage
\newpage
\section{\cardextension \ - \cardextensiontitle}

\begin{tikzpicture}
	\card
	\cardstrip
	\cardbanner{banner/white.png}
	\cardicon{banner/coin.png}
	\cardprice{2}
	\cardtitle{\scriptsize{Eingeborenendorf}}
	\cardcontent{Sobald du dein erstes Eingeborenendorf nimmst oder kaufst, nimmst du auch ein Tableau Eingeborenendorf zu dir. Wenn du das Eingeborenendorf ausspielst, musst du eine der beiden folgenden Möglichkeiten wählen: Entweder du siehst die oberste Karte von deinem Nachziehstapel an und legst sie dann verdeckt zur Seite auf das Tableau Eingeborenendorf. Oder du nimmst alle zur Seite gelegten Karten vom Tableau Eingeborenendorf auf die Hand. Du kannst jede der beiden Möglichkeiten wählen, auch wenn du sie nicht ausführen kannst. Du darfst die zur Seite gelegten Karten jederzeit ansehen. Das ausgespielte Eingeborenendorf selbst legst du nicht zur Seite, sondern legst es in der Aufräumphase ab. Bei Spielende nimmst du alle Karten, die noch auf dem Tableau Eingeborenendorf liegen, zu deinen übrigen Karten.}
\end{tikzpicture}
\hspace{-1cm}
\begin{tikzpicture}
	\card
	\cardstrip
	\cardbanner{banner/white.png}
	\cardicon{banner/coin.png}
	\cardprice{2}
	\cardtitle{Embargo}
	\cardcontent{Du kannst jeden Stapel des Vorrats wählen. Wenn mehrere Embargo-Marker auf einem Stapel liegen, musst du dir für jeden Marker eine Fluchkarte nehmen, wenn du eine Karte von diesem Stapel kaufst. Du nimmst dir nur Flüche, wenn du dir eine Karte von diesem Stapel kaufst. Wenn du dir auf eine andere Weise eine Karte von diesem Stapel nimmst (z. B. durch die Schmuggler), nimmst du dir keinen Fluch. Wenn du das Embargo auf einen Thronsaal spielst, legst du 2 Embargo-Marker. Die beiden Embargo-Marker darfst du auch zu unterschiedlichen Stapeln legen. Sollten keine Embargo-Marker mehr vorhanden sein, benutze bitte einen adäquaten Ersatz. Wenn keine Fluchkarten mehr im Vorrat sind, haben die Embargo-Marker keinen E ekt.}
\end{tikzpicture}
\hspace{-1cm}
\begin{tikzpicture}
	\card
	\cardstrip
	\cardbanner{banner/orange.png}
	\cardicon{banner/coin.png}
	\cardprice{2}
	\cardtitle{Hafen}
	\cardcontent{ Du ziehst zuerst eine Karte nach, dann wählst du eine Karte aus deiner Hand und legst diese verdeckt zur Seite auf den ausgespielten Hafen. Du musst deinen Mitspielern die zur Seite gelegte Karte nicht zeigen. Du musst eine Karte aus deiner Hand zur Seite legen, wenn du den Hafen ausspielst. Der Hafen und die zur Seite gelegte Karte bleiben bis zum Beginn deines nächsten Zuges liegen. Bei Beginn deines nächsten Zuges nimmst du die zur Seite gelegte Karte auf die Hand. Den Hafen legst du in der Aufräumphase dieses nächsten Zuges ab.}
\end{tikzpicture}
\hspace{-1cm}
\begin{tikzpicture}
	\card
	\cardstrip
	\cardbanner{banner/orange.png}
	\cardicon{banner/coin.png}
	\cardprice{2}
	\cardtitle{Leuchtturm}
	\cardcontent{Du erhältst +1 Aktion und +1 Geld in diesem Zug. Bei Beginn deines nächsten Zuges erhältst du nochmals +1 Geld. Solange der Leuchtturm o en vor dir ausliegt, bist du von Angriffskarten deiner Mitspieler nicht betroffen, selbst wenn du das möchtest. Du kannst trotz ausliegendem Leuchtturm auch mit der Geheimkammer oder anderen Reaktionskarten auf Angriffe reagieren, um deren Vorteil zu erhalten. Angriffskarten, die du selbst ausspielst, werden durch den Leuchtturm nicht abgewehrt. Der Leuchtturm bleibt bis zur Aufräumphase deines nächsten Zuges im Spiel.}
\end{tikzpicture}
\hspace{-1cm}
\begin{tikzpicture}
	\card
	\cardstrip
	\cardbanner{banner/white.png}
	\cardicon{banner/coin.png}
	\cardprice{2}
	\cardtitle{Perlentaucher}
	\cardcontent{Zu ziehst zuerst eine Karte nach, bevor du dir die unterste Karte deines Nachziehstapels ansiehst. Nimm die unterste Karte so vom Stapel, dass du nicht auch die nächste Karte sehen kannst.}
\end{tikzpicture}
\hspace{-1cm}
\begin{tikzpicture}
	\card
	\cardstrip
	\cardbanner{banner/white.png}
	\cardicon{banner/coin.png}
	\cardprice{3}
	\cardtitle{Ausguck}
	\cardcontent{Sieh dir alle 3 Karten an, bevor du entscheidest, welche Karte du entsorgst, welche du ablegst und welche du zurück auf den Nachziehstapel legst. Wenn du weniger als 3 Karten im Nachziehstapel hast (auch nach dem eventuell nötigen Mischen deines Ablagestapels), führst du die Anweisungen in der Reihenfolge auf der Karte aus, d. h. zuerst entsorgst du eine Karte, dann legst du eine Karte ab.}
\end{tikzpicture}
\hspace{-1cm}
\begin{tikzpicture}
	\card
	\cardstrip
	\cardbanner{banner/white.png}
	\cardicon{banner/coin.png}
	\cardprice{3}
	\cardtitle{Botschafter}
	\cardcontent{Du wählst 1 Karte aus deiner Hand, zeigst sie deinen Mitspielern und nimmst sie auf die Hand zurück. Dann darfst du bis zu 2 dieser Karten aus deiner Hand zurück in den Vorrat legen. Du darfst jedoch auch entscheiden, keine Karte zurück in den Vorrat zu legen. Danach muss sich jeder Mitspieler, beginnend mit dem Spieler
	zu deiner Linken, eine solche Karte aus dem Vorrat nehmen. Wenn der Stapel leer ist, werden keine Karten mehr genommen. Wenn du nach dem Ausspielen des Botschafters keine Karte mehr auf der Hand hast, legst du nichts zurück in den Vorrat und die Mitspieler nehmen auch keine Karten. Wenn zu irgendeinem Zeitpunkt während des Zuges entweder der dritte Stapel oder der Provinzstapel leer wird, endet das Spiel nach diesem Zug, auch wenn der Stapel später wieder aufgefüllt wird. 
	\\
	\smallskip
	\\
	Der Botschafter darf keine Unterschlupf-Karten (Dominion – Dark Ages) zurücklegen.}
\end{tikzpicture}
\hspace{-1cm}
\begin{tikzpicture}
	\card
	\cardstrip
	\cardbanner{banner/orange.png}
	\cardicon{banner/coin.png}
	\cardprice{3}
	\cardtitle{Fischerdorf}
	\cardcontent{Du erhältst +2 Aktionen und +1 Geld in diesem Zug. Bei Beginn deines nächsten Zuges erhältst du nochmals +1 Aktion und +1 Geld. Das Fischerdorf bleibt bis zur Aufräumphase deines nächsten Zuges im Spiel.}
\end{tikzpicture}
\hspace{-1cm}
\begin{tikzpicture}
	\card
	\cardstrip
	\cardbanner{banner/white.png}
	\cardicon{banner/coin.png}
	\cardprice{3}
	\cardtitle{Lagerhaus}
	\cardcontent{Wenn du (auch nach dem eventuell nötigen Mischen deines Ablagestapels) nicht mehr genügend Karten nachziehen kannst, ziehst du so viele Karten wie möglich. Du musst dann 3 Karten ablegen, auch wenn du weniger Karten nachgezogen hast. Wenn du weniger als 3 Karten auf der Hand hast, legst du alle Handkarten ab.}
\end{tikzpicture}
\hspace{-1cm}
\begin{tikzpicture}
	\card
	\cardstrip
	\cardbanner{banner/white.png}
	\cardicon{banner/coin.png}
	\cardprice{3}
	\cardtitle{Schmuggler}
	\cardcontent{Es ist egal, auf welche Weise der Spieler rechts von dir die Karte genommen oder gekauft hat (z. B. ebenfalls durch die Schmuggler). Wenn der Spieler rechts von dir mehrere Karten genommen bzw. gekauft hat, darfst du wählen, welche Karte du dir nimmst. Gibt es keine entsprechende Karte mehr im Vorrat, erhältst du nichts. Die Schmuggler sind kein Angriff, sie können nicht durch den Leuchtturm oder den Burggraben gestoppt werden. Diese Karte bezieht sich nie auf deinen eigenen vorherigen Zug. Wenn der Spieler rechts von dir im vorherigen Zug keine Karte genommen oder gekauft hat, die 6 Geld oder weniger gekostet hat, haben die Schmuggler keine Wirkung.}
\end{tikzpicture}
\hspace{-1cm}
\begin{tikzpicture}
	\card
	\cardstrip
	\cardbanner{banner/white.png}
	\cardicon{banner/coin.png}
	\cardprice{4}
	\cardtitle{\scriptsize{Beutelschneider}}
	\cardcontent{Deine Mitspieler müssen 1 Kupferkarte aus ihrer Hand ablegen. Hat ein Mitspieler kein Kupfer auf der Hand, muss er seine Kartenhand vorzeigen.}
\end{tikzpicture}
\hspace{-1cm}
\begin{tikzpicture}
	\card
	\cardstrip
	\cardbanner{banner/whitegreen.png}
	\cardicon{banner/coin.png}
	\cardprice{4}
	\cardtitle{Insel}
	\cardcontent{Die Insel ist zugleich eine Aktions- und eine Punktekarte. Sobald du deine erste Insel nimmst oder kaufst, nimmst du auch ein Tableau Insel zu dir. Wenn du die Karte Insel ausspielst, legst du die ausgespielte Insel und eine weitere Karte aus deiner Hand o en zur Seite auf das Tableau Insel, wo sie bis zum Spielende verbleiben. Wenn du keine weitere Karte auf der Hand hast, nachdem du die Insel ausgespielt hast, legst du nur die Insel zur Seite. Wenn du die Insel auf einen Thronsaal ausspielst, legst du zunächst die Insel und eine weitere Karte zur Seite, dann legst du noch eine Karte aus deiner Hand zur Seite auf das Tableau Insel. Du und auch deine Mitspieler dürfen die zur Seite gelegten Karten auf dem Tableau Insel jederzeit ansehen. Bei Spielende nimmst du alle Karten, die auf dem Tableau Insel liegen, zu deinen Karten. Bei Spielende ist die Insel 2 Punkte wert. Im Spiel mit 3 oder 4 Spielern werden 12 Inseln verwendet, im Spiel mit 2 Spielern nur 8.}
\end{tikzpicture}
\hspace{-1cm}
\begin{tikzpicture}
	\card
	\cardstrip
	\cardbanner{banner/orange.png}
	\cardicon{banner/coin.png}
	\cardprice{4}
	\cardtitle{Karawane}
	\cardcontent{Ziehe eine Karte bei Beginn deines nächsten Zuges. Die Karawane bleibt bis zur Aufräumphase dieses nächsten Zuges im Spiel.}
\end{tikzpicture}
\hspace{-1cm}
\begin{tikzpicture}
	\card
	\cardstrip
	\cardbanner{banner/white.png}
	\cardicon{banner/coin.png}
	\cardprice{4}
	\cardtitle{Müllverwerter}
	\cardcontent{Du musst eine Karte entsorgen, wenn du noch eine auf der Hand hast. Wenn du keine Karte auf der Hand hast, erhältst du nur den +1 Kauf, jedoch kein virtuelles Geld.}
\end{tikzpicture}
\hspace{-1cm}
\begin{tikzpicture}
	\card
	\cardstrip
	\cardbanner{banner/white.png}
	\cardicon{banner/coin.png}
	\cardprice{4}
	\cardtitle{Navigator}
	\cardcontent{Du legst entweder alle 5 Karten ab oder keine davon. Wenn du die Karten nicht ablegst, darfst du sie in beliebiger Reihenfolge zurück auf deinen Nachziehstapel legen. Wenn du (auch nach dem eventuell nötigen Mischen deines Ablagestapels) nicht mehr genügend Karten ansehen kannst, siehst du so viele Karten wie möglich an und entscheidest dann, ob du sie ablegst oder zurück auf deinen Nachziehstapel legst.}
\end{tikzpicture}
\hspace{-1cm}
\begin{tikzpicture}
	\card
	\cardstrip
	\cardbanner{banner/white.png}
	\cardicon{banner/coin.png}
	\cardprice{4}
	\cardtitle{Piratenschiff}
	\cardcontent{Sobald du dein erstes Piratenschiff nimmst oder kaufst, nimmst du auch ein Tableau Piratenschiff zu dir. Wenn du das Piratenschiff ausspielst, entscheidest du dich zunächst für eine der beiden Anweisungen. Entweder wählst du die Möglichkeit, Geldkarten deiner Mitspieler zu entsorgen oder +X Geld . Wählst du die erste Möglichkeit, muss jeder Mitspieler höchstens eine Geldkarte entsorgen. Hat ein Mit- spieler mehrere Geldkarten aufgedeckt, wählst du aus, welche davon er entsorgen muss. Hat mindestens ein Mitspieler eine Geldkarte entsorgt, erhältst du einen Geld-Marker, den du auf dein Tableau Piratenschiff legst. Du erhältst höchstens einen Geld-Marker, auch wenn mehrere Mitspieler Geldkarten entsorgen mussten. Wählst du die zweite Möglichkeit, erhältst du +1 Geld für jeden Geld-Marker auf deinem Tableau Piratenschiff. Die Geld-Marker bleiben bis zum Spielende auf dem Tableau. Wenn du das Piratenschiff z. B. 3mal (erfolgreich) verwendet hast, um Geldkarten deiner Mitspieler zu entsorgen, hast du 3 Geld-Marker auf deinem Tableau Piratenschiff. Damit kannst du jedesmal, wenn du das Piratenschiff ausspielst +3 Geld erhalten. Das Piratenschiff ist eine Angriffskarte, deine Mitspieler können also z. B. mit der Geheimkammer reagieren, auch wenn du +X Geld wählst.}
\end{tikzpicture}
\hspace{-1cm}
\begin{tikzpicture}
	\card
	\cardstrip
	\cardbanner{banner/white.png}
	\cardicon{banner/coin.png}
	\cardprice{4}
	\cardtitle{Schatzkarte}
	\cardcontent{Du kannst diese Karte ausspielen, auch wenn du keine weitere Schatzkarte auf der Hand hast. In diesem Fall entsorgst du die ausgespielte Schatzkarte, erhältst jedoch nichts dafür. Du musst wirklich 2 Schatzkarten entsorgen, um dir die 4 Gold zu nehmen. Wenn du z. B. eine Schatzkarte auf einen Thronsaal ausspielst und noch 2 weitere Schatzkarten auf der Hand hast, entsorgst du zuerst die ausgespielte Schatzkarte und eine weitere Schatzkarte aus deiner Hand. Beim zweiten Ausspielen entsorgst du die andere Schatzkarte aus deiner Hand, erhältst jedoch nichts, da du nur eine Schatzkarte entsorgt hast. Die ausgespielte Schatzkarte hast du bereits beim ersten Ausspielen entsorgt. Wenn weniger als 4 Gold im Vorrat sind, nimmst du dir die übrigen Gold. Die Karten legst du verdeckt auf deinen Nachziehstapel.}
\end{tikzpicture}
\hspace{-1cm}
\begin{tikzpicture}
	\card
	\cardstrip
	\cardbanner{banner/white.png}
	\cardicon{banner/coin.png}
	\cardprice{4}
	\cardtitle{Seehexe}
	\cardcontent{Ein Spieler, dessen Nachziehstapel leer ist, mischt zuerst seinen Ablagestapel und legt dann die oberste Karte von seinem neuen Nachziehstapel ab. Danach müssen alle Mitspieler, beginnend bei dem Spieler links vom Angreifer, einen Fluch aus dem Vorrat nehmen und diesen verdeckt auf ihren Nachziehstapel legen. Gibt es nicht mehr genügend Fluchkarten im Vorrat, werden die übrigen in Spielerreihenfolge verteilt.}
\end{tikzpicture}
\hspace{-1cm}
\begin{tikzpicture}
	\card
	\cardstrip
	\cardbanner{banner/orange.png}
	\cardicon{banner/coin.png}
	\cardprice{5}
	\cardtitle{Außenposten}
	\cardcontent{Der Extrazug ist wie ein normaler Zug, mit dem Unterschied, dass du diesen Zug
	nur mit 3 Karten ausführst. Du ziehst also in der Aufräumphase des Zuges, in
	dem du den Aussenposten ausspielst nur 3, statt den üblichen 5 Karten. Lass den Aussenposten offen vor dir liegen, bis zum Ende des Extrazuges. Wenn du den Aussenposten zusammen mit einer \enquote{zu Beginn deines nächsten Zuges}-Karte ausspielst, wie z. B. das Handelsschiff, ist der Extrazug vom Handelsschiff dieser nächste Zug, in dem du +2 Geld bekommst. Wenn du den Aussenposten mehr als einmal ausspielst, bekommst du trotzdem nur einen Extrazug. Wenn du während des Extrazuges einen Aussenposten ausspielst, bekommst du keinen weiteren Extrazug. Am Ende deines Extrazuges ziehst du wieder 5 Karten nach.}
\end{tikzpicture}
\hspace{-1cm}
\begin{tikzpicture}
	\card
	\cardstrip
	\cardbanner{banner/white.png}
	\cardicon{banner/coin.png}
	\cardprice{5}
	\cardtitle{Bazar}
	\cardcontent{Du ziehst eine Karte nach, darfst 2 weitere Aktionen ausführen und erhältst für die Kaufphase + 1 Geld.}
\end{tikzpicture}
\hspace{-1cm}
\begin{tikzpicture}
	\card
	\cardstrip
	\cardbanner{banner/white.png}
	\cardicon{banner/coin.png}
	\cardprice{5}
	\cardtitle{Entdecker}
	\cardcontent{Wenn du eine Provinz aus deiner Hand aufdeckst, nimmst du eine Goldkarte direkt auf die Hand. Wenn du keine Provinz auf der Hand hast oder die Provinz nicht aufdecken möchtest, nimmst du dir ein Silber direkt auf die Hand.}
\end{tikzpicture}
\hspace{-1cm}
\begin{tikzpicture}
	\card
	\cardstrip
	\cardbanner{banner/white.png}
	\cardicon{banner/coin.png}
	\cardprice{5}
	\cardtitle{Geisterschiff}
	\cardcontent{Deine Mitspieler entscheiden, welche Karten sie aus ihrer Hand zurück auf den Nachziehstapel legen und in welcher Reihenfolge sie die Karten zurücklegen. Spieler, die nur 3 oder weniger Karten auf der Hand halten, sind nicht betroffen.}
\end{tikzpicture}
\hspace{-1cm}
\begin{tikzpicture}
	\card
	\cardstrip
	\cardbanner{banner/orange.png}
	\cardicon{banner/coin.png}
	\cardprice{5}
	\cardtitle{Handelsschiff}
	\cardcontent{Du erhältst +2 Geld für die Kaufphase in diesem Zug und nochmals +2 Geld für die Kaufphase in deinem nächsten Zug. Das Handelsschiff bleibt bis zur Aufräumphase dieses nächsten Zuges im Spiel.}
\end{tikzpicture}
\hspace{-1cm}
\begin{tikzpicture}
	\card
	\cardstrip
	\cardbanner{banner/white.png}
	\cardicon{banner/coin.png}
	\cardprice{5}
	\cardtitle{Schatzkammer}
	\cardcontent{Wenn du eine oder mehrere Karten kaufst und mindestens eine davon eine Punktekarte ist, darfst du die Schatzkammer nicht zurück auf den Nachziehstapel legen. Hast du mehrere Schatzkammern ausgespielt und in diesem Zug keine Punktekarte gekauft, darfst du in der Aufräumphase beliebig viele deiner ausgespielten Schatzkammern zurück auf deinen Nachziehstapel legen. Wenn du vergisst, die Schatzkammer auf deinen Nachziehstapel zu legen, hast du dich dafür entschieden, die Karte abzulegen. Du darfst die Schatzkammer nicht nachträglich aus dem Ablagestapel heraussuchen, um sie noch zurück auf den Nachziehstapel zu legen. Wenn du eine Punktekarte nimmst, sie aber nicht gekauft hast (z. B. durch die Schmuggler), darfst du die Schatzkammer zurück auf deinen Nachziehstapel legen.}
\end{tikzpicture}
\hspace{-1cm}
\begin{tikzpicture}
	\card
	\cardstrip
	\cardbanner{banner/orange.png}
	\cardicon{banner/coin.png}
	\cardprice{5}
	\cardtitle{Taktiker}
	\cardcontent{Du ziehst die 5 Karten erst bei Beginn deines nächsten Zuges. Du ziehst die Karten nicht in der Aufräumphase des Zuges, in dem du den Taktiker ausgespielt hast. Der Taktiker bleibt bis zur Aufräumphase deines nächsten Zuges o en vor dir ausliegen. Wenn du den Taktiker nach einem Thronsaal ausspielst, erhältst du den Bonus nur einmal, da du auch beim 2. Mal mindestens eine Karte ablegen müsstest. Du kannst jedoch nach dem 1. Ausspielen keine Karten mehr auf der Hand haben.}
\end{tikzpicture}
\hspace{-1cm}
\begin{tikzpicture}
	\card
	\cardstrip
	\cardbanner{banner/orange.png}
	\cardicon{banner/coin.png}
	\cardprice{5}
	\cardtitle{Werft}
	\cardcontent{Du ziehst zuerst 2 Karten nach und erhältst einen weiteren Kauf für diesen Zug. Bei Beginn deines nächsten Zuges ziehst du 2 weitere Karten und erhältst nochmals einen zusätzlichen Kauf. Du ziehst die beiden Karten nicht, bevor dein nächster Zug begonnen hat. Die Werft bleibt bis zur Aufräumphase deines nächsten Zuges im Spiel.}
\end{tikzpicture}
\hspace{-1cm}
\begin{tikzpicture}
	\card
	\cardstrip
	\cardbanner{banner/white.png}
	\cardtitle{\scriptsize{Empfohlene 10er Sätze\qquad}}
	\cardcontent{\emph{Blütezeit}\\ 
		\smallskip \emph{Auf hoher See:} \\ 
		Embargo, Hafen, Ausguck, Schmuggler, Insel, Karawane, Piratenschiff, Bazar, Entdecker, Werft \\
		\smallskip \emph{Geheime Pläne:} \\ 
		Leuchtturm, Perlentaucher, Botschafter, Fischerdorf, Lagerhaus, Beutelschneider, Außenposten, Schatzkarte, Taktiker, Werft \\
		\smallskip \emph{Schiffswracks:} \\ 
		Eingeborenendorf, Perlentaucher, Lagerhaus, Schmuggler, Müllverwerter, Navigator, Seehexe, Geisterschiff, Handelsschiff, Schatzkammer \\
	}
\end{tikzpicture}
\hspace{-1cm}
\begin{tikzpicture}
	\card
	\cardstrip
	\cardbanner{banner/white.png}
	\cardtitle{\scriptsize{Empfohlene 10er Sätze\qquad}}
	\cardcontent{\emph{Blütezeit und Basisspiel:}\\ 
		\smallskip \emph{Griff nach den Sternen:} \\ 
		Keller, Ausguck, Dorf, Beutelschneider, Seehexe, Spion, Geisterschiff, Ratsversammlung, Schatzkarte, Abenteurer
		\smallskip \emph{Wiederholungen:} \\ 
		Perlentaucher, Kanzler, Werkstatt, Karawane, Miliz, Piratenschiff, Außenposten, Entdecker, Jahrmarkt, Schatzkammer
		\smallskip \emph{Geben und Nehmen:} \\ 
		Hafen, Botschafter, Fischerdorf, Schmuggler, Geldverleiher, Insel, Müllverwerter, Bibliothek, Hexe, Markt
	}
\end{tikzpicture}
\hspace{1cm}