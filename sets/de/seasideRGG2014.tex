% Basic settings for this card set
\renewcommand{\cardcolor}{seaside}
\renewcommand{\cardextension}{Erweiterung I}
\renewcommand{\cardextensiontitle}{Seaside}

\clearpage
\newpage
\section{\cardextension \ - \cardextensiontitle}

\begin{tikzpicture}
	\card
	\cardstrip
	\cardbanner{banner/white.png}
	\cardicon{banner/coin.png}
	\cardprice{2}
	\cardtitle{\scriptsize{Eingeborenendorf}}
	\cardcontent{Wenn du dein erstes \emph{EINGEBORENENDORF} nimmst oder kaufst, erhältst du ein Eingeborenen-Tableau und legst es vor dir ab. 
	\\
	\medskip
	\\
	Immer wenn du ein \emph{EINGEBORENENDORF} ausspielst, wählst du genau eine der beiden Anweisungen und führst sie wenn möglich aus. Du darfst eine Anweisung auch wählen, wenn du sie nicht ausführen kannst. Karten, die du auf das Tableau legst, werden immer verdeckt abgelegt. Du darfst dir jederzeit die Karten auf deinem Tableau ansehen. 
	\\
	\medskip
	\\
	Die ausgespielte Aktionskarte \emph{EINGEBORENENDORF} legst du in der Aufräumphase ab. Alle Karten auf dem Tableau gehören auch zum Kartensatz eines Spielers. Alle Karten auf den Tableaus werden bei Spielende mit berücksichtigt.}
\end{tikzpicture}
\hspace{-1cm}
\begin{tikzpicture}
	\card
	\cardstrip
	\cardbanner{banner/white.png}
	\cardicon{banner/coin.png}
	\cardprice{2}
	\cardtitle{Embargo}
	\cardcontent{Wenn du das \emph{EMBARGO} in deiner Aktionsphase ausspielst, musst du es entsorgen und einen Embargomarker auf einen beliebigen Vorratsstapel (Königreichkarten \emph{und} Geldkarten sind erlaubt) legen. Um die +\coin{2} in der Kaufphase nicht zu „vergessen“, empfehlen wir, das entsorgte \emph{EMBARGO} zunächst separat neben den Müllstapel zu legen und erst in der Aufräumphase endgültig zu entsorgen.
	\\
	\medskip
	\\Wenn du das \emph{EMBARGO} auf einen \emph{THRONSAAL} folgend ausspielst, legst du 2 Embargomarker. Du kannst sie auf denselben oder unterschiedliche Stapel legen. Wenn keine Embargomarker mehr vorhanden sind, benutze einen geeigneten Ersatz (z.B. echte Geldmünzen). Auf jeden Vorratsstapel dürfen beliebig viele Embargomarker gelegt werden. 
	\\
	\medskip
	\\
	Spieler, die Karten von einem Vorratsstapel mit einem oder mehreren Embargomarkern kaufen, müssen pro Marker eine Fluchkarte nehmen. Wer Karten von einem Stapel mit Embargomarker(n) auf eine andere Weise nimmt (z. B. durch den \emph{SCHMUGGLER}), nimmt \emph{keine} Fluchkarte. Wenn keine Fluchkarten mehr vorrätig sind, haben die Marker keinen Effekt.}
\end{tikzpicture}
\hspace{-1cm}
\begin{tikzpicture}
	\card
	\cardstrip
	\cardbanner{banner/orange.png}
	\cardicon{banner/coin.png}
	\cardprice{2}
	\cardtitle{Hafen}
	\cardcontent{Der \emph{HAFEN} ist eine Dauerkarte. Lege eine Handkarte verdeckt auf den \emph{HAFEN}. Diese und der \emph{HAFEN} werden in der Aufräumphase nicht abgelegt. Nimm zu Beginn deines nächsten Zuges die zur Seite gelegte Karte auf die Hand. Lege den \emph{HAFEN} in der Aufräumphase ab.}
\end{tikzpicture}
\hspace{-1cm}
\begin{tikzpicture}
	\card
	\cardstrip
	\cardbanner{banner/orange.png}
	\cardicon{banner/coin.png}
	\cardprice{2}
	\cardtitle{Leuchtturm}
	\cardcontent{Der \emph{LEUCHTTURM} ist eine Dauerkarte. Solange der \emph{LEUCHTTURM} offen vor dir liegt (im Spielbereich oder darüber), bist du grundsätzlich nicht betroffen, wenn Mitspieler Angriffskarten ausspielen (sogar wenn du das möchtest). Selber ausgespielte Angriffskarten werden vom \emph{LEUCHTTURM} nicht abgewehrt. Auf Angriffe von Mitspielern darfst du weiterhin zusätzlich Reaktionskarten ausspielen. Lege den \emph{LEUCHTTURM} in der Aufräumphase des nächsten Zuges ab.}
\end{tikzpicture}
\hspace{-1cm}
\begin{tikzpicture}
	\card
	\cardstrip
	\cardbanner{banner/white.png}
	\cardicon{banner/coin.png}
	\cardprice{2}
	\cardtitle{Perlentaucher}
	\cardcontent{Zieh die unterste Karte des Nachziehstapels so hervor, dass du die benachbarte Karte nicht sehen kannst. Schaue sie dir an und lege sie dann verdeckt oben auf den Nachziehstapel \emph{oder} zurück unter den Nachziehstapel.}
\end{tikzpicture}
\hspace{-1cm}
\begin{tikzpicture}
	\card
	\cardstrip
	\cardbanner{banner/white.png}
	\cardicon{banner/coin.png}
	\cardprice{3}
	\cardtitle{Ausguck}
	\cardcontent{Sieh dir erst alle 3 Karten an, bevor du die Anweisungen ausführst. Solltest du weniger als 3 Karten im Nachziehstapel haben, auch nachdem du ggf. den Ablagestapel gemischt hast, führst du die Anweisungen der Reihenfolge nach aus. Anweisungen, für die es keine Karten mehr im Stapel gibt, entfallen.}
\end{tikzpicture}
\hspace{-1cm}
\begin{tikzpicture}
	\card
	\cardstrip
	\cardbanner{banner/white.png}
	\cardicon{banner/coin.png}
	\cardprice{3}
	\cardtitle{Botschafter}
	\cardcontent{Wähle eine beliebige Handkarte und zeige sie deinen Mitspielern. Du darfst dann bis zu 2 dieser Karten von deiner Hand zurück in den Vorrat legen. Jeder Mitspieler nimmt sich anschließend eine solche Karte aus dem Vorrat (reihum im Uhrzeigersinn, beginnend beim linken Mitspieler). Der ausgespielte Botschafter kann nicht in den Vorrat zurückgelegt werden.}
\end{tikzpicture}
\hspace{-1cm}
\begin{tikzpicture}
	\card
	\cardstrip
	\cardbanner{banner/orange.png}
	\cardicon{banner/coin.png}
	\cardprice{3}
	\cardtitle{Fischerdorf}
	\cardcontent{Das \emph{FISCHERDORF} ist eine Dauerkarte. Du \emph{darfst} 2 weitere Aktionen ausführen und erhältst für die Kaufphase +\coin{1}. 
	\\
	\medskip
	\\
	In deinem nächsten Zug \emph{darfst} du eine weitere Aktion ausführen und erhältst für die Kaufphase +\coin{1}.}
\end{tikzpicture}
\hspace{-1cm}
\begin{tikzpicture}
	\card
	\cardstrip
	\cardbanner{banner/white.png}
	\cardicon{banner/coin.png}
	\cardprice{3}
	\cardtitle{Lagerhaus}
	\cardcontent{Ziehe 3 Karten und spiele dann eine Aktionskarte. Danach legst du 3 Handkarten ab. Wenn du weniger als 3 Karten auf der Hand hast, legst du alle Handkarten ab.}
\end{tikzpicture}
\hspace{-1cm}
\begin{tikzpicture}
	\card
	\cardstrip
	\cardbanner{banner/white.png}
	\cardicon{banner/coin.png}
	\cardprice{3}
	\cardtitle{Schmuggler}
	\cardcontent{Hat der rechts von dir sitzende Mitspieler in seinem letzten Zug eine Karte mit Kosten von \coin{6} oder weniger genommen, gekauft oder auf andere Art erhalten, nimmst du dir eine gleiche Karte vom Vorrat. Hat der Spieler mehrere Karten genommen, darfst du wählen, welche du nimmst. Da der \emph{SCHMUGGLER} keine Angriffskarte ist, dürfen keine Reaktionskarten ausgespielt werden.}
\end{tikzpicture}
\hspace{-1cm}
\begin{tikzpicture}
	\card
	\cardstrip
	\cardbanner{banner/white.png}
	\cardicon{banner/coin.png}
	\cardprice{4}
	\cardtitle{\scriptsize{Beutelschneider}}
	\cardcontent{Alle Mitspieler müssen eine Kupferkarte aus der Hand ablegen. Da der Beutelschneider eine Angriffskarte ist, dürfen die Mitspieler mit einer Reaktionskarte auf diesen Angriff reagieren.}
\end{tikzpicture}
\hspace{-1cm}
\begin{tikzpicture}
	\card
	\cardstrip
	\cardbanner{banner/whitegreen.png}
	\cardicon{banner/coin.png}
	\cardprice{4}
	\cardtitle{Insel}
	\cardcontent{Die \emph{INSEL} ist eine kombinierte Aktions- und Punktekarte. Sie kann in der Aktionsphase eingesetzt werden und bringt zusätzlich bei Spielende 2 Punkte. Wenn du deine erste \emph{INSEL} nimmst oder kaufst, erhältst du ein Insel-Tableau und legst es vor dir ab. 
	\\
	\medskip
	\\
	Immer wenn du eine \emph{INSEL} ausspielst, legst du die ausgespielte \emph{INSEL} und eine beliebige Handkarte offen auf dein Insel-Tableau. Dort verbleiben sie bis zum Spielende. Wenn du mindestens eine Karte auf der Hand hast, musst du eine Handkarte auf dein Insel-Tableau legen. Wenn du keine Karte auf der Hand hast, nachdem du die \emph{INSEL} ausgespielt hast, legst du nur die \emph{INSEL} auf das Tableau. 
	\\
	\medskip
	\\
	Bei Spielende nimmst du alle Karten vom Insel-Tableau zu deinen Karten.}
\end{tikzpicture}
\hspace{-1cm}
\begin{tikzpicture}
	\card
	\cardstrip
	\cardbanner{banner/orange.png}
	\cardicon{banner/coin.png}
	\cardprice{4}
	\cardtitle{Karawane}
	\cardcontent{Die \emph{KARAWANE} ist eine Dauerkarte. Sie wird in der Aufräumphase nicht abgelegt. Ziehe zu Beginn des nächsten Zuges eine Karte und lege die \emph{KARAWANE} in der Aufräumphase dieses Zuges ab.}
\end{tikzpicture}
\hspace{-1cm}
\begin{tikzpicture}
	\card
	\cardstrip
	\cardbanner{banner/white.png}
	\cardicon{banner/coin.png}
	\cardprice{4}
	\cardtitle{Müllverwerter}
	\cardcontent{Du musst eine Karte entsorgen, sofern du eine auf der Hand hast. Entsprechend der Kosten der entsorgten Karte erhältst du für die Kaufphase +\emph{X}. Wenn du keine Karte entsorgen kannst, erhältst du kein zusätzliches Geld.}
\end{tikzpicture}
\hspace{-1cm}
\begin{tikzpicture}
	\card
	\cardstrip
	\cardbanner{banner/white.png}
	\cardicon{banner/coin.png}
	\cardprice{4}
	\cardtitle{Navigator}
	\cardcontent{Schau dir die obersten 5 Karten deines Nachziehstapels an. Sind nicht genügend Karten im Stapel, mischst du deinen Ablagestapel und legst ihn verdeckt unter deinen Nachziehstapel. Sind es nun immer noch weniger als 5 Karten, schaust du dir alle an und legst sie dann entweder ab oder in einer beliebigen Reihenfolge zurück auf den Nachziehstapel.}
\end{tikzpicture}
\hspace{-1cm}
\begin{tikzpicture}
	\card
	\cardstrip
	\cardbanner{banner/white.png}
	\cardicon{banner/coin.png}
	\cardprice{4}
	\cardtitle{Piratenschiff}
	\cardcontent{Wenn du dein erstes \emph{PIRATENSCHIFF} nimmst oder kaufst, erhältst du ein Piratenschiff-Tableau und legst es vor dir ab. 
	\\
	\medskip
	\\
	Immer wenn du ein \emph{PIRATENSCHIFF} ausspielst, wählst du \emph{eine} der beiden Anweisungen: 
	\\
	\smallskip
	\\
	Entweder die \emph{erste Anweisung}: Alle Mitspieler decken die beiden obersten Karten ihres Nachziehstapels auf. Dann entsorgen sie jeweils eine Geldkarte nach deiner Wahl. Hat ein Mitspieler keine Geldkarte aufgedeckt, entsorgt er keine Karte. Die restlichen aufgedeckten Karten werden abgelegt. Wird mindestens eine Karte entsorgt, erhältst du einen \emph{Geldmarker} und legst ihn auf dein Tableau; 
	\\
	\smallskip
	\\
	oder die \emph{zweite Anweisung}: Du erhältst pro \emph{Geldmarker} auf deinem Piratenschiff-Tableau in der Kaufphase +\coin{1}.
	\\
	\medskip
	\\
	Nach der Nutzung in der Kaufphase verbleiben die Geldmarker auf dem Tableau und können beim erneuten Ausspielen eines \emph{PIRATENSCHIFFES} wieder eingesetzt werden. Mitspieler können auf das Ausspielen eines \emph{PIRATENSCHIFFES} mit Reaktionskarten reagieren, auch wenn du die zweite Anweisung wählst, die deine Mitspieler nicht direkt betrifft.}
\end{tikzpicture}
\hspace{-1cm}
\begin{tikzpicture}
	\card
	\cardstrip
	\cardbanner{banner/white.png}
	\cardicon{banner/coin.png}
	\cardprice{4}
	\cardtitle{Schatzkarte}
	\cardcontent{Nur wenn du zusätzlich zu der ausgespielten \emph{SCHATZKARTE} noch eine weitere auf der Hand hast und beide entsorgst, erhältst du 4 Gold. Sollten weniger als  4 Gold im Vorrat sein, nimmst du dir soviele Goldkarten wie vorhanden sind. Lege alle auf diese Weise erhaltenen Goldkarten verdeckt auf den Nachziehstapel. Solltest du nur eine \emph{SCHATZKARTE} auf der Hand haben und diese ausspielen, musst du diese Karte entsorgen, erhältst aber nichts dafür.}
\end{tikzpicture}
\hspace{-1cm}
\begin{tikzpicture}
	\card
	\cardstrip
	\cardbanner{banner/white.png}
	\cardicon{banner/coin.png}
	\cardprice{4}
	\cardtitle{Seehexe}
	\cardcontent{Sollte der Nachziehstapel eines Mitspielers leer sein, mischt er seinen Ablagestapel und legt die oberste Karte des neuen Nachziehstapels ab. Hat ein Spieler keine Karten mehr in seinem Nachziehstapel, kann er zwar keine Karte ablegen, nimmt sich aber trotzdem eine Fluchkarte. Beginnend mit dem Mitspieler links von dem Spieler, der die \emph{SEEHEXE} ausgespielt hat, nimmt sich jeder Mitspieler einen \emph{FLUCH} vom Vorrat. Sollten nicht mehr genügend Fluchkarten für alle Spieler vorhanden sein, werden die restlichen in o. g. Reihenfolge verteilt. }
\end{tikzpicture}
\hspace{-1cm}
\begin{tikzpicture}
	\card
	\cardstrip
	\cardbanner{banner/orange.png}
	\cardicon{banner/coin.png}
	\cardprice{5}
	\cardtitle{Außenposten}
	\cardcontent{Der \emph{AUSSENPOSTEN} ist eine Dauerkarte, die bis zum Ende des nächsten Zuges (Extrazug) im Spiel bleibt und erst in der Aufräumphase des nächsten Zuges (Extrazug) abgelegt wird. Der \emph{AUSSENPOSTEN} kommt erst in der Aufräumphase des Zuges, in dem er ausgespielt wird, zum Einsatz. Du ziehst in diesem Fall nur 3 statt 5 Karten nach und führst den Extrazug \emph{sofort} aus. 
	\\
	\medskip
	\\
	Wenn du den \emph{AUSSENPOSTEN} zusammen mit weiteren Dauerkarten ausgespielt hast, kommen die \enquote{Zu Beginn deines nächsten Zuges}-Anweisungen der Dauerkarten in deinem Extrazug zum Einsatz. Spielst du in deinem Extrazug einen weiteren \emph{AUSSENPOSTEN}, erhältst du keinen weiteren Extrazug. Am Ende deines Extrazuges legst du den \emph{AUSSENPOSTEN} ab und ziehst 5 Karten nach.}
\end{tikzpicture}
\hspace{-1cm}
\begin{tikzpicture}
	\card
	\cardstrip
	\cardbanner{banner/white.png}
	\cardicon{banner/coin.png}
	\cardprice{5}
	\cardtitle{Bazar}
	\cardcontent{Du \emph{musst} eine Karte nachziehen, \emph{darfst} 2 weitere Aktionen ausführen und erhältst für die Kaufphase +\coin{2}.}
\end{tikzpicture}
\hspace{-1cm}
\begin{tikzpicture}
	\card
	\cardstrip
	\cardbanner{banner/white.png}
	\cardicon{banner/coin.png}
	\cardprice{5}
	\cardtitle{Entdecker}
	\cardcontent{Wenn du eine Provinz aus der Hand aufdeckst, erhältst du ein Gold. Wenn du das nicht tun kannst (weil du keine Provinz auf der Hand hast) oder willst (weil du deine Provinz nicht zeigen möchtest), erhältst du ein Silber. Nimm das Gold oder Silber auf die Hand.}
\end{tikzpicture}
\hspace{-1cm}
\begin{tikzpicture}
	\card
	\cardstrip
	\cardbanner{banner/white.png}
	\cardicon{banner/coin.png}
	\cardprice{5}
	\cardtitle{Geisterschiff}
	\cardcontent{Deine Mitspieler müssen Karten aus ihrer Hand verdeckt auf den Nachziehstapel legen, bis sie nur noch 3 Karten auf der Hand haben. Welche Karten sie auf den Nachziehstapel legen, entscheiden die Mitspieler selbst. Spieler, die zum Zeitpunkt des Angriffs bereits 3 oder weniger Karten auf der Hand haben, müssen keine Karten auf den Nachziehstapel legen. }
\end{tikzpicture}
\hspace{-1cm}
\begin{tikzpicture}
	\card
	\cardstrip
	\cardbanner{banner/orange.png}
	\cardicon{banner/coin.png}
	\cardprice{5}
	\cardtitle{Handelsschiff}
	\cardcontent{Das \emph{HANDELSSCHIFF} ist eine Dauerkarte. Du erhältst für deine Kaufphase +\coin{2}. 
	\\
	\medskip
	\\
	Zu Beginn deines nächsten Zuges erhältst du +\coin{2} für die Kaufphase. Lege das \emph{HANDELSSCHIFF} in der Aufräumphase dieses Zuges ab.}
\end{tikzpicture}
\hspace{-1cm}
\begin{tikzpicture}
	\card
	\cardstrip
	\cardbanner{banner/white.png}
	\cardicon{banner/coin.png}
	\cardprice{5}
	\cardtitle{Schatzkammer}
	\cardcontent{Wenn du eine \emph{SCHATZKAMMER} spielst und in diesem Zug \emph{keine} Punktekarte gekauft hast, \emph{darfst} du die ausgespielte \emph{SCHATZKAMMER} in der Aufräumphase zurück auf den Nachziehstapel legen. 
	\\
	\medskip
	\\
	Wenn du mehrere \emph{SCHATZKAMMERN} ausgespielt hast, darfst du auch diese \emph{SCHATZKAMMERN} auf den Nachziehstapel zurücklegen. Wenn du eine Punktekarte auf andere Art nimmst bzw. erhältst (d. h. \emph{nicht} kaufst), darfst du \emph{SCHATZKAMMERN} zurück auf den Nachziehstapel legen. 
	\\
	\medskip
	\\
	Wenn du deine ausgespielte \emph{SCHATZKAMMER} gern zurücklegen möchtest, das aber in der Aufräumphase vergisst und die Karte bereits auf den Ablagestapel gelegt hast, darfst du dies nachträglich \emph{nicht} rückgängig machen.}
\end{tikzpicture}
\hspace{-1cm}
\begin{tikzpicture}
	\card
	\cardstrip
	\cardbanner{banner/orange.png}
	\cardicon{banner/coin.png}
	\cardprice{5}
	\cardtitle{Taktiker}
	\cardcontent{Der \emph{TAKTIKER} ist eine Dauerkarte. Sobald du diese Karte ausspielst, legst du alle Handkarten ab. Nur wenn du auf diese Weise mindestens eine Handkarte abgelegt hast, ziehst du zu Beginn deines nächsten Zuges 5 Karten. Außerdem erhältst du dann im nächsten Zug eine zusätzliche Aktion und einen zusätzlichen Kauf. 
	\\
	\medskip
	\\
	\emph{Grundsätzlich gilt: Nur wenn du mindestens eine Handkarte ablegen kannst, erhältst du den Bonus im nächsten Zug.}
	\\
	\medskip
	\\
	Wenn du den \emph{TAKTIKER} auf einen \emph{THRONSAAL} spielst, erhältst du den Bonus im nächsten Zug nur einmal, da du beim zweiten Ausspielen des \emph{TAKTIKERS} keine Handkarte mehr auf der Hand hast und damit die Bedingung nicht erfüllst. }
\end{tikzpicture}
\hspace{-1cm}
\begin{tikzpicture}
	\card
	\cardstrip
	\cardbanner{banner/orange.png}
	\cardicon{banner/coin.png}
	\cardprice{5}
	\cardtitle{Werft}
	\cardcontent{ Die \emph{WERFT} ist eine Dauerkarte. Du \emph{musst} sofort 2 Karten nachziehen und \emph{darfst} einen weiteren Kauf tätigen. 
	\\
	\medskip
	\\
	Zu Beginn deines nächsten Zuges (nicht vorher) \emph{musst} du wieder 2 Karten ziehen und \emph{darfst} einen weiteren Kauf tätigen.}
\end{tikzpicture}
\hspace{-1cm}
\begin{tikzpicture}
	\card
	\cardstrip
	\cardbanner{banner/white.png}
	\cardicon{}
	\cardprice{}
	\cardtitle{\scriptsize{Empfohlene 10er Sätze\qquad}}
	\cardcontent{\emph{Auf hoher See:}
	\\
	Ausguck, Bazar, Embargo, Entdecker, Hafen, Insel, Karawane, Piratenschiff, Schmuggler, Werft
	\\
	\smallskip
	\\
	\emph{Vergrabene Schätze:}
	\\
	Außenposten, Beutelschneider, Botschafter, Fischerdorf, Lagerhaus, Leuchtturm, Perlentaucher, Schatzkarte, Taktiker, Werft
	\\
	\smallskip
	\\
	\emph{Schiffswracks:}
	\\
	Eingeborenen, Geisterschiff, Handelsschiff, Lagerhaus, Leuchtturm, Perlentaucher, Schatzkammer, Schmuggler, Seehexe
	\\
	\smallskip
	\\
	\emph{Griff nach den Sternen} (Seaside + \textit{Basisspiel}:)
	\\
	Ausguck, Beutelschneider, Geisterschiff, Schatzkarte, Seehexe, \textit{Abenteurer}, \textit{Dorf}, \textit{Keller}, \textit{Ratsversammlung}, \textit{Spion}
	\\
	\smallskip
	\\
	\emph{Wiederholungen:} (Seaside + \textit{Basisspiel}:)
	\\
	Außenposten, Entdecker, Karawane, Perlentaucher, Piratenschiff, Schatzkammer, \textit{Jahrmarkt}, \textit{Kanzler}, \textit{Miliz}, \textit{Werkstatt}
	\\
	\smallskip
	\\
	\emph{Geben und Nehmen:} (Seaside + \textit{Basisspiel}:)
	\\
	Botschafter, Fischerdorf, Hafen, Insel, Müllverwerter, Schmuggler, \textit{Bibliothek}, \textit{Geldverleiher}, \textit{Hexe}, \textit{Markt}
	\\
	}
\end{tikzpicture}
\hspace{1cm}